\chapter{Johdanto}

Analyyttinen geometria tarkastelee geometrisiä ongelmia matemaattisen analyysin ja algebran keinoin karteesisessa koordinaatistossa. Sen perusongelmia ovat kuvioiden ja kappaleiden yhtälöt, tasoleikkaukset, etäisyydet ja muodot.

Lukion pitkän matematiikan kurssilla MAA4 Analyyttinen geometria käydään läpi näitä ongelmia monipuolisesti. Oppikirja on rakennettu siten, että aiheet esitellään lukujen alussa ja havainnollistetaan esimerkein. Tehtäviä on runsaasti ja niiden tarkoituksena on saada opiskelija sisäistämään opiskellut asiat ja siirtämään ne käytäntöön.

Tässä kirjassa käsittelemme opetussuunnitelman mukaiset keskeiset sisällöt, joita ovat
\begin{itemize}
\item pistejoukon yhtälö
\item suoran, ympyrän ja paraabelin yhtälöt
\item itseisarvoyhtälön ja epäyhtälön ratkaiseminen
\item yhtälöryhmän ratkaiseminen
\item pisteen etäisyys suorasta
\end{itemize}

Opetussuunnitelman mukaiset kurssin keskeiset tavoitteet ovat, että opiskelija
\begin{itemize}
\item ymmärtää kuinka analyyttinen geometria luo yhteyksiä geometristen ja algebrallisten
käsitteiden välille 
\item ymmärtää pistejoukon yhtälön käsitteen ja oppii tutkimaan yhtälöiden avulla pisteitä,
suoria, ympyröitä ja paraabeleja 
\item syventää itseisarvokäsitteen ymmärtämystään ja oppii ratkaisemaan sellaisia itseisarvoyhtälöitä ja vastaavia epäyhtälöitä, jotka ovat tyyppiä $|f(x)|=a$ tai $|f(x)|=|g(x)|$
\item vahvistaa yhtälöryhmän ratkaisemisen taitojaan
\end{itemize}

Avoimet oppimateriaalit ry tuottaa ja julkaisee oppimateriaaleja ja kirjoja, jotka ovat kaikille vapaita käyttää. Vapaa matikka -sarja on suunnattu lukion pitkän matematiikan kursseille ja täyttää valtakunnallisen opetussuunnitelman vaatimukset.
