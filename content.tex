\Opensolutionfile{ans}[content/LIITE_vastaukset]

\chapter{Esitietoja}
	\section{Itseisarvo ja itseisarvoyhtälöt}

\qrlinkki{http://opetus.tv/maa/maa4/itseisarvo/}{Opetus.tv: Itseisarvolauseke ja itseisarvon ominaisuuksia}

% Rumasti toteutettu, mutta toistaiseksi toimii.
% Korjatkaa, jos joku saa siistimmin aikaiseksi.
\begin{lukusuora}{-5}{5}{10}
	\lukusuorapiste{0}{$0$}
	\lukusuorapiste{3}{}
	\lukusuorapiste{-3}{}
	\lukusuoraalanimi{3.1}{$a$}
	\lukusuoraalanimi{-3.3}{$-a$}
	\lukusuoranimi{1.5}{$\overbrace{\hspace{27 mm}}^{|a|}$}
	\lukusuoranimi{-1.5}{$\overbrace{\hspace{27 mm}}^{|-a|}$}
	
	\lukusuorapiste{2}{}
	\lukusuorapiste{-2}{}
	\lukusuoraalanimi{-2.3}{$-2$}
	\lukusuoraalanimi{2.1}{$2$}
	\lukusuoraalanimi{1}{$\underbrace{\hspace{18 mm}}_{|2|=2}$}
	\lukusuoraalanimi{-1}{$\underbrace{\hspace{18 mm}}_{|-2|=2}$}
\end{lukusuora}

Itseisarvo voidaan tulkita lukusuoralla pisteen etäisyytenä nollasta.
Epänegatiivisen luvun itseisarvo on luku itse ja
negatiivisen luvun itseisarvo on luvun vastaluku.

\laatikko[Itseisarvon määritelmä]{
	\[ |x|=\begin{cases}
		x, & \kun x \geq 0 \\
		-x, & \kun x < 0
	\end{cases} \]
}

% TÄMÄ oli minusta tällaisenaan hyvin epäselvä (Jokke) eikä kovin tarpeellinen
% Itseisarvo on funktio
% \[
% ||: \rr \rightarrow \rr, \;
% |x| = \begin{cases}
% 		x, & \kun x \geq 0 \\
% 		-x, & \kun x < 0.
% 	\end{cases}
% \]

\begin{esimerkki}
	Esitä ilman itseisarvomerkkejä
	\alakohdat
		§ $|3-\pi|$
		§ $|x-3|$
	\loppu
	\begin{esimratk}
		\alakohdat
			§ Koska $3-\pi\approx-0,14<0$, niin $|3-\pi|=-(3-\pi)=\pi-3$
			§ Koska $x-3\geq 0$, kun $x\geq3$, niin
				\[ |x-3| = \begin{cases}
					x-3, & \kun x \geq 3 \\
					-x+3, & \kun x < 3
				\end{cases} \]
		\loppu
	\end{esimratk}
\end{esimerkki}

\laatikko[Itseisarvon ominaisuuksia]{
	\begin{tabular}{ll}
		$|a|\geq0$ & Itseisarvo on aina epänegatiivinen \\
		$|a|=|-a|$ & Luvun ja sen vastaluvun itseisarvot ovat yhtäsuuret \\
		$|a|^2=a^2$ & Luvun itseisarvon neliö on yhtäsuuri kuin luvun neliö \\
		$|ab|=|a||b|$ & Tulon itseisarvo on tekijöiden itseisarvojen tulo \\
		$\Bigl|\dfrac{a}{b}\Bigr|=\dfrac{|a|}{|b|}$ & Osamäärän itseisarvo on tekijöiden itseisarvojen osamäärä
	\end{tabular}
}

\begin{esimerkki}
	Esitä lauseke $|x^2-16|+3x$ ilman itseisarvomerkkejä.
	\begin{esimratk}
		Tutkitaan ensin lausekkeen $x^2-16$ merkit. Aloitetaan etsimällä lausekkeen nollakohdat:
		\begin{align*}
			x^2-16 & =0 \\
			x^2 & =16 \\
			x & =\pm\sqrt{16} \\
			x & =\pm 4
		\end{align*}
		
		Lausekkeen $x^2-16$ kuvaaja on ylöspäin aukeneva paraabeli, joka leikkaa $x$-akselin kohdissa $-4$ ja $4$.
		
		\begin{lukusuora}{-10}{10}{8}
			\lukusuoraparaabeli{-4}{4}{-1.5}
			\lukusuorapiste{-4}{$-4$}
			\lukusuorapiste{4}{$4$}
			\lukusuoranimi{-6}{$+$}
			\lukusuoranimi{6}{$+$}
			\lukusuoraalanimi{0}{$-$}
		\end{lukusuora}

		Kun $-4<x<4$, niin $x^2-16<0$. Tällöin
		\[ |x^2-16|+3x = -(x^2-16)+3x = -x^2+16+3x=-x^2+3x+16. \]
		Kun $x\leq-4$ tai $x\geq4$, on $x^2-16\geq0$, ja tällöin
		\[ |x^2-16|+3x =x^2-16+3x=x^2+3x-16. \]
	\end{esimratk}
	\begin{esimvast}
		\[ |x^2-16|+3x = \begin{cases}
			-x^2+3x+16, & \kun -4<x<4 \\
			x^2+3x-16, & \text{kun $x\leq -4$ tai $x\geq 4$}
		\end{cases} \]
	\end{esimvast}
\end{esimerkki}

\subsection*{Yhtälö $|f(x)|=a$}

\qrlinkki{http://opetus.tv/maa/maa4/itseisarvoyhtalo/}{Opetus.tv: Itseisarvoyhtälö}

Itseisarvoyhtälössä muuttuja on itseisarvomerkkien sisällä. Itseisarvoyhtälö voidaan ratkaista esittämällä yhtälö ensin ilman itseisarvomerkkejä. Tässä voi hyödyntää itseisarvon määritelmää ja ominaisuuksia. Sen jälkeen ratkaistaan saadut yhtälöt tavalliseen tapaan.

\laatikko[Yhtälön $|f(x)|=a$ ratkaisu]{
	Jos $a\geq0$, niin
	\[ f(x)=a  \quad \loppui \quad f(x)=-a \]
	Jos $a<0$, niin yhtälöllä ei ole ratkaisuja.
}

\begin{esimerkki}
	Ratkaise yhtälö
	\alakohdat
		§  $|x|=3$
		§  $|x|=-2$
	\loppu
	\begin{esimratk}
		\alakohdat
			§ Ainoastaan lukujen $3$ ja $-3$ itseisarvot ovat 3, joten $x=3$ tai $x=-3$. Siis $x=\pm 3$
			§ Itseisarvo ei voi olla negatiivinen, joten yhtälöllä ei ole ratkaisua.
		\loppu
	\end{esimratk}
\end{esimerkki}

\begin{esimerkki}
	Ratkaise yhtälö $|3x-4|=2$.
	\begin{esimratk}
		Yhtälö toteutuu, kun luvun $3x-4$ etäisyys nollasta on 2, eli ainoastaan jos luku on $2$ tai $-2$. Saadaan:
		\begin{align*}
			3x-4 & =2 & & \loppui & 3x-4 & =-2 \\
			3x & =6 && & 3x & = 2 \\
			x & =2 && & x & =\frac{2}{3}
		\end{align*}
		Ratkaisun oikeellisuuden voi tarkistaa sijoittamalla saadut ratkaisut alkupäiseen yhtälöön.
		\[ |3\cdot2-4|=|6-4|=|2|=2 \quad \text{ja} \quad
			\left|3\cdot\frac{2}{3}-4\right|=|2-4|=|-2|=2. \]
	\end{esimratk}
	\begin{esimvast}
		$x=2$ tai $x=\dfrac{2}{3}$.
	\end{esimvast}
\end{esimerkki}

\begin{esimerkki}
	Ratkaise yhtälö $3x=3+|2x-3|$. (YO S82/1)
	\begin{esimratk}
		Siirretään aluksi termejä niin, että itseisarvolauseke jää yksin omalle puolelleen:
		\begin{align*}
			3x & =3+|2x-3|  \\
			3x-3 & =\underbrace{|2x-3|}_{\geq 0}
		\end{align*}
		Koska $|2x-3|\geq0$ aina, niin on oltava myös $3x-3\geq0$. Tällöin
		\begin{align*}
			3x-3&\geq0 \\
			3x&\geq3 \\
			x&\geq1
		\end{align*}
		Itseisarvoyhtälöstä saadaan:
		\begin{align*}
			2x-3&=3x-3   & &\loppui & 2x-3 & =-(3x-3) \\
			-x&=0        && & 2x-3&=-3x+3 \\
			x&=0         && & 5x&=6 \\
			\text{ei kelpaa, } & \text{oltava $x\geq1$}  && & x&=\frac{6}{5}, \text{ kelpaa}
		\end{align*}
	\end{esimratk}
	\begin{esimvast}
		$x=\dfrac{6}{5}=1\dfrac{1}{5}$.
	\end{esimvast}
\end{esimerkki}

Edellisen esimerkin tilannetta $3x-3=|2x-3|$ voidaan tarkastella myös piirtämällä kuvaajat $y=3x-3$ ja $y=|2x-3|$. Lausekkeet ovat yhtäsuuret, kun niiden saamat $y$:n arvot ovat yhtäsuuret, siis kohdissa, joissa kuvaajat leikkaavat toisensa.

\begin{kuva}
    kuvaaja.pohja(-1.5, 4.5, -2, 3, nimiX = "$x$", nimiY = "$y$")
    piste((1.2, 0.6), "", 180)
    vari("red")
    kuvaaja.piirra("3*x-3", nimi = "$y=3x - 3$", suunta = 0, kohta=0.7)
    vari("blue")
    kuvaaja.piirra("abs(2*x-3)", nimi = "$y=|2x - 3|$", kohta = 2.2)
\end{kuva}

Huomaamme, että lausekkeet saavat saman arvon kohdassa, jossa \mbox{$x=\frac{6}{5}$}. Esimerkkiä ratkaistaessa vastaan tullut epäkelpo ratkaisu $x=0$ olisi vastannut suoran $y=2x-3$ (ilman itseisarvomerkintää) leikkauspistettä suoran $y=3x-3$ kanssa, eli tilannetta, jossa sininen suora olisi kulkenut myös $x$-akselin alapuolella.

\subsection*{Yhtälö $|f(x)|=|g(x)|$}

Merkintä $|a|=|b|$ tarkoittaa, että luvut $a$ ja $b$ ovat lukusuoralla yhtä kaukana nollasta. Tällöin lukujen $a$ ja $b$ täytyy olla joko samat tai toistensa vastaluvut.

\laatikko{
	$|f(x)|=|g(x)| \quad \ekvi \quad f(x)=g(x) \quad \loppui \quad f(x)=-g(x)$.
}

Tällaisen yhtälön voi ratkaista myös neliöön korottamalla ominaisuuden $|a|^2=a^2$ avulla. Jos yhtälön molemmat puolet ovat ei-negatiivisia, niin yhtälön yhtäsuuruus säilyy neliöön korotettaessa.

\laatikko{
	$|f(x)|=|g(x)| \quad \ekvi \quad f(x)^2=g(x)^2$.
}

\begin{esimerkki}
	Ratkaise yhtälö $|2x-4|=|3-x|$.
	\begin{esimratk}
		\textbf{(Tapa 1)} Lukujen $2x-4$ ja $3-x$ tulee olla yhtäsuuret tai toistensa vastaluvut.
		\begin{align*}
			2x-4 & =3-x & &\loppui & 2x-4 & =-(3-x) \\
			3x & =7 && & 2x-4 & =-3+x \\
			x & =\frac{7}{3} && & x & =1
		\end{align*}
	\end{esimratk}
	\begin{esimvast}
		$x=\dfrac{7}{3}$ tai $x=1$.
	\end{esimvast}
	\begin{esimratk}
		\textbf{(Tapa 2)} Yhtälön molemmat puolet voidaan korottaa neliöön.
		\begin{align*}
			\underbrace{|2x-4|}_{\geq0} &= \underbrace{|3-x|}_{\geq0}    \\
			|2x-4|^2 &= |3-x|^2   \\
			(2x-4)^2 &= (3-x)^2   \\
			4x^2-16x+16 &= 9 -6x + x^2   \\
			3x^2-10x+7 &= 0   \\
			x &= \frac{10\pm\sqrt{(-10)^2-4\cdot 3\cdot 7}}{2\cdot 3}   \\
			x &= \frac{10\pm\sqrt{16}}{6}   \\		
			x &= \frac{10\pm4}{6}     \\
			x=\frac{14}{6}=\frac{7}{3} \quad  &\loppui \quad x=\frac{6}{6}=1 \\
		\end{align*}
	\end{esimratk}
	\begin{esimvast}
		$x=\dfrac{7}{3}$ tai $x=1$.
	\end{esimvast}
\end{esimerkki}

\subsection*{Yhtälö $|f(x)|=g(x)$}

\begin{esimerkki}
	Ratkaise yhtälö $|2x-4|=|3-x|+2x$.
	\begin{esimratk}
		Huomataan, etteivät edellisen esimerkin ratkaisutavat toimi, sillä yhtälö ei ole muotoa $|f(x)|=|g(x)|$. Neliöön korotuskaan ei toimi, sillä yhtälön oikea puoli voi saada negatiivisia arvoja. Ratkaistaan yhtälö poistamalla itseisarvomerkit ja ratkaisemalla syntyvät yhtälöt alueittain. Ratkaisun apuna voidaan hyödyntää merkkikaaviota, johon merkitsemme itseisarvojen sisäisten lausekkeiden saamat merkit kullakin välillä.

		\begin{lukusuora}{0}{4}{4}
			\lukusuorakuvaaja{2*x-4}
			\lukusuoranimi{1}{$2x-4$}
			\lukusuoraalanimi{1}{$-$}
			\lukusuoranimi{3}{$+$}
			\lukusuorapiste{2}{$2$}
		\end{lukusuora}
		\begin{lukusuora}{1}{5}{4}
			\lukusuorakuvaaja{3-x}
			\lukusuoraalanimi{2}{$3-x$}
			\lukusuoranimi{2}{$+$}
			\lukusuoraalanimi{4}{$-$}
			\lukusuorapiste{3}{$3$}
		\end{lukusuora}
		
		\begin{center}
			\begin{merkkikaavio}{2}
				\merkkikaavioKohta{$2$}
				\merkkikaavioKohta{$3$}

				\merkkikaavioFunktio{$2x-4$}
				\merkkikaavioMerkki{$-$}
				\merkkikaavioMerkki{$+$}
				\merkkikaavioMerkki{$+$}

			\merkkikaavioUusirivi
				\merkkikaavioFunktio{$3-x$}
				\merkkikaavioMerkki{$+$}
				\merkkikaavioMerkki{$+$}
				\merkkikaavioMerkki{$-$}
			\end{merkkikaavio}
		\end{center}
		
		Ratkaistaan yhtälö alueittain.
		\vaiheet
			§ Jos $x<2$, niin saadaan:
				\begin{align*}
					|\underbrace{2x-4}_{<0}|&=|\underbrace{3-x}_{>0}|+2x \quad \ppalkki x<2  \\
					-(2x-4)&=(3-x)+2x \\
					-2x+4&=3-x+2x \\
					-3x &= -1 \\
					x &= \frac{1}{3} \quad \text{(kelpaa)}
				\end{align*}

			§ Jos $2\leq x\leq 3$, niin saadaan:
				\begin{align*}
					|\underbrace{2x-4}_{\geq0}|&=|\underbrace{3-x}_{\geq0}|+2x \quad \ppalkki 2\leq x\leq 3  \\
					(2x-4)&=(3-x)+2x \\
					2x-4&=3-x+2x \\
					x &= 7 \quad \text{(ei kelpaa, sillä ei ole välillä $2\leq x\leq 3$)}
				\end{align*}

			§ Jos $x>3$, niin saadaan:
				\begin{align*}
					|\underbrace{2x-4}_{>0}|&=|\underbrace{3-x}_{<0}|+2x \quad \ppalkki x>3  \\
					(2x-4)&=-(3-x)+2x \\
					2x-4&=-3+x+2x \\
					-x &= 1 \\
					x &= -1  \quad \text{(ei kelpaa, sillä ei ole välillä $x>3$)}
				\end{align*}
		\loppu
	\end{esimratk}
	\begin{esimvast}
		\quad $x=\dfrac{1}{3}$.
	\end{esimvast}
\end{esimerkki}

\begin{tehtavasivu}

\sarjaA
\sarjaB
\sarjaC
\sarjaD

TÄHÄN TEHTÄVIÄ SIJOITTAMISTA ODOTTAMAAN

\begin{tehtava}
	Esitä lauseke ilman itseisarvomerkkejä.
	\alakohdat
		§ $|\pi-2^2|$
		§ $|2x-6|$
		§ $x+|6-3x|$
	\loppu
	\begin{vastaus}
		\alakohdat
			§ $-(\pi-4)=-\pi+4=4-\pi$
			§ $\begin{cases}
					-2x+6, & \jos x<3 \\
					2x-6, & \jos x \geq 3
				\end{cases}$
			§ $\begin{cases}
					x+(6-3x), & \jos 6-3x \geq0 \\
					x-(6-3x), & \jos 6-3x <0 
				\end{cases}\\
				=\begin{cases}
					x+6-3x, & \jos -3x \geq-6 \\
					x-6+3x, & \jos -3x <-6 
				\end{cases}\\
				=\begin{cases}
					-2x+6, & \jos x \leq2 \\
					4x-6, & \jos x >2 
				\end{cases}$
		\loppu
	\end{vastaus}
\end{tehtava}

\begin{tehtava}
	Esitä lauseke ilman itseisarvomerkkejä.
	\alakohdat
		§ $2x-x|2-x|$
		§ $|x^2+3|$
		§ $|x^2-4|$
	\loppu
	\begin{vastaus}
		\alakohdat
			§ $\begin{cases}
					x^2, & \jos x \leq2 \\
					-x^2+4x, & \jos x>2 
				\end{cases}$
			§ $x^2+3$
			§ $\begin{cases}
					x^2-4, & \jos x \leq -2 \loppui x \geq 2 \\
					-x^2+4, & \jos -2<x<2 
				\end{cases}$
		\loppu
	\end{vastaus}
\end{tehtava}

\begin{tehtava}
	Esitä lauseke ilman itseisarvomerkkejä.
	\alakohdat
		§ $(x-1)|4x^2+4x+1|$
		§ $|-3x^2+4x-2|$
	\loppu
	\begin{vastaus}
		\alakohdat
			§ $4x^3-3x-1$
			§ $3x^2-4x+2$
		\loppu
	\end{vastaus}
\end{tehtava}

\begin{tehtava}
	Esitä lauseke ilman itseisarvomerkkejä.
	\alakohdat
		§ $3|9-x^2|-2|x^2-9|$, kun $x\leq-3$
		§ $\dfrac{|-x^2+4x+5|}{|x^2-5x|}$, kun $x>5$
	\loppu
	\begin{vastaus}
		\alakohdat
			§ $x^2-9$ (vinkki: vastalukujen itseisarvot ovat yhtä suuret)
			§ $\frac{x+1}{x}$
		\loppu
	\end{vastaus}
\end{tehtava}

\begin{tehtava}
	Ratkaise yhtälö.
	\alakohdat
		§ $|x-2|=2$
		§ $|3x|=4$
		§ $|5x+7|=0$
		§ $|-4x+2|+1=0$
	\loppu
	\begin{vastaus}
		\alakohdat
			§ $x=0$ tai $x=4$
			§ $x=\frac{4}{3}$ tai $x=-\frac{4}{3}$
			§ $x=-\frac{7}{5}$
			§ ei ratkaisuja
		\loppu
	\end{vastaus}
\end{tehtava}

\begin{tehtava}
	Ratkaise yhtälö.
	\alakohdat
		§ $|x-2|=|3x|$
		§ $|3x|=|5x+7|$
		§ $|5x+6|=|5x+4|$
		§ $|-2x+2|=|x^2+2x+6|$
	\loppu
	\begin{vastaus}
		\alakohdat
			§ $x=-1$ tai $x=\frac{1}{2}$
			§ $x=-\frac{7}{2}$ tai $x=-\frac{7}{8}$
			§ $x=1$
			§ $x=-2$
		\loppu
	\end{vastaus}
\end{tehtava}

\begin{tehtava}
	Ratkaise yhtälö.
	\alakohdat
		§ $|-x| = |2x|$
		§ $|3x+1|= |7x-3|$
		§ $x^2+4|x|+4 = |x|^2+3|-x|+7$
		§ $|3x|+2 = |-2x|+1$
	\loppu
	\begin{vastaus}
		\alakohdat
			§ $x = 0$
			§ $x = 1$ tai $x = \frac{1}{5}$
			§ $x = \pm 3$
			§ Ei ratkaisuja.
		\loppu
	\end{vastaus}
\end{tehtava}

\begin{tehtava}
	Ratkaise
	\alakohdat
		§ $|x| = x^2$
		§ $3(|x|-1) = x^2-1$
		§ $2x^2+|x|-1 = 0$
	\loppu
	\begin{vastaus}
		\alakohdat
			§ $x = 0$ tai $x = \pm 1$
			§ $x = \pm 1$ tai $x = \pm 2$
			§ $x = \pm \frac{1}{2}$
		\loppu
	\end{vastaus}
\end{tehtava}

\begin{tehtava}
	Ratkaise yhtälö $|x+a| = |x|+|a|$ vapaan parametrin $a$ funktiona.
	\begin{vastaus}
		$x \in \rr$, jos $a=0$. $x>0$, jos $a>0$. $x<0$, jos $a<0$.
	\end{vastaus}
\end{tehtava}

\begin{tehtava}
	Itseisarvoyhtälöt voidaan ratkaista usein kätevästi neliöön korottamalla, vaikka yhtälön molemmat puolet eivät olisikaan välttämättä positiivisia. Tällöin on kuitenkin huomattava, että yhtälönratkaisun päättelyketjua ei voida suoraan suorittaa toiseen suuntaan: lopun ratkaisukandidaatit eivät välttämättä ole yhtälön ratkaisuja, mutta ne ovat ainoita mahdollisia, jolloin tarkistamalla ne yhtälö on ratkaistu. 

	Aina neliöminen ei kuitenkaan ole kannattavaa: neliöiminen saattaa johtaa identtisesti toteen yhtälöön/äärettömän moneen ratkaisukandidaattiin, joista ei välttämättä ole hyötyä alkuperäisen yhtälön ratkaisussa.

	Ratkaise yhtälöt.
	\alakohdat
		§ $|x+1|-|2x-1| = x$
		§ $||2x-1|-|3x-2|| = x+1$
		§ $||2x-1|-|3x-2|| = x-1$
	\loppu
	\begin{vastaus}
		\alakohdat
			§ $x = 0$ tai $x = 1$
			§ $x = 0$
			§ $x \geq 1$
		\loppu
	\end{vastaus}
\end{tehtava}

\end{tehtavasivu}

	\newpage \section{Itseisarvoepäyhtälöt}

\laatikko{
KIRJOITA TÄHÄN LUKUUN

\begin{itemize}
\item miten ratkaistaan epäyhtälöitä tyyliin
\item $|x|<a$,  $|x|>a$
\item $|f(x)|<a$, $|f(x)|<|g(x)|$, $|f(x)|<g(x)$
\end{itemize}

KIITOS!}

Itseisarvoepäyhtälön ratkaiseminen riippuu hyvin paljon epäyhtälön muodosta. Helpointa on hahmotella kutakin tilannetta erikseen lukusuoran avulla ja yrittää palauttaa itseisarvoepäyhtälö yhdeksi tai useammaksi tavalliseksi epäyhtälöksi. Teoriassa kaikki itseisarvoepäyhtälöt voidaan ratkaista tarkastelemalla itseisarvoissa olevien lausekkeiden merkkejä eri väleillä. Tutustutaan erilaisiin tilanteisiin esimerkkien avulla.

\begin{esimerkki}
Ratkaise epäyhtälö $|x|<3$.

\begin{esimratk}
On löydettävä kaikki sellaiset luvut, joiden itseisarvo on pienempi kuin $3$. Esimerkiksi luvut $2$ ja $-1$ käyvät, mutta luvut $4$ ja $-5,5$ eivät käy. Epäyhtälön ratkaisuja ovat täsmälleen ne luvut, joiden etäisyys nollasta on pienempi kuin $3$.

(Tähän kuva.)

Siten ratkaisu on $-3<x<3$.
\end{esimratk}

\begin{esimvast}
$-3<x<3$.
\end{esimvast}

\end{esimerkki}

\begin{esimerkki}
Ratkaise epäyhtälö $|x|>5$.

\begin{esimratk}
On löydettävä kaikki sellaiset luvut, joiden itseisarvo on suurempi kuin $5$. Esimerkiksi luvut $2$ ja $-4,5$ eivät ole epäyhtälön ratkaisuja, mutta luvut $5,5$ ja $-6$ ovat. Epäyhtälön ratkaisuja ovat täsmälleen ne luvut, joiden etäisyys nollasta on suurempi kuin $5$.
 
(Tähän kuva.)

Nyt ratkaisu on ilmoitettava kahdessa osassa: $x<-5$ tai $x>5$.
\end{esimratk}

\begin{esimvast}
$x<-5$ tai $x>5$
\end{esimvast}
\end{esimerkki}

\begin{esimerkki}
Ratkaise epäyhtälö $|x+4|<2$.

\begin{esimratk}
Nyt luvun $x+4$ itseisarvo on pienempi kuin kaksi, joten aikaisemman esimerkin perusteella tiedetään, että $-2<x+4<2$. Näin saadaan ratkaistaviksi epäyhtälöt $-2<x+4$ ja $x+4<2$. Ratkaistaan nämä kaksi epäyhtälöä erikseen:
\begin{align*}
-2&<x+4 & \ppalkki{+2} \\
0&<x+6 & \ppalkki{-6} \\
-6&<x &
\end{align*}
ja
\begin{align*}
x+4&<2 & \ppalkki{-4} \\
x&<-2 & \ppalkki{-6}
\end{align*}

Kun vastaukset yhdistetään, saadaan ratkaisuksi $-6<x<-2$.

(Tähän kuva.)

\end{esimratk}

\begin{esimvast}
$-6<x<-2$.
\end{esimvast}
\end{esimerkki}

Itseisarvoepäyhtälöiden ratkaisu voidaan usein myös palauttaa jo tuttuihin tekijätarkasteluihin, kun huomataan, että

\begin{align*}
|a| \geq |b| \Leftrightarrow a^2 \geq b^2 \Leftrightarrow (a-b)(a+b) \geq 0
\end{align*}

\begin{esimerkki} Ratkaise epäyhtälö $|2x-1|>|x+2|$.

\begin{esimratk} \textbf{Tapa 1} Koska molemmat puolet ovat epänegatiivisia, itseisarvomerkit voitaisiin poistaa korottamalla epäyhtälö puolittain toiseen potenssiin. Mutta miten käy epäyhtälön merkille? Jos kaksi epänegatiivista lukua korotetaan toiseen potenssiin, niiden keskinäinen järjestys säilyy. Voimme siis korottaa tehtävän epäyhtälön puolittain toiseen potenssiin, ja epäyhtälömerkin suunta säilyy. Päättely voidaan suorittaa myös toiseen suuntaan: Jos vasemman puolen neliö on oikean puolen neliötä suurempi, on myös itselausekkeen oltava itseisarvoltaan suurempi. Tämän jälkeen epäyhtälöä käsitellään toisen asteen epäyhtälönä.

\begin{align*}
|2x-1|^2 & >|x+2|^2 \\
(2x-1)^2 & >(x+2)^2 \\
4x^2-4x+1 & >x^2+4x+4 \\
3x^2-8x-3 & >0
\end{align*}

Toisen asteen epäyhtälö ratkaistaan etsimällä lausekkeen nollakohdat ja päättelemällä merkkikaavion tai kuvaajan avulla, millä alueilla epäyhtälö toteutuu.
\begin{align*}
3x^2-8x-3 & =0 \\
x & =\frac{8\pm\sqrt{8^2-4\cdot3\cdot(-3)}}{2\cdot 3} \\
x & =\frac{8\pm\sqrt{64+36}}{6} \\
x & =\frac{8\pm 10}{6} \\
x=3 \quad & \text{tai} \quad x=-\frac{1}{3}
\end{align*}

TÄHÄN MERKKIKAAVIO JA KUVAAJA

Epäyhtälö toteutuu lausekkeen $3x^2-8x-3$ nollakohtien välisen alueen ulkopuolella eli silloin, kun $x<-\dfrac{1}{3}$ tai $x>3$.
\end{esimratk}

\begin{esimvast}
$x<-\dfrac{1}{3}$ tai $x>3$.
\end{esimvast}

\begin{esimratk} \textbf{Tapa 2}
Niin kuin havaittiin,
\begin{align*}
|2x-1| > |x+2| \Leftrightarrow (2x-1)^2 > (x+2)^2 \Leftrightarrow ((2x-1)-(x+2))((2x-1)+(x+2)) > 0
\end{align*}
eli
\begin{align*}
(x-3)(3x+1)>0.
\end{align*}
Tekijöiden nollakohdat ovat tietenkin $x = 3$ ja $x = -\frac{1}{3}$. Nyt voidaan jatkaa merkkikaaviolla niin kuin ensimmäisessä ratkaisussa.
\end{esimratk}
\begin{esimvast}
$x<-\dfrac{1}{3}$ tai $x>3$.
\end{esimvast}
\end{esimerkki}

\begin{tehtavasivu}

\subsubsection*{Opi perusteet}

\begin{tehtava}
Ratkaise seuraavat epäyhtälöt.
% vai: Esitä ilman itseisarvomekkejä.
	\begin{alakohdat}
		\alakohta{$|x|<6$}
		\alakohta{$|x|>10$}
		\alakohta{$|x|<1,6$}
	\end{alakohdat}
	\begin{vastaus}
		\begin{alakohdat}
			\alakohta{$-6<x<6$}
			\alakohta{$x<-10$ tai $x>10$}
			\alakohta{$-1,6<x<1,6$}
		\end{alakohdat}
	\end{vastaus}
\end{tehtava}

\begin{tehtava}
Ratkaise seuraavat epäyhtälöt.
	\begin{alakohdat}
		\alakohta{$|x+6|>3$}
		\alakohta{$|x-5|<2$}
	\end{alakohdat}
	\begin{vastaus}
		\begin{alakohdat}
			\alakohta{$x<-9$ tai $x>-3$}
			\alakohta{$3<x<7$}
		\end{alakohdat}
	\end{vastaus}
\end{tehtava}

\subsubsection*{Hallitse kokonaisuus}

\subsubsection*{Sekalaisia tehtäviä}

TÄHÄN TEHTÄVIÄ SIJOITTAMISTA ODOTTAMAAN

\begin{tehtava}
Ratkaise seuraavat epäyhtälöt.
	\begin{alakohdat}
		\alakohta{$|x|\le 6$}
		\alakohta{$|x|>-3$}
	\end{alakohdat}
	\begin{vastaus}
		\begin{alakohdat}
			\alakohta{$-6 \le x \le 6$}
			\alakohta{ei ratkaisua}
		\end{alakohdat}
	\end{vastaus}
\end{tehtava}


\begin{tehtava}
	Ratkaise epäyhtälö $|x^2+1| \ge 3$.
	\begin{vastaus}
		$x<-\sqrt{2}$ tai $x>\sqrt{2}$
	\end{vastaus}
\end{tehtava}

\begin{tehtava}
	Ratkaise epäyhtälö $|x+a| \leq |x|+|a|$ vapaan parametrin $a$ funktiona.
	\begin{vastaus}
		$x \in \R$
	\end{vastaus}
\end{tehtava}

\begin{tehtava}
Ratkaise seuraavat epäyhtälöt.
	\begin{alakohdat}
		\alakohta{$|x+1|<|3x-3|$}
		\alakohta{$|22+x| \geq |5x+2|$}
		\alakohta{$|12x-7| \leq -3x+8$}
	\end{alakohdat}
	\begin{vastaus}
		\begin{alakohdat}
			\alakohta{$x< \frac{1}{2}$ tai $x>2$}
			\alakohta{$-4 \leq x \leq 5$}
			\alakohta{$-\frac{1}{9} \leq x \leq 1$}
		\end{alakohdat}
	\end{vastaus}
\end{tehtava}

\begin{tehtava}
Millä vakion $a$ arvoilla epäyhtälö on tosi, kun $x=3$?
	\begin{alakohdat}
		\alakohta{$|2x+a|>|7|$}
		\alakohta{$|12-x| > |ax+5a|$}
	\end{alakohdat}
	\begin{vastaus}
		\begin{alakohdat}
			\alakohta{$a< -13$ tai $a>1$}
			\alakohta{$-\frac{9}{8} < a < \frac{9}{8}$}
		\end{alakohdat}
	\end{vastaus}
\end{tehtava}

\begin{tehtava}
	Todista, että
	\begin{align*}
		|a| \geq |b| \Leftrightarrow a^2 \geq b^2 \Leftrightarrow (a-b)(a+b) \geq 0
	\end{align*}
	\begin{vastaus}
		Vinkki: Huomaa, että $|a|^2 = a^2$. Mitä voidaan sanoa lausekkeesta $|a|^2-|b|^2$ muistikaavan avulla?
	\end{vastaus}
\end{tehtava}

\end{tehtavasivu}

	% itseisarvoepäyhtälöt
	\newpage \section{Lineaariset yhtälöryhmät} 

\laatikko{
KIRJOITA TÄHÄN LUKUUN

\begin{itemize}
\item mikä yhtälöryhmä on
\item miten ratkaistaan yhtälöpari (sijoitus, yhteenlaskumenetelmä)
\item että ratkaisuja voi olla yksi, nolla tai äärettömän monta
\item miten useamman tuntemattoman yhtälöryhmä ratkaistaan
\end{itemize}

KIITOS!}

\subsection*{Johdanto: Lineaarinen yhtälö}

Kursseilla MAA1 ja MAA2 on tutustuttu ensimmäisen 
\[ax +b = 0\]
ja toisen asteen yhtälöihin
\[ax^2 + bx +c = 0\],
jotka molemmat ovat yhden muuttujan ($x$) yhtälöitä.

Edellä olemme käsitelleet suoran yhtälöä
\[ax + by + c = 0,\]
joka on kahden muuttujan, $x$:n ja $y$:n, yhtälö, jonka termit ovat ensimmäistä astetta tai vakioita.

Yleisesti tällaisiä yhtälöitä, joissa termit eivät ole ensimmäistä astetta suurempia kutsutaan \termi{lineaarinen yhtälö}{lineaariksi yhtälöiksi}.

\laatikko{
  \termi{lineaarinen yhtälö}{Lineaarisessa yhtälössä} kukin termi on korkeintaan korkeintaan ensimmäistä astetta. Esimerkiksi
  \[ax + b = 0,\]
  \[ax + bx + c = 0\]
  ja
  \[ax + by + cz + d = 0\]
  ovat lineaarisia yhtälöitä.
}

\subsection*{Yhtälöryhmä}

\termi{yhtälöryhmä}{Yhtälöryhmällä} tarkoitetaan useaa yhtälöä, joiden täytyy
päteä samanaikaisesti. \footnote{Englannin kielessä yhtälöryhmästä käytetäänkin usein nimeä 
  \emph{simultaneous equations} eli samanaikaiset yhtälöt.
  Termi \emph{system of equations} on tosin vähintäänkin yhtä yleinen.}
Yhtälöryhmän ratkaisujen tulee siis toteuttaa kaikki ryhmän yhtälöt; ratkaisuja ei välttämättä ole.

Tässä luvussa käsitellään yhtälöryhmiä, joissa kaikki yhtälöt ovat \emph{lineaarisia}. 
Tällaisia yhtälöryhmiä kutsutaan \termi{lineaarinen yhtälöryhmä}{lineaarisiksi yhtälöryhmiksi}.

Esimerkiksi
\[
\left\{
\begin{aligned}
3x-2y-z&= -5 \\
5x+6y+5z&= 1 \\
-x+5y&= 0.
\end{aligned}
\right.
\]
on lineaarinen yhtälöryhmä. 

%Lisäksi myöhemmin kirjassa käsitellään muutamia erikoistapauksia toisen asteen yhtälöryhmistä
%ratkaistaessa kahden ympyrän tai ympyrän ja suoran leikkauspisteitä.
% re ^:
%  Ympyrä-kappaleessa implisiittisesti? onko menetelmät jos esitelty siellä, kyllä? 
%  Jos kyllä, harkittava pitäisikö sitten siinä kohdin mainita 'ratkaistaan \termi{epälineaarinen yhtälöryhmä} blah blah...'
%  Vai oma, erillinen kappale, tyyliin 'Luku n. Joitain epälineaarisia yhtälöryhmiä', vaikkapa lisämateriaaliksi?

\subsection*{Yhtälöpari}

Yksinkertaisin mielenkiintoinen yhtälöryhmä on
\termi{lineaarinen yhtälöpari}{lineaarinen yhtälöpari}.
Lineaarisessa yhtälöparissa on kaksi ensimmäistä astetta olevaa yhtälöä.
Tällä kurssilla yhtälöparilla tarkoitetaan useimmiten kahden muuttujan lineaarista yhtälöparia.

Lineaarinen yhtälöpari voidaan esittää monella tapaa. Tässä
kirjassa käytämme pääasiallisesti muotoa, jossa kumpikin yhtälö on normaalimuodossa.

\laatikko[Kahden muuttujan yhtälöparin normaalimuoto]{
\begin{samepage}
  \[
    \left\{
      \begin{aligned}
      a_1x+b_1y+c_1 &= 0 \\
      a_2x+b_2y+c_2 &= 0,
      \end{aligned}
    \right.
  \]
missä $a_1, a_2, b_1, b_2, c_1, c_2 \in \rr$
\end{samepage}}

Kahden muuttujan lineaarisen yhtälöparin ratkaisu on lukupari $(x, y)$, joka toteuttaa molemmat yhtälöt.

Lineaarisia yhtälöpareja ratkotaan lukiomatematiikassa pääasiallisesti kahdella menetelmällä.
Nämä menetelmät ovat \termi{sijoitusmenetelmä}{sijoitusmenetelmä} ja
\termi{yhteenlaskumenetelmä}{yhteenlaskumenetelmä}.
\begin{esimerkki}
Ratkaise yhtälöpari
\[
\left\{
\begin{aligned}
3x-2y&= 1 \\
-x+5y&= 2.
\end{aligned}
\right.
\]
\begin{esimratk}
On löydettävä kaikki ne luvut $x$ ja $y$ jotka toteuttavat molemmat yhtälöt.

Käytetään \termi{sijoitusmenetelmä}{sijoitusmenetelmää}. Ratkaistaan tuntematon $x$ alemmasta yhtälöstä ja sijoitetaan se ylempään yhtälöön. Alemmasta yhtälöstä $-x+5y= 2$ saadaan $x=5y-2$. Sijoitetaan tämä ylempään yhtälöön $3x-2y=1$:
\begin{align*}
3x-2y&=1 && \ppalkki \text{Sijoitetaan $x=5y-2$.} \\
3(5y-2)-2y&=1 && \\
15y-6-2y&=1 && \\
13y&=7 && \\
y&=\frac{13}{7} && \\
\end{align*}
Sijoitetaan $y=13/7$ alempaan yhtälöön, jotta voidaan ratkaista $x$:
\begin{align*}
-x+5y&= 2 && \ppalkki \text{Sijoitetaan $y=\frac{13}{7}$.} \\
-x+\frac{65}{7}&= 2 && \\
-x&= -\frac{51}{7}&& \\
x&= \frac{51}{7}&&
\end{align*}
\end{esimratk}
\begin{esimvast}
Yhtälön ratkaisu on $x= 51/7$, $y=13/7$.
\end{esimvast}
\end{esimerkki}

\begin{esimerkki}
Määritä suorien $-2x-y= 4$ ja $3x-2y=1$ leikkauspiste.
\begin{esimratk}
Saadaan ratkaistavaksi yhtälöpari
\[
\left\{
\begin{aligned}
-2x-y&= 4 \\
3x-2y&= 1.
\end{aligned}
\right.
\]
Käytetään \termi{yhteenlaskumenetelmä}{yhteenlaskumenetelmää}, jolla voidaan eliminoida toinen tuntemattomista.
\begin{align*}
  &\left\{
    \begin{aligned}
    2x-y&= 4 && \ppalkki \cdot 3\\
    3x-2y&= 1 && \ppalkki \cdot (-2)
    \end{aligned}
  \right. \\
  &\left\{
    \begin{aligned}
    6x-3y&= 12 && \ppalkki \text{Lasketaan yhtälöt yhteen.}\\
    -6x+4y&= -2.&&
    \end{aligned}
  \right.\\
  &y= 10 \\
\end{align*}
Sijoitetaan $y=10$ jompaan kumpaan alkuperäisistä yhtälöistä, jotta saadaan ratkaistua $x$. Käytetään vaikkapa yhtälöä $3x-2y=1$:
\begin{align*}
3x-2y&=1 && \ppalkki \text{Sijoitetaan $x=10$.} \\
3x-20&=1 && \\
3x&=21 && \\
x&=7 &&.
\end{align*}
\end{esimratk}
\begin{esimvast}
Yhtälön ratkaisu on $x=7$, $y=10$.
\end{esimvast}
\end{esimerkki}

%EDELLISEN ESIMERKIN ULKOASUA PITÄÄ VIELÄ MUOKATA. TASAUKSET EIVÄT TOIMI JA LISÄKSI YHTÄLÖPARIN ALLE PITÄISI SUMMAUKSEN KOHDALLA SAADA VIIVA. SAMA ONGELMA MYÖHEMMISSÄ ESIMERKEISSÄ.

\subsection{Yhtälöparin ratkaisujen määrä ja geometrinen tulkinta}

Aiemmin on todettu, että normaalimuotoinen ensimmäisen asteen yhtälö voidaan tulkita suorana
$xy$-tasossa. Näin ollen lineaariselle yhtälöparille on geometrinen tulkinta: sen
ratkaisut ovat ne tason pisteet, joissa yhtälöitä vastaavat
suorat leikkaavat. Näitä voi olla
$0$ (suorat ovat yhdensuuntaiset, mutta eivät sama suora),
$1$ (suorat eivät ole yhdensuuntaiset) tai
äärettömän monta (suorat ovat sama suora).

\laatikko{Yhtälöparilla on 0, 1 tai äärettömän monta ratkaisua.}

Tarkastellaan esimerkkiä yhtälöryhmästä, jolla ei ole lainkaan ratkaisuja.

\begin{esimerkki}
Ratkaise yhtälöpari
\[
\left\{
\begin{aligned}
x-2y&= 1 \\
5x-10y&= 2.
\end{aligned}
\right.
\]
\begin{esimratk}
Käytetään yhteenlaskumenetelmää:
\begin{align*}
&\left\{
\begin{aligned}
-x+2y&= 1 && \ppalkki \cdot 5 \\
5x-10y&= 2. &&
\end{aligned}
\right. \\
&\left\{
\begin{aligned}
-5x-10y&= 5 && \ppalkki \cdot \text{Lasketaan yhtälöt yhteen.} \\
5x-10y&= 2. &&
\end{aligned}
\right. \\
&0=7 \\
\end{align*}
Koska päädytään mahdottomaan yhtälöön, yhtälöparilla ei ole ratkaisua.
\end{esimratk}
\begin{esimvast}
Yhtälöparilla ei ole ratkaisua.
\end{esimvast}
\end{esimerkki}

Edellisessä esimerkissä yhtälöryhmällä ei ollut lainkaan ratkaisuja. Geometrisesti tämä tarkoittaa sitä, että suorilla $x-2y= 1$ ja $5x-10y= 2$ ei ole leikkauspisteitä. Ne ovat siis yhdensuuntaiset.
Tämä voi nähdä myös kirjoittamalla suorien yhtälöt muodossa
\begin{align*}
  x-2y &= 1 \\
  2y &= x - 1 \\
  y &= \frac{1}{2}x -1
\end{align*}
ja
\begin{align*}
  5x-10y &= 2 \\
  10y &= 5x -2 \\
  y &= \frac{1}{2}x - \frac{1}{5},
\end{align*}
josta nähdään että suorien kulmakertoimet ovat samat.

TÄHÄN KUVA?

Tutkitaan vielä yhtälöparia, jolla on äärettömän monta ratkaisua.

\begin{esimerkki}
Ratkaise yhtälöpari
\[
\left\{
\begin{aligned}
-3x+y&= -1 \\
6x-2y&= 2.
\end{aligned}
\right.
\]
\begin{esimratk}
Käytetään sijoitusmenetelmää. Ensimmäisestä yhtälöstä saadaan $y=3x-1$. Sijoitetaan tämä toiseen yhtälöön:
\begin{align*}
6x-2y&= 2 && \ppalkki \text{sijoitetaan $y=3x-1$} \\
6x-2(3x-1)&= 2 && \\
6x-6x+2&= 2 && \\
2&= 2. &&
\end{align*}
\end{esimratk}
Näin saadusta yhtälöstä $2=2$ ei voikaan ratkaista tuntematonta $x$ niin kuin oli tarkoitus. Yhtälö ei anna mitään lisätietoa tuntemattomista. Tämä johtuu siitä, että alkuperäiset yhtälöt $-3x+y= -1$ ja $6x-2y= 2$ ovat yhtäpitäviä. Ensimmäisestä saadaan jälkimmäinen kertomalla luvulla $-2$.

Oleellisesti tarkasteltavana onkin vain yksi yhtälö, $-3x+y= -1$. Yhtälöparin ratkaisuja ovat kaikki ne luvut $x$ ja $y$, jotka toteuttavat tämän yhtälön. Ratkaisuja ovat esimerkiksi $x=0$, $y=-1$ ja $x=1$, $y=2$. Ratkaisuja on äärettömän monta.
\begin{esimvast}
Ratkaisuja on äärettömän monta.
\end{esimvast}
\end{esimerkki}

Edellisessä esimerkissä yhtälöparilla oli äärettömän monta ratkaisua. Geometrisesti tämä tarkoittaa sitä, että yhtälöt $-3x+y= -1$ ja $6x-2y= 2$ ovat saman suoran yhtälö. Siten kaikki suoralla $-3x+y= -1$ (tai yhtä hyvin suoralla $6x-2y= 2$) olevat pisteet toteuttavat yhtälöparin.



\subsection*{Useamman kuin kahden yhtälön yhtälöryhmät}

Kun ratkaistavia yhtälöitä voi olla useampia kuin kaksi, puhutaan yleisesti \termi{yhtälöryhmä}{yhtälöryhmästä}. Yhtälöpari on yhtälöryhmän erikoistapaus. \footnote{Joskus myös yhtälöparia kutsutaan yhtälöryhmäksi, kun ei haluta korostaa yhtlöiden lukumäärää.}

Tässä tarkastellaan lähinnä kolmen yhtälön lineaarisia yhtälöryhmiä. Tätä useamman yhtälön
lineaarisista yhtälöryhmistä esitetään joitakin helppoja esimerkkejä. 

Tähän asti esimerkeissä on käsitelty yhtälöryhmiä, joissa muuttujia on ollut yhtä monta kuin yhtälöitä, mutta yhtälöitä ja muuttujia voi olla myös eri määrä, mikä vaikuttaa yhtälöryhmän mahdollisten ratkaisujen lukumäärään.

Yleisesti ottaen yhtälöryhmiä ei käytännössä ratkaista tällä kurssilla esitetyin keinoin. Useimmiten yhtälöryhmiä ratkaistaan likimääräisesti tietokoneella käyttäen numeerista matriisilaskentaa.
%LISÄMATERIAALEIHIN MAININTA, ESITTELY? Soveltava kurssi, Vapaa Matikka MaaN.N.: Lukion lineaarialgebraa?


\begin{esimerkki}
Ratkaise lineaarinen yhtälöryhmä
\[
\left\{
\begin{aligned}
4x+2y+5z&=0 \\
6x-3y-z&= 23 \\
x-2y+3z&= -1.
\end{aligned}
\right.
\]
\begin{esimratk}
Käytetään yhteenlaskumenetelmää kahteen ylimpään yhtälöön ja eliminoidaan niistä tuntematon $x$:
\begin{align*}
&\left\{
\begin{aligned}
4x+2y+5z&=0 && \ppalkki \cdot 3 \\
6x-3y-z&= 23 && \ppalkki \cdot (-2) \\
\end{aligned}
\right. \\
&\left\{
\begin{aligned}
12x+6y+15z&=0 \\
-12x+6y+2z&= -46 \\
\end{aligned}
\right. \\
&12y+17z=-46 \\
\end{align*}

Tehdään sitten sama toiselle ja kolmannelle yhtälölle:
\begin{align*}
&\left\{
\begin{aligned}
6x-3y-z&= 23 && \\
x-2y+3z&= -1 && \ppalkki \cdot (-6)
\end{aligned}
\right. \\
&\left\{
\begin{aligned}
6x-3y-z&= 23 \\
-6x+12y-18z&=6
\end{aligned}
\right. \\
&9y-19z=29. \\
\end{align*}

Nyt on saatu kaksi yhtälöä, joissa ei ole tuntematonta $x$. Ratkaistaan näin muodostuva yhtälöpari:
\begin{align*}
&\left\{
\begin{aligned}
12y+17z&=-46 && \ppalkki \cdot 3 \\
9y-19z&=29 && \ppalkki \cdot (-4) \\
\end{aligned}
\right. \\
&\left\{
\begin{aligned}
36y+51z&=-138  \\
-36y+76z&=-116  \\
\end{aligned}
\right. \\
&127z=-254 \\
&z=-2 \\
\end{align*}

Nyt tiedetään, että $z=-2$. Sijoitetaan tämä aiemmin saatuun kahden tuntemattoman yhtälöön $12y+17z=-46$:
\begin{align*}
12y+17z&=-46 && \ppalkki \text{Sijoitetaan $z=-2$.} \\
12y-34&=-46 && \\
12y&=-12 && \\
y&=-1 && \\
\end{align*}

Lopuksi sijoitetaan $y=-1$ ja $z=-2$ johonkin alkuperäisistä yhtälöistä, vaikkapa ensimmäiseen yhtälöön:
\begin{align*}
4x+2y+5z&=0 && \ppalkki \text{Sijoitetaan $y=-1$ ja $z=-2$.} \\
4x-2-10&=0 && \\
4x&=12 && \\
x&=3 && \\
\end{align*}

Nyt on siis saatu ratkaisu $x=3$, $y=-1$, $z=-2$. Tarkistetaan vielä, että nämä luvut tosiaankin toteuttavat kaikki alkuperäisen yhtälöryhmän
\[
\left\{
\begin{aligned}
4x+2y+5z&=0 \\
6x-3y-z&= 23 \\
x-2y+3z&= -1.
\end{aligned}
\right.
\]
yhtälöt. Tämä tehdään sijoittamalla luvut yhtälöiden vasemmalle puolella ja tarkistamalla, että tulos täsmää yhtälön oikean puolen kanssa.

Koska $4 \cdot 3+2\cdot(-1)+5\cdot(-2)=12-2-10=0$, ne toteuttavat ensimmäisen yhtälöistä. Samalla tavalla $6 \cdot 3-3\cdot(-1)-\cdot(-2)=18+3+2=23$, joten luvut toteuttavat myös toisen yhtälön. Lopuksi todetaan, että $3-2\cdot(-1)+3\cdot(-2)=3+2-6=0-1$, ja siten kolmaskin yhtälöistä toteutuu.

\end{esimratk}
\begin{esimvast}
Yhtälöryhmän ratkaisu on $x=3$, $y=-1$, $z=-2$.
\end{esimvast}
\end{esimerkki}

\begin{tehtavasivu}

\subsubsection*{Opi perusteet}

\begin{tehtava}
  Mitkä seuraavista ovat lineaarisia yhtälöitä?
  \begin{alakohdat}
    \alakohta{$2x^2 + y +1 = 0$}
    \alakohta{$y = 0$}
    \alakohta{$3x - 2y = 5$}
  \end{alakohdat}
  \begin{vastaus}
    \begin{alakohdat}
      \alakohta{Ei.}
      \alakohta{On.}
      \alakohta{On.}
    \end{alakohdat}
  \end{vastaus}
\end{tehtava}

\begin{tehtava}
    Ratkaise yhtälöparit.
    \begin{align*}
        x-2y &= 0 \\
        -2x+y+3 &=0
    \end{align*}
    \begin{vastaus}
        $x = 2, \, y = 1$
    \end{vastaus}
\end{tehtava}

\begin{tehtava}
    Ratkaise yhtälöpari.
    \begin{align*}
        x+y+1 &= 0 \\
        x+2y+1 &=0
    \end{align*}
    \begin{vastaus}
        $x = -1, \, y = 0$
    \end{vastaus}
\end{tehtava}

\begin{tehtava}
    Ratkaise yhtälöpari.
    \begin{align*}
        2x+5y+1 &= 0 \\
        2x+2y+7 &=0
    \end{align*}
    \begin{vastaus}
        $x = -\frac{11}{2}, \, y = 2$
    \end{vastaus}
\end{tehtava}

\begin{tehtava}
    Ratkaise yhtälöpari.
    \begin{align*}
        2a+3b &= 8 \\
        6a+2b &= -4
    \end{align*}
    \begin{vastaus}
        $a = -2, \, b = 4$
    \end{vastaus}
\end{tehtava}

\begin{tehtava}
    Ratkaise yhtälöpari.
    \begin{align*}
        \frac{x}{3}+\frac{y}{7} + 1 &= 3 \\
        x - \frac{y-1}{3} &= -5
    \end{align*}
    \begin{vastaus}
        $x = -45, \, y = 119$
    \end{vastaus}
\end{tehtava}

\begin{tehtava}
	Ratkaise yhtälöpari.
	\begin{align*}
		x^2-y+1 &= 0 \\
		x+y-2 &= 0
	\end{align*}
	\begin{vastaus}
		$x = \frac{-1+\sqrt{5}}{2}, y = \frac{10-2\sqrt{5}}{4}$ tai $x = \frac{-1-\sqrt{5}}{2}, y = \frac{10+2\sqrt{5}}{4}$
	\end{vastaus}
\end{tehtava}

\begin{tehtava}
    Ratkaise yhtälöpari. $t \in \rr$ on vapaa parametri, joka saa sisältyä vastaukseen.
    \begin{align*}
        x+2y-t-1 &= 0 \\
        x+y+t^2 &=0
    \end{align*}
    \begin{vastaus}
        $x = -2t^2-t-1, \, y = t^2+t+1$
    \end{vastaus}
\end{tehtava}

\begin{tehtava}
    Ratkaise yhtälöryhmä.   
    \begin{align*}
        x+2y+1 &=0 \\
        x+2z+3 &=0 \\
        y+2z+5 &=0
    \end{align*}
    \begin{vastaus}
        $x = 1, \, y = -1, \, z = -2$
    \end{vastaus}
\end{tehtava}

\subsubsection*{Hallitse kokonaisuus}

\begin{tehtava}
    Ratkaise yhtälöryhmä.
    \begin{align*}
        x+y+z+8 &= 0 \\
        x+y+6 &=0 \\
        x+z-70 &=0
    \end{align*}
    \begin{vastaus}
        $x = 72, \, y = -78, \, z = -2$
    \end{vastaus}
\end{tehtava}

\begin{tehtava}
    Ratkaise yhtälöryhmä.
    \begin{align*}
        x+y+2z+12 &= 0 \\
        2x+2y+3z+1 &=0 \\
        3x-4 &=0
    \end{align*}
    \begin{vastaus}
        $x = \frac{4}{3}, \, y = \frac{98}{3}, \, z = -23$
    \end{vastaus}
\end{tehtava}

\begin{tehtava}
    Ratkaise yhtälöryhmä.
    \begin{align*}
        2x+3y+5z+8 &= 0 \\
        3x+5y+8z &=0 \\
        x+y-1 &=0
    \end{align*}
    \begin{vastaus}
        $x = -\frac{63}{2}, \, y = \frac{65}{2}, \, z = -\frac{17}{2}$
    \end{vastaus}
\end{tehtava}

\begin{tehtava}
	Ratkaise yhtälöryhmä.
	\begin{align*}
		x+w+3 &= 0 \\
		x+y+z &= 0 \\
		y-w-3 &= 0 \\
		w-2z+5 &= 0
	\end{align*}
	\begin{vastaus}
		$x=2, y=-2, z=0, w=-5$
	\end{vastaus}
\end{tehtava}

\begin{tehtava}
	Ratkaise yhtälöryhmä.
	\begin{align*}
		x+2y-z &= 0 \\
		x+3y-w &= 0 \\
		x+y+z+w-4 & = 0 \\
		x+2y+2z+2w-6 &= 0
	\end{align*}
	\begin{vastaus}
		$x=2, y=-\frac13, z=\frac43, w=1$
	\end{vastaus}
\end{tehtava}

\begin{tehtava}
	Ratkaise yhtälöryhmä.
    	\begin{align*}
        	x+y+tz &=1 \\
        	x+ty+z &=1 \\
        	tx+y+z &=1
    	\end{align*}
	\begin{vastaus}
		$x = y = z = \frac{1}{t+2}$, kun $1 \neq t \neq -2$. Kun $t = -2$ yhtälöryhmällä ei ole ratkaisuja.
		Kun $t = 1$ kaikki kolmikot muotoa $x = r$, $y = s$ ja $z = 1-r-s$, jollain reaaliluvuilla $r$ ja $s$ ovat ratkaisuja.
	\end{vastaus}
\end{tehtava}

\subsubsection*{Sekalaisia tehtäviä}

TÄHÄN TEHTÄVIÄ SIJOITTAMISTA ODOTTAMAAN

\end{tehtavasivu}

	% sijoitusmenetelmä
	% yhtälöiden laskeminen yhteen
	% ratkaisujen määrä
	\newpage \section{Koordinaatisto ja yhtälön kuvaaja}

\laatikko{
KIRJOITA TÄHÄN LUKUUN

\luettelo{
§ ihan lyhyt koordinaatistokertaus
§ kahden pisteen välinen etäisyys (pysty-tai vaakasuoraan helpolla, Pythagoraan lauseella yleensä)
§ esimerkkejä käyrän yhtälöistä, esim. suora, paraabeli, kartesiuksen lehti
}

KIITOS!}

Analyyttisen geometrian perusajatus on käsitellä geometrisia kuvioita koordinaatistossa.
Koordinaatistossa kuvion jokainen piste voidaan ilmoittaa sen koordinaattien avulla.
Esimerkiksi oheisessa kuvassa on kolmio $ABC$, jonka nurkkapisteiden koordinaatit ovat
\[
A(1, 2), \quad B(-1, -1) \quad \text{ja} \quad C(2, -2).
\]

\begin{kuva}
    kuvaaja.pohja(-2, 3, -3, 3, korkeus = 4, nimiX = "$x$", nimiY = "$y$", ruudukko = True)
    geom.jana((1, 2), (-1, -1))
    geom.jana((-1, -1), (2, -2))
    geom.jana((2, -2), (1, 2))
    kuvaaja.piste((1, 2), "$A$", 45)
    kuvaaja.piste((-1, -1), "$B$", 180)
    kuvaaja.piste((2, -2), "$C$", -45)
\end{kuva}

Koordinaattiakselit leikkaavat toisensa kohtisuoraan pisteessä, jota nimitetään \termi{origo}{origoksi}.
Tuota pistettä voi ajatella koordinaatiston keskuksena.
Vaaka-akselia on yleensä tapana nimittää $x$-akseliksi ja pystyakselia $y$-akseliksi.
Näiden mukaan koko koordinaatistoa kutsutaan toisinaan $xy$-koordinaatistoksi.

Kunkin pisteen koordinaatit määräytyvät siitä, missä kohdassa se on $x$- ja $y$-akselien asteikkoihin verrattuna.
Aivan kuten lukusuoralla kutakin pistettä vastaa tietty reaaliluku $x$ ja päinvastoin, koordinaatistossa kutakin pistettä vastaa yksi yhteen tietty lukupari $(x, y)$.

\subsection{Pisteiden välinen etäisyys}

Geometriassa tärkeää on päästä mittaamaan pituuksia.
Tätä varten on selvitettävä, miten voidaan määrittää kahden koordinaatiston pisteen välinen etäisyys.
Aivan kuten tavallisessa geometriassa, emme voi aina turvautua mittaamiseen; se ei vain yksinkertaisesti aina onnistu. Esim. maan ja auringon etäisyyden määrittämiseen ei pisinkään viivoitin riitä, ja betonikuution lävistäjän pituuden mittaamiseski ei voi suoraan käyttää mittanauhaa. Sen sijaan pyritään selvittämään pisteiden väliset etäisyydet niiden koordinaattien perusteella.

\begin{kuva}
    kuvaaja.pohja(-4, 3, 0, 6, korkeus = 4, nimiX = "$x$", nimiY = "$y$", ruudukko = True)
    piste((1, 2), "$A$")
    piste((1, 5), "$B$")
    piste((-3, 2), "$C$")
\end{kuva}

Helpointa etäisyyden määrittäminen on silloin, kun pisteet ovat samalla vaaka- tai pystysuoralla.
Toisin sanoen niillä on sama $x$- tai $y$-koordinaatti.
Tällöin niiden etäisyys saadaan yksinkertaisesti laskemalla toisistaan poikkeavien koordinaattien erotus.

Esimerkiksi yllä olevassa kuvassa pisteiden $A(1, 2)$ ja $B(1, 5)$ välinen etäisyys on $5-2=3$.
Toisaalta pisteiden $A(1, 2)$ ja $C(-3, 2)$ välinen etäisyys on $1-(-3)=4$.

Etäisyyden tulee olla aina epänegatiivinen.
Jos ei ole varma pisteiden järjestyksestä, voi käyttää itseisarvoja.
Esimerkiksi edellä pisteiden $A$ ja $C$ etäisyys voidaan laskea myös järjestyksessä $|-3-1|=|-4|=4$.

% \begin{esimerkki}
% jotkin helpot etäisyydet
% \end{esimerkki}

\begin{kuva}
    kuvaaja.pohja(-2, 3, -1, 5, korkeus = 4, nimiX = "$x$", nimiY = "$y$", ruudukko = True)
    piste((1, 2), "$A$")
    piste((4, 3), "$B$")
    geom.jana((1, 2), (1, 3))
    geom.jana((1, 3), (4, 3))
    geom.jana((4, 3), (1, 2))
\end{kuva}

Kun kahdella pisteellä on sekä eri $x$-koordinaatit että eri $y$-koordinaatit, etäisyys on määritettävä toisella tapaa.
Nyt voidaan turvautua Pythagoraan lauseeseen ja siihen, että koordinaatisto on suorakulmainen.

Edellä olevassa kuvassa pisteiden $A(1, 2)$ ja $B(4, 3)$ etäisyys saadaan piirtämällä kuvan mukainen suorakulmainen kolmio $ABC$.
Kateetin $AC$ pituus on pisteiden $A$ ja $B$ $x$-koordinaattien erotus eli $4-1=3$.
Kateetin $BC$ pituus on puolestaan $A$ ja $B$ $y$-koordinaattien erotus eli $3-2=1$.
Hypotenuusan $AB$ pituus saadaan Pythagoraan lauseesta:
\[
|AB|=\sqrt{3^2+1^2}=\sqrt{9+1}=\sqrt{10}.
\]
Pisteiden $A$ ja $B$ välinen etäisyys on siis $\sqrt{10}$.
Tätä ei olisi voinut selvittää mittaamalla.

Yleisessä tapauksessa kahden pisteen välinen etäisyys saadaan seuraavasta kaavasta.
\laatikko[Pisteiden $(x_1, y_1)$ ja $(x_2, y_2)$ välinen etäisyys.]{
\[
\sqrt{(x_1-x_2)^2+(y_1-y_2)^2}
\]
}

\begin{esimerkki}
Mitkä ovat luvun alussa olleen kolmion $ABC$ sivujen pituudet?

\begin{esimratk}
Pisteiden koordinaatit olivat $A(1, 2)$, $B(-1, -1)$ ja $C=(2, -2)$.
Kolmion sivujen pituudet ovat sen kärkipisteiden väliset etäisyydet.
Yllä olevaa kaavaa soveltamalla saadaan sivun $AB$ pituudeksi
\[
\sqrt{\bigl(1-(-1)\bigr)^2+\bigl(2-(-1)\bigr)^2}=\sqrt{2^2+3^2}=\sqrt{4+9}=\sqrt{13}.
\]
Sivun $AC$ pituudeksi tulee
\[
\sqrt{(1-2)^2+\bigl(2-(-2)\bigr)^2}=\sqrt{(-1)^2+4^2}=\sqrt{1+16}=\sqrt{17}.
\]
Sivun $BC$ pituudeksi tulee
\[
\sqrt{\bigl((-1)-2\bigr)^2+\bigl((-1)-(-2)\bigr)^2}=\sqrt{(-3)^2+1^2}=\sqrt{9+1}=\sqrt{10}.
\]
\end{esimratk}

\begin{esimvast}
Sivujen pituudet ovat $|AB|=\sqrt{13}$, $|AC|=\sqrt{17}$ ja $|BC|=\sqrt{10}$.
\end{esimvast}
\end{esimerkki}

\subsection{Käyrät ja niiden yhtälöt}

Koordinaatistoon voidaan pisteiden ja janojen lisäksi piirtää myös suoria ja erilaisia käyriä.
Esimerkiksi polynomien kuvaajia piirrettiin ja tutkittiin pitkän matematiikan kurssissa 2 ja suoriin on tutustuttu jo aiemmin.
Yhteistä kaikille näille on se, että koordinaatistoon piirrettyä käyrää (jollaiseksi myös suora voidaan laskea) vastaa jokin yhtälö, jossa esiintyvät tuntemattomat $x$ ja $y$.

Alla olevan kuvan suoraa vastaa yhtälö $y=2x-1$.
Yhtälö tulkitaan siten, että jokaisen suoralla olevan pisteen $(x, y)$ koordinaatit toteuttavat yhtälön.
Esimerkiksi suoran piste $(2, 3)$ toteuttaa yhtälön, sillä $3=2\cdot 2-1$.
Toisaalta piste $(5, 2)$ ei ole suoralla, sillä $2\cdot 5-1=9\neq 2$.


\begin{kuva}
	kuvaaja.pohja(-4, 6, -4, 6, leveys=7)
	kuvaaja.piirra("2*x-1", nimi="$2x-1$")
	piste((1,1), "$(1, 1)$")
\end{kuva}

Kääntäen voidaan sanoa, että jos piste toteuttaa yhtälön $y=2x-1$, se on suoralla.
Yhtälö siis täysin määrittää suoran pisteet.
Etsittäessä pisteitä, jotka muodostavat suoran, riittää tutkia sen yhtälöä.

Monilla tutuilla käyrillä on omat yhtälönsä.
Alle on piirretty paraabeli, jonka yhtälö on $y=x^2-2x+1$.
Aivan kuten suoran tapauksessa, paraabeli koostuu täsmälleen niistä pisteistä, jotka toteuttavat tämän yhtälön.
Toisinaan sanotaan, että paraabeli on \emph{niiden pisteiden joukko, jotka toteuttavat yhtälön $y=x^2-2x+1$}.


\begin{kuva}
	kuvaaja.pohja(-4, 6, -3, 7, leveys=7)
	kuvaaja.piirra("x**2-2*x+1")
\end{kuva}

Tällä kurssilla opitaan käsittelemään erityisesti suoria ja ympyröitä niiden yhtälöiden avulla.
Ympyrän yhtälö muodostetaan vaatimuksesta, että jokainen ympyrän piste on yhtä kaukana ympyrän keskipisteestä.
Tällöin tulee käyttöön edellä opittu kahden pisteen etäisyyden kaava.
Esimerkiksi alla olevassa kuvassa on piirretty ympyrä, jonka yhtälö on $(x-1)^2+(y-2)^2=4$.

\begin{kuva}
	kuvaaja.pohja(-4, 6, -3, 7, leveys=7)
	geom.ympyra((1, 2), 2, nimi="$(x-1)^2+(y-2)^2=4$")
\end{kuva}

Yllä olevasta ympyrän yhtälöstä nähdään, että käyrien yhtälöt eivät ole välttämättä muotoa $y=f(x)$.
Toinen esimerkki on niin sanottu Cartesiuksen lehti, jonka yhtälö on $x^3+y^3-3xy=0$.
Kartesiuksen lehti on piirretty alla olevaan kuvaan.

\begin{kuva}
	kuvaaja.pohja(-2, 3, -2, 3, leveys=7)
	kuvaaja.piirraParametri("(3*t)/(1+t**3)", "(3*t**2)/(1+t**3)", -15, 15)
\end{kuva}


Esimerkiksi piste $(3/2, 3/2)$ on käyrällä, sillä
\[
\left(\frac{3}{2}\right)^3+\left(\frac{3}{2}\right)^3-3\cdot\frac{3}{2}\cdot\frac{3}{2}
=\frac{27}{8}+\frac{27}{8}-\frac{27}{4}=\frac{27+27-54}{8}=0.
\]

\begin{tehtavasivu}

\subsubsection*{Opi perusteet}

\begin{tehtava}
	Määritä pisteiden etäisyys, kun pisteet ovat
	\alakohdat{
		§ $(1,2)$ ja $(1,5)$
		§ $(17,55)$ ja $(17,-1876)$
		§ $(-999999,423)$ ja $(999999,423)$.
	}
	\begin{vastaus}
		\alakohdat{
			§ $3$
			§ $1931$
			§ $1999998$
		}
	\end{vastaus}
\end{tehtava}

\begin{tehtava}
	Määritä pisteiden etäisyys, kun pisteet ovat
	\alakohdat{
		§ $(1,2)$ ja $(2,5)$
		§ $(8,5)$ ja $(17,0)$
		§ $(-12,34)$ ja $(56,-78)$.
	}
	\begin{vastaus}
		\alakohdat{
			§ $\sqrt{10}$
			§ $\sqrt{106}$
			§ $\sqrt{17168} = 4 \sqrt{1073}$
		}
	\end{vastaus}
	\end{tehtava}
	
\begin{tehtava}
	Laske kolmion $ABC$ piiri, kun sen kärjet ovat pisteet $A = (1,1)$, $B = (2,7)$ ja $C = (-4,-1)$.
	\begin{vastaus}
		$\sqrt{65}+10+\sqrt{29}$
	\end{vastaus}
\end{tehtava}

\begin{tehtava}
	Millä $t$:n arvoilla pisteiden $(1,t)$ ja $(-2t,3)$ etäisyys on
	\alakohdat{
		§ $\sqrt{10}$
		§ $4$
		§ $3t$
	}
	\begin{vastaus}
		\alakohdat{
			§ $t = 0$ tai $t = \frac{2}{5}$
			§ $t = \frac{1 \pm \sqrt{31}}{5}$
			§ $t = \frac{-1 \pm \sqrt{41}}{4}$
		}
	\end{vastaus}
\end{tehtava}

\subsubsection*{Hallitse kokonaisuus}

\subsubsection*{Sekalaisia tehtäviä}

TÄHÄN TEHTÄVIÄ SIJOITTAMISTA ODOTTAMAAN

\end{tehtavasivu}
	% yleistä käyristä, esim. Kartesiuksen lehdestä jotain
	% kahden pisteen välinen etäisyys (Pythagoraan lauseella)

\chapter{Suorat}
	\input{content/suora}
	% ratkaistu muoto y = kx + b, kulmakerroin ja vakiotermi
	% nollakohdat ja leikkauspisteet
	% vaaka- ja pystysuorat
	\newpage \section{Suoran yhtälön muut muodot}

\laatikko{
KIRJOITA TÄHÄN LUKUUN

\luettelo
§ suoran yhtälö normaalimuodossa $ax+by+c=0$
§ suoran yhtälö muodossa $y-y_0=k(x-x_0)$, suoran yhtälön muodostaminen pisteiden avulla
\loppu

KIITOS!}

%\subsection*{Valeorigomuoto} % tämä ei ole vakiintunut termi!!!

%Toisinaan suoran yhtälöä on helpompaa tarkastella muodossa $y-y_0=k(x-x_0)$.
%Tämän voidaan ajatella olevan origon kautta kulkeva suora, jos origo olisi
%pisteessä $(x_0, y_0)$. Valeorigomuoto on kätevin silloin, kun tiedämme suoran
%kulmakertoimen ja yhden pisteen, jonka kautta suora kulkee.

%\begin{esimerkki}
%    Esimerkkejä valeorigomuodon käytöstä:
%    \alakohdat
%        § Suoran kulmakerroin on $5$ ja suora kulkee pisteen $(3, 0)$ kautta.
%        \[y-y_0 = k(x-x_0) \ekvi y-0 = 5(x-3) \ekvi y = 5x-15\]
%        § Suoran kulmakerroin on $4$ ja suora kulkee pisteen $(5, 7)$ kautta.
%        \[y-y_0 = k(x-x_0) \ekvi y-7 = 4(x-5) \ekvi y-7 = 4x-20 \ekvi y = 4x-13\]
%    \loppu
%\end{esimerkki}

\subsection*{Suoran yhtälön määrittäminen}

\begin{esimerkki}
Suoran kulmakerroin on $4$ ja se kulkee pisteen $(-3, 2)$ kautta. Määritä suoran yhtälö.

\begin{esimratk}
Olkoon piste $(x, y)$ suoralla. Nyt suoran kulmakerroin on 
\[
\frac{y-2}{x-(-3)}.
\]
Toisaalta tiedetään kulmakertoimen olevan 4. Tästä saadaan yhtälö
\[
\frac{y-2}{x+3}=4.
\]
Ratkaistaan yhtälö:
\begin{align*}
\frac{y-2}{x+3}&=4 \\
y-2&=4(x+3) \\
y-2&=4x+12 \\
y&=4x+14. \\
\end{align*}
\end{esimratk}

\begin{esimvast}
Suoran yhtälö on $y=4x+14$.
\end{esimvast}
\end{esimerkki}

Samanlainen kaava voidaan johtaa yleisemmin. Oletetaan, että suoran kulmakerroin on $k$ ja se kulkee pisteen $(x_0, y_0)$ kautta. Määritetään suoran yhtälö.
Samaan tapaan kuin edellisessä esimerkissä saadaan yhtälö
\[
\frac{y-y_0}{x-x_0}=k.
\]
Ratkaistaan se:
Ratkaistaan yhtälö:
\begin{align*}
\frac{y-y_0}{x-x_0}&=k && \ppalkki{\cdot (x-x_0)}\\
y-y_0&=k(x-x_0) && \ppalkki{+y_0} \\
y-y_0&=k(x-x_0). &&
\end{align*}

\laatikko{
Jos suora kulkee pisteen $(x_0, y_0)$ ja sen kulmakerroin on $k$, suoran yhtälö on
\[y-y_0=k(x-x_0).\]
}

\begin{esimerkki}
Suora kulkee pisteiden $(3, 4)$ ja $(-1, 5)$ kautta. Määritä suoran yhtälö.

\begin{esimratk}
Lasketaan ensin suoran kulmakerroin $k$. Se on
\[
k=\frac{-5+4}{-1-3}=\frac{1}{4}.
\]
Käytetään sitten edellä johdettua suoran yhtälön kaavaa $y-y_0=k(x-x_0)$, missä $(x_0, y_0)$ on jokin suoralla oleva piste. Tässä tapauksessa pisteeksi voidaan valita kumpi tahansa pisteistä $(3, 4)$ ja $(-1, 5)$.
Valitaan vaikkapa $(x_0, y_0)=(3, 4)$. Nyt suoran yhtälö on $y-4=\frac{1}{4}(x-3)$. Se saadaan edelleen muotoon $y=\frac{1}{4}(x-3)+4$.
\end{esimratk}
\begin{esimvast}
Suoran yhtälö on $y=\frac{1}{4}(x-3)+4$.
\end{esimvast}
\end{esimerkki}

\subsection*{Normaalimuoto}

Suoran yhtälön voi kirjoittaa monessa eri muodossa. Esimerkiksi $y=3x-4$, $y-4=3x$ ja $3x-y-4=0$ ovat kaikki saman suoran yhtälöitä. Viimeistä niistä kutsutaan 
suoran yhtälön normaalimuodoksi.

\laatikko{Suoran yhtälön normaalimuoto on
\[ ax+by+c=0, \]
missä joko $a \neq 0$ tai $b \neq 0$.}

\begin{esimerkki}
Kirjoitetaan suoran $y=-5x+2$ yhtälö normaalimuodossa.
\begin{esimratk}
\begin{align*}
y&=-5x+2 && \ppalkki{+5x} \\
5x+y&=2 && \ppalkki{-2} \\
5x+y-2&=0 &&
\end{align*}
\end{esimratk}
\begin{esimvast}
Suoran yhtälön normaalimuoto on $5x+y-2=0$.
\end{esimvast}
\end{esimerkki}


Mikä tahansa suora voidaan kirjoittaa normaalimuodossa. Esimerkiksi pystysuoraa $x=-3$ ei voi kirjoittaa muodossa $y=kx+b$, sillä suoralla ei ole kulmakerrointa. Suora voidaan kuitenkin kirjoittaa normaalimuodossa. Sen normaalimuoto on $x+3=0$.

\begin{tehtavasivu}

\sarjaA % Opi perusteet

\begin{tehtava}
Mikä on suoran yhtälö ratkaistussa muodossa?
\alakohdat
§ $3x + y -5 = 0$
§ $-15x + 5y + 20 =0$
§ $2x - 3y + 4 = 0$
\loppu
\begin{vastaus}
\alakohdat
§ $y=-3x+5$
§ $y=3x-4$
§ $y=\frac{2x}{3} + \frac{4}{3}$
\loppu
\end{vastaus}
\end{tehtava}

\begin{tehtava}
Mikä on suoran yhtälö normaalimuodossa?
\alakohdat
§ $y=-15x+2$
§ $2y=11x+7$
§ $2y+5x-8=13y-6x-8$
\loppu
\begin{vastaus}
\alakohdat
§ $15x+y-2=0$
§ $11x-2y+7=0$
§ $-11x+11y=0$
\loppu
\end{vastaus}
\end{tehtava}

\begin{tehtava}
Suoran kulmakerroin on $\frac{1}{2}$ ja suora kulkee pisteen
\alakohdat
§ $(-12, 4)$
§ $(3, 9)$
\loppu
kautta. Mikä on suoran yhtälö?
\begin{vastaus}
\alakohdat
§ $y=\frac{1}{2}x+10$
§ $y=\frac{1}{2}x+\frac{15}{2}$
\loppu
\end{vastaus}
\end{tehtava}

\begin{tehtava}
Suora kohtaa $x$-akselin, kun $x=15$, ja suora kulkee pisteen
\alakohdat
§ $(-13, 1)$
§ $(7, 8)$
\loppu
kautta. Mikä on suoran yhtälö?
\begin{vastaus}
\alakohdat
§ $y=\frac{-1}{28}x+\frac{15}{28}$
§ $y=-x+15$
\loppu
\end{vastaus}
\end{tehtava}

\begin{tehtava}
Suora kohtaa $y$-akselin, kun $y=10$, ja suora kulkee pisteen
\alakohdat
§ $(-12, 2)$
§ $(4, 14)$
\loppu
kautta. Mikä on suoran yhtälö?
\begin{vastaus}
\alakohdat
§ $y=\frac{2}{3}x+10$
§ $y=x+10$
\loppu
\end{vastaus}
\end{tehtava}

\begin{tehtava}
Mikä on pisteiden
\alakohdat
§ $(1, -2)$ ja $(3, 1)$
§ $(0, 0)$ ja $(-4, 4)$ kautta kulkevan suoran yhtälö?
\loppu
\begin{vastaus}
\alakohdat
§ $y=\frac{3}{2}x-\frac{7}{2}$
§ $y=-x$
\loppu
\end{vastaus}
\end{tehtava}

\begin {tehtava}
Suora kulkee pisteiden $(3, 4)$ ja $(\sqrt{3}, 1)$ kautta. Määritä suoran kulmakerroin.
\begin {vastaus}
$\frac{\sqrt{3}-1}{\sqrt{3}}$
\end {vastaus}
\end {tehtava}

\sarjaB % Hallitse kokonaisuus

\begin{tehtava}
Tutki ovatko pisteet  
\alakohdat
§ $(1, -5)$, $(4, -23)$ja $(4, -239)$
§ $(7, 3)$, $(-2, 10)$ ja $(-3, 90)$ samalla suoralla?
\loppu
\begin{vastaus}
\alakohdat
§ kyllä
§ ei
\loppu
\end{vastaus}
\end{tehtava}

\begin{tehtava}
Määritä luku $t$ niin, että pisteet $(-t+3, -4)$, $(6, t-5)$ ja $(5, -4)$ ovat samalla suoralla.
\begin{vastaus}
$t=-2$ tai $t=1$
\end{vastaus}
\end{tehtava}

\begin{tehtava}
Voidaan osoittaa, että tason pisteet $A$, $B$ ja $C$ ovat samalla suoralla jos ja vain jos jokin pisteiden välisistä etäisyyksistä, $AB$, $BC$ tai $CA$, on kahden muun summa. (Jos suoralta kiinnitetään ensin kaksi pistettä, tämä voidaan ottaa myös suoran määritelmäksi.) Ovatko pisteet samalla suoralla
	\alakohdat
		§ $(13,5)$, $(20,0)$ ja $(7,9)$
		§ $(-10,-8)$, $(15,22)$ ja $(5,10)$?
	\loppu
	\begin{vastaus}
		\alakohdat
			§ Eivät
			§ Ovat
		\loppu
	\end{vastaus}
\end{tehtava}

\sarjaC % Syvennä osaamistasi

\begin{tehtava}
Suora kulkee pisteiden $(\frac{1}{r+4}, r^2)$ ja $(\frac{1}{r+4}+1, 16)$ kautta. Mikä on suoran $y$-koordinaatti reaaliluvun $r \ne -4$ funktiona, kun $x=1$?
\begin{vastaus}
$r+12$
\end{vastaus}
\end{tehtava}

\sarjaD % Sekalaisia tehtäviä

TÄHÄN TEHTÄVIÄ SIJOITTAMISTA ODOTTAMAAN

\end{tehtavasivu}

	% esitys y-y_0=k(x-x_0)
	% esitys ax + by + c = 0 (normaalimuoto)
	\newpage \input{content/suora_asema}
	% suorien keskinäinen asema, yhdensuuntaiset suorat
	% suoralle ja sen normaalille k_1 * k_2 = -1
	\newpage \section{Pisteen etäisyys suorasta}

\laatikko{
KIRJOITA TÄHÄN LUKUUN

\luettelo
§ TEHTY pisteen etäisyyden suorasta laskeminen yhdenmuotoisilla
kolmioilla
§ TEHTY pisteen etäisyys suorasta -kaava: $d=\frac{|ax_0+by_0+c|}{\sqrt{a^2+b^2}}$
§ sovelluksia
§ kaavan todistuksen voi laittaa tähän osioon tai liitteeksi,
käytetään yhdenmuotoisia kolmioita. TODISTUS ON (muttei kolmioilla)
\loppu

KIITOS!}

Pisteen etäisyydellä suorasta tarkoitetaan pisteen ja mielivaltaisen suoran pisteen pienintä mahdollista etäisyyttä.
Jos tunnetaan jokin vaaka- tai pystysuora suora ja jokin koordinaatiston piste, kyseisen pisteen etäisyys annetusta suorasta on helppo määrittää.

\begin{minipage}{0.45\textwidth}
\begin{kuva}
    kuvaaja.pohja(-2, 3, -1, 5, korkeus = 4, nimiX = "$x$", nimiY = "$y$", ruudukko = True)
    with palautin():
        vari("lightgray")
        kuvaaja.piirraParametri("1.7", "t", a = 1, b = 3.3)
    kuvaaja.piirra("1", nimi = "$y=1$", suunta = 0)
    kuvaaja.piste((1.7, 3.3), "$(1,7; 3,3)$", 95)
\end{kuva}

Etäisyys on $3,3-1=2,3$.
\end{minipage}
\begin{minipage}{0.45\textwidth}
\begin{kuva}
    kuvaaja.pohja(-2, 3, -1, 5, korkeus = 4, nimiX = "$x$", nimiY = "$y$", ruudukko = True)
    with palautin():
        vari("lightgray")
        kuvaaja.piirraParametri("t", "3.3", a = -1, b = 1.7)
    kuvaaja.piirraParametri("-1", "t", a = -1, b = 5, nimi = "$x = -1$", suunta = 90)
    kuvaaja.piste((1.7, 3.3), "$(1,7; 3,3)$", 90)
\end{kuva}

Etäisyys on $1,7-(-1)=2,7$.
\end{minipage}

Jos suora on kalteva, etäisyyden määrittäminen ei ole näin suoraviivaisesti.
Seuraavaksi tutustutaan erääseen tapaan tämän pulman ratkaisemiseksi.

\subsection*{Pisteen etäisyys suorasta yhdenmuotoisten kolmioiden avulla}

Tarkastellaan esimerkkinä suoraa $l$, jonka normaalimuotoinen yhtälö on $3x-4y -12=0$.
Selvitetään pisteen $P=(8, 5)$ etäisyys suorasta $l$.

\begin{kuva}
    kuvaaja.pohja(-1, 10, -4, 5, nimiX = "$x$", nimiY = "$y$", ruudukko = True)
    with palautin():
        vari("lightgray")
        kuvaaja.piirraParametri("8+0.75*t", "5-t", a = 0, b = 1.28)
    kuvaaja.piirra(".75*x-3", nimi = "$l$", kohta = 2, suunta = -45)    
    kuvaaja.piste((8,5), "$P$", -135)
    kuvaaja.piste((8.96, 3.72), "$Q$", -45)
\end{kuva}

%    kuvaaja.piirra(".75*x-3", nimi = "$3x -4y- 12=0$", kohta = 2, suunta = -45)    

Nähdään, että pienin mahdollinen etäisyys pisteestä suoralle on sellaisen janan $PQ$ pituus, joka kulkee pisteestä $P$ suoralle $l$ siten, että se on kohtisuorassa suoraan $l$ nähden. (Mikäli tämä ei tunnu itsestään selvältä, voit havainnoillistaa tilannetta esimerkiksi piirtämällä ympyrän, jonka keskipiste on $P$ ja joka sivuaa suoraa $l$ tarkalleen yhdessä pisteessä $Q$.)

Tehtävä ratkeaa, kun huomataan, että valitsemalla janan $PQ$ lisäksi suoralta sopivan kolmannen pisteen $R$ saadaan kolmio, joka on yhdenmuotoinen suoran $l$ ja koordinaattiakselien rajaaman kolmion kanssa. Alla on tilanteesta kuva:

%%%[EI OLE VALMIS, KUVIA JA TEKSTISELITYSTÄ VOISI VIELÄ MIETTIÄ]
%%%FIXME: (^Koskien yo:) tekstiä hieman uusiksi (pyritty avaamaan päättelyä), voisiko silti tätä vielä hioa?
%%%Lisäksi korjattu alla todistuksessa sekaisin menneet OA/OB. (Kuvastahan näkee että OA =4!) -aa-m-sa

%    kuvaaja.pohja(-1, 10, -4, 5)
\begin{kuva}\\
    kuvaaja.pohja(-1, 10, -4, 5, ruudukko = False)
    kuvaaja.piirra(".75*x-3", a = -1, b = 10)
    kuvaaja.piirraParametri("0", "t", a = -3, b = 0)
    kuvaaja.piirraParametri("t", "0", a = 0, b = 4)
    kuvaaja.piirraParametri("8", "t", a = 3, b = 5)
    kuvaaja.piirraParametri("8+0.75*t", "5-t", a = 0, b = 1.28)
    kuvaaja.piste((8,5), "$P$", 180)
    kuvaaja.piste((4,0), "$A$", -45)
    kuvaaja.piste((0,-3), "$B$", -45)
    kuvaaja.piste((0,0), "$O$", 135)
    kuvaaja.piste((8.96, 3.72), "$Q$", -45)
    kuvaaja.piste((8,3), "$R$", -45)
\end{kuva}

Haluamamme etäisyys on suoralle $l$ piirretty normaali $PQ$. Piste $R$ on suoralla $l$ siten, että $PR$ on $y$-akselin suuntainen. $A$ ja $B$ ovat suoran $l$ ja $x$- ja $y$-akselin leikkauspisteet ja $O$ on origo.

Kolmiot $ABO$ ja $PRQ$ ovat yhdenmuotoisia, sillä molemmat ovat suorakulmaisia ja lisäksi kulmat $ABO$ ja $PRQ$ ovat samankohtaisina yhtä suuret. Koska kolmiot ovat yhdenmuotoiset, saadaan verranto
\begin{align*}
\frac{PQ}{PR} &=\frac{OA}{AB}\\
\intertext{josta saadaan etäisyydelle PQ yhtälö}
PQ &=\frac{OA}{AB}\cdot PR.
\end{align*}

Kolmion $OAB$ sivut $OA$ ja $AB$ selviävät, kun ratkaistaan, missä pisteissä suora leikkaa $x$- ja $y$-akselit.
Asettamalla suoran yhtälössä $x=0$ saadaan
\[
3x-4\cdot 0=12, \quad \text{josta} \quad x=\frac{12}{3}=4.
\]
Pisteen $A$ koordinaatit ovat siis $(4, 0)$. Toisaalta kun $x=0$, saadaan
\[
3\cdot 0-4\cdot y=12, \quad \text{josta} \quad y=-\frac{12}{4}=-3.
\]
Pisteen $B$ koordinaatit ovat siis $(0, -3)$. Koska piste $O$ on origo $(0,0),$ tunnetaan nyt sivut $OA=4$ ja $OB=3$. Sivu $AB$ saadaan Pythagoraan lauseen perusteella
\[
AB=\sqrt{OA^2+OB^2}=\sqrt{4^2+3^2}=\sqrt{25}=5.
\]

Enää on selvitettävä sivu $PR$. Tämä on sama kuin pisteiden $P$ ja $R$ välinen etäisyys.

Pisteen $R$ $x$-koordinaatti on sama kuin pisteen $P$, eli 8. Koska $R$ on suoralla $l$, sen $y$-koordinaatti saadaan suoran yhtälöstä:
\begin{align*}
3\cdot 8-4y & =12 \\
-4y & =12-3\cdot 8 \\
-4y & =-12 \\
y & =3. \\
\end{align*}
Nyt siis $R=(8, 3)$. Pisteiden $P$ ja $R$ välinen etäisyys on siis $5-3=2$, ja tämä on sivun $PR$ pituus.

Kun verrannosta saatuun yhtälöön sijoitetaan tunnetut sivujen pituudet, saadaan
\begin{align*}
PQ &=\frac{OA}{AB}\cdot PR  && \ppalkki OA= 4, AB=5, PR= 2\\
PQ & =\frac{4\cdot 2}{5} = \frac{8}{5}.
\end{align*}
Siispä pisteen $P$ etäisyys suorasta $l$ on $\dfrac{8}{5}$.

Tässä valittu piste $R$ ei suinkaan ollut ainoa vaihtoehto. Minkä (tai mitkä?) muun pisteen olisi voinut valita?

\subsection*{Pisteen etäisyys suorasta kaavan avulla}

Pisteen etäisyydelle suorasta johtaa myös yleinen kaava, jonka käyttö voi olla joskus edellä esitettyä yhdenmuotoisten kolmioiden menetelmän soveltamista helpompaa. Kaavan voi johtaa esimerkiksi yhdenmuotoisilla kolmioilla tai suoraan analyyttisesti (ks. harjoitustehtävät ja liite). Kolmas, vektoreihin perustuva todistus esitetään kurssilla MAA5 Vektorit.

Jos suoran yhtälö on annettu normaalimuodossa $ax+by+c=0$ ja pisteen koordinaatit ovat $(x_0, y_0)$, etäisyys $d$ saadaan seuraavasta kaavasta.
\laatikko[Pisteen etäisyys suorasta]{
\[
d=\frac{|ax_0+by_0+c|}{\sqrt{a^2+b^2}}
\]
}

%%% Kaavan johtamiset: 
%%%  * yhdenmuotoisilla kolmioilla harjoitustehtäväksi (esimerkki ja muutamat vihjeet riittänevät)
%%%  * analyyttinen todistus suht' teknistä pyörittelyä, liitteeseen (kuten aiemmin kommenteissa ehdotettu todistusta tästä liitteeseen)

\begin{esimerkki} Lasketaan aiemman esimerkin pisteen $P=(8, 5)$ etäisyys suorasta $l$, jonka normaalimuotoinen yhtälö on $3x-4y-12=0$.
\begin{esimratk}
Käytetään kaavaa, jolloin $a=3$, $b=-4$ ja $c=12$, sekä $x_0=8$ ja $y_0=5$. Kaavan mukaan etäisyys on
\[
d=\frac{|ax_0+by_0+c|}{\sqrt{a^2+b^2}}
=\frac{|3\cdot 8-4\cdot 5-12|}{\sqrt{3^2+(-4)^2}}
=\frac{|24-20-12|}{\sqrt{9+16}}=\frac{|-8|}{\sqrt{25}}
=\frac{8}{5}.
\]
\end{esimratk}
\begin{esimvast}
Etäisyys on $\dfrac{8}{5}$.
\end{esimvast}
\end{esimerkki}

\begin{esimerkki} Etsitään ne pisteet, joiden $x$-koordinaatti on 6 ja joiden etäisyys suorasta $-6x+8y+3=0$.
\begin{esimratk}
Piste, jonka $x$-koordinaatti on 6, on muotoa $(6, y)$. Sijoitetaan tämä etäisyyden kaavaan ja sievennetään.
Nyt $A=-6$, $B=8$, $C=3$, $x_0=6$ ja $y_0=y$.
\[
d=\frac{|-6\cdot 6+8y+3|}{\sqrt{(-6)^2+8^2}}
=\frac{|-36+8y+3|}{\sqrt{36+64}}
=\frac{|8y-33|}{\sqrt{100}}
=\frac{|8y-33|}{10}.
\]
Tehtävänannon mukaan etäisyyden pitäisi olla $d=6$. Tästä saadaan yhtälö
\[
\frac{|8y-33|}{10}=6 \quad \text{eli} \quad |8y-33|=60.
\]
Tämä itseisarvoyhtälö ratkeaa jakautumalla kahteen tapaukseen:
\begin{align*}
8y-33 & =60 & &\text{tai} & 8y-33 & =-60 \\
8y & =99 & & & 8y & =-27 \\
y & =\frac{99}{8} & & & y & =-\frac{27}{8}.
\end{align*}
\end{esimratk}
\begin{esimvast}
Pisteet ovat $\bigl(6, \frac{99}{8}\bigr)$ ja $\bigl(6, -\frac{27}{8}\bigr)$.
\end{esimvast}
\end{esimerkki}


\begin{tehtavasivu}

\subsubsection*{Opi perusteet}

\begin{tehtava} % tarkistettu / Niko
Määritä annetun pisteen etäisyys annetusta suorasta.
\alakohdat
§ $(-1,-3),5x-2y+7 = 0 $
§ $(1,1),3x+5y-8 = 0 $
§ $(-7,17), x + 6 = 0$
§ $(e,\pi),\pi x+ e y = 0$
\loppu
\begin{vastaus}
\alakohdat
§ $\frac{8}{\sqrt{29}}$
§ $0$, eli piste on suoralla
§ $1$
§ $\frac{2\pi e}{\sqrt{\pi^2+e^2}}$
\loppu
\end{vastaus}
\end{tehtava}

\begin{tehtava}
Määritä annetun pisteen etäisyys annetusta suorasta.
\alakohdat
§ $(0,-3),y = x+3$
§ $(0,0),y = 2x+7$
§ $(-7,-7), y = 3$
§ $(-1,2), y = 5x-19$
\loppu
\begin{vastaus}
\alakohdat
§ $3\sqrt{2}$
§ $\frac{7\sqrt{5}}{5}$
§ $10$
§ $\sqrt{26}$
\loppu
\end{vastaus}
\end{tehtava}

\begin{tehtava}
Kumpi annetuista pisteistä on lähempänä annettua suoraa?
\alakohdat
§ Pisteet $(3,0)$ ja $(4,3)$, suora $5x-3y-4=0$
§ Pisteet $(2,4)$ ja $(3,3)$, suora $6x+7y-2=0$
§ Pisteet $(5,6)$ ja $(6,5)$, suora $x-y+1=0$
§ Pisteet $(3,7)$ ja $(3,0)$, suora $6x-y-15=0$
\loppu
\begin{vastaus}
\alakohdat
§ $(4,3)$
§ $(3,3)$
§ $(5,6)$
§ $(3,0)$
\loppu
\end{vastaus}
\end{tehtava}

\begin{tehtava}
Kumpi annetuista pisteistä on lähempänä annettua suoraa?
\alakohdat
§ Pisteet $(3,1)$ ja $(4,-2)$, suora $y=2x+1$
§ Pisteet $(-2,2)$ ja $(3,5)$, suora $y=5x+4$
§ Pisteet $(6,5)$ ja $(4,-1)$, suora $y=4x-10$
§ Pisteet $(-5,7)$ ja $(5,7)$, suora $y=2x+6$
\loppu
\begin{vastaus}
\alakohdat
§ $(3,1)$
§ $(-2,2)$
§ $(4,-1)$
§ $(5,7)$
\loppu
\end{vastaus}
\end{tehtava}

\subsubsection*{Hallitse kokonaisuus}

\begin{tehtava}
Millä vakion $a$:n arvoilla pisteen $(a,1)$ etäisyys suorasta $x+2ay+a^2 = 0$ on $a$?

\begin{vastaus}
$a = \sqrt{\frac{11}{3}}+1$
\end{vastaus} 
\end{tehtava}

\subsubsection*{Sekalaisia tehtäviä}

\begin{tehtava}
Määritä kaikki pisteet $(x,y)$, jotka ovat yhtä etäällä suorista
\alakohdat
§ $y = 0$ ja $y-2x = 0$
§ $y - 3x - 1 = 0$ ja $2x+4y -3= 0$
\loppu
Pistejoukkoja kutsutaan \emph{kulmanpuolittajiksi}, miksi?
\begin{vastaus}
\alakohdat
§ $(\sqrt{5}+1)x +2y = 0$ ja $(\sqrt{5}-1)x-2*y = 0$
§ $14(\sqrt{2}-1)x-14y+10+\sqrt{2} = 0$ ja $14(\sqrt{2}+1)x+14y-10+\sqrt{2} = 0$
Syntyneet suorat puolittavat alkuperäisten suorien määrämät kulmat.
\loppu
\end{vastaus}
\end{tehtava}

\begin{tehtava}
\alakohdat
§ Määritä kaikki ne suorat, joiden etäisyydet pisteistä $(0,1)$ ja $(2,-1)$ ovat yhtäsuuret.
§ Entä jos vaaditaan, että suora on yhtä etäällä myös pisteestä $(-1,-1)$
\loppu
\begin{vastaus}
\alakohdat
§ Suoria ovat pisteiden keskipisteen $(1,0)$ kautta kulkevat suorat, eli suorat muotoa $ax+by-a$, sekä pisteiden kautta kulkevan suoran kanssa yhdensuuntaiset suorat, eli suorat muotoa $y+x-c = 0$
§ Ne kolme suoraa, jotka kulkevat pisteiden määrittämän kolmion sivujen joidenkin kahden keskipisteen kautta, eli suorat $y = 0$, $2x-y-2 = 0$, sekä $2x+2y+1 = 0$
\loppu
\end{vastaus}
\end{tehtava}

\begin{tehtava}
Määritä kaikki suorat, joiden etäisyys origosta on $1$
\begin{vastaus}
Suorat muotoa $ax+by \pm \sqrt{a^2+b^2} = 0$
\end{vastaus}
\end{tehtava}

\begin{tehtava}
  Määritä ne pisteet, joiden etäisyys suorasta $2x-y+4 = 0$ on $a$.
  \begin{vastaus}
    Pisteet $(x_0, y_0)$, jotka muodostavat suorat $2x_0 -y_0 +4 \mp \sqrt{5}a = 0$.
  \end{vastaus}
\end{tehtava}


\begin{tehtava}
Matti haluaa uimaan pitkälle suoralle joelle. Hän tietää, että jos hän matkaa suoraan pohjoiseen, matkaa kertyy $10$ kilometriä, mutta jos hän päättää lähteä suoraan itään, hänen on käveltävä vain $7,0$ kilometriä. Kuinka kaukana Matti on joesta?
\begin{vastaus}
$5,7$ kilometrin etäisyydellä
\end{vastaus}
\end{tehtava}

\begin{tehtava}
Matti on (autiolla) kolmion muotoisella saarella. Saaren eteläkärki on $5.0$ kilometrin päässä Matista, suoraan etelään, toinen saaren kärjistä löytyy $8.0$ kilmetrin etäisyydeltä suoraan idästä, ja kolmas suoraan luoteesta, $11$ kilometrin etäisyydeltä. Jos Matti haluaa lähimmälle rannalle, mille rannoista hänen on suunnattava, ja kuinka paljon matkaa kertyy?
\begin{vastaus}
Lähin ranta on etelä- ja luoteiskärkiä yhdistävä, ja matkaa sille kertyy n. $2.6$ kilometriä
\end{vastaus}
\end{tehtava}

\begin{tehtava}
  Todista kaava pisteen pisteen $P=(x_0,y_0)$ etäisyydelle yleisestä suorasta $l$, jonka yhtälö on $ax + by +c = 0$ yleistämällä yhdenmuotoisten kolmioiden menetelmää.
  \alakohdat
      § Voit olettaa ensin, että (esimerkin kuvan merkinnöin) koordinaattiakselien ja suoran $l$ väliin jäävä kolmio $OAB$ on olemassa.
      § Entä kun kolmiota $OAB$ ei ole olemassa? Milloin näin tapahtuu?
  \loppu
  \begin{vastaus}
    \alakohdat
        § \emph{Vihje.} Etsi nyt suoran $l$ yhtälön avulla yhtälöt pisteiden $A, B$ ja $R$ koordinaateille. $A, B:$ Origo on edelleen $(0,0)$. $R:$ $R = (x_r, y_r) = (x_0, y_r)$, ratkaise $y_r$.
        § \emph{Vihje.} Janan $AB$ pituus.
    \loppu
  \end{vastaus}
\end{tehtava}

\begin{tehtava}
	Pisteen etäisyyden suorasta voi laskea myös kurssi Geometria (MAA3) tietojen avulla. Jos pisteet $A$ ja $B$ ovat suoralla $l$, kolmion $ABP$ ala voidaan laskea kolmion sivujen pituuksien, mutta myös sivun $AB$ pituuden, ja sitä vastaan piirretyn korkeusjanan pituuden avulla. Kun vielä huomataan, että korkeusjanan pituus on yhtäsuuri kuin pisteen $P$ etäisyys suorasta $l$. Heronin kaava antaa yhteyden kolmion pinta-alan ja sen sivujen pituuksien välille:
	\[
	A = \sqrt{p(p-a)(p-b)(p-c)},
	\]
	missä $a$, $b$ ja $c$ ovat kolmion sivujen pituudet, ja $p = (a+b+c)/2$.
	
	Määritä origon etäisyys suorasta, jolla on pisteet $(-1,1)$ ja $(3,4)$.
	\begin{vastaus}
		$\frac{7}{5}$
	\end{vastaus}
\end{tehtava}


TÄHÄN TEHTÄVIÄ SIJOITTAMISTA ODOTTAMAAN

\end{tehtavasivu}

	% kaava pisteen etäisyydelle suorasta

\chapter{Toisen asteen käyrät}
	\input{content/diskriminantti}
	% diskriminantin pikakertaus
	\newpage \section{Ympyrä}

\laatikko{
KIRJOITA TÄHÄN LUKUUN

\luettelo{
§ ...
}

KIITOS!}

\begin{kuva}
    kuvaaja.pohja(-3.5, 3.5, -3.5, 3.5, korkeus = 4, nimiX = "$x$", nimiY = "$y$", ruudukko = True)
    kuvaaja.piirraParametri("3*cos(t)", "3*sin(t)", a = 0, b = 2*pi)
    piste((3*cos(0.75), 3*sin(0.75)), "(x, y)", -120)
\end{kuva}

Kuvaan on piirretty käyrä, jonka pisteiden etäisyys origosta on 3. Käyrän huomataan olevan ($3$-säteinen) ympyrä. Piste $(x, y)$ on tällä ympyrällä täsmälleen silloin, kun sen etäisyys origosta on 3. Toisin sanoen täytyy päteä $\sqrt{x^2+y^2}=3$. Kun yhtälön molemmat puolet korotetaan vielä toiseen potenssiin, saadaan $x^2+y^2=9$. $3$-säteisen ympyrän yhtälö on siis $x^2+y^2=9$.

\laatikko{\termi{ympyrä}{Ympyrä} muodostuu tason pisteistä, jotka ovat vakioetäisyydellä jostakin kiinteästä pisteestä. Tätä kiinteää pistettä kutsutaan ympyrän \termi{keskipiste}{keskipisteeksi} ja vakioetäisyyttä ympyrän \termi{säde}{säteeksi}.}

Johdetaan yhtälö ympyrälle, jonka keskipiste on $(x_0, y_0)$ ja säde $r$. Piste $(x, y)$ on ympyrälllä täsmälleen silloin, jos sen etäisyys pisteestä $(x_0, y_0)$ on $r$. Pisteiden välisen etäisyyden kaavalla saadaan pisteiden $(x, y)$ ja $(x_0, y_0)$ väliseksi etäisyydeksi $\sqrt{(x-x_0)^2+(y-y_0)^2}$. Tuloksena on siis yhtälö
\[
\sqrt{(x-x_0)^2+(y-y_0)^2}=r.
\]
Koska säde $r$ ei voi olla negatiivinen, voidaan yhtälön molemmat puolet korottaa toiseen potenssiin ja saadaan yhtäpitävä yhtälö
\[
(x-x_0)^2+(y-y_0)^2=r^2.
\]

Erityisesti, jos ympyrän keskipiste on origo eli $(x_{0}, y_{0})= (0, 0)$, yhtälö saa muodon

\[
x^{2}+y^{2} = r^{2}.
\]

\laatikko{
Jos ympyrän keskipiste on $(x_{0}, y_{0})$ ja säde $r > 0$, ympyrän yhtälö on
\[
(x-x_{0})^{2}+(y-y_{0})^{2} = r^{2}.
\]
}

Jos säde $r$ on nolla, yhtälön toteuttaa vain piste $(x_{0}, y_{0})$. Tällöin kyseessä on vain yksi piste.


%%% FIX ME Eikö tämä ja alempi kommentti ole aivan pätevä esimerkki + yksikköympyrän määritelmä?
%\begin{esimerkki}
%Ympyrän keskipiste on $(-4, 1)$ ja säde $5$. Määritä ympyrän yhtälö ja hahmottele ympyrä koordinaatistoon.
%\begin{esimratk}
%Ympyrän yhtälö saadaan käyttämällä edellä annettua kaavaa. Nyt $x_0=-4$, $y_0=1$ ja $r=5$. Ympyrän yhtälöksi saadaan $(x-(-4))^2+(y-1)^2=25$ eli
%\[
%(x+4)^2+(y-1)^2=25.
%\]
%\end{esimratk}
%\begin{esimvast}
%Ympyrän yhtälö on $(x+4)^2+(y-1)^2=25$.
%\end{esimvast}
%\end{esimerkki}

%Tähän kuva ympyrästä.

%Edellisen esimerkin ympyrän yhtälö voidaan kirjoittaa myös toisenlaisessa muodossa.
%\begin{align*}
%(x+4)^2+(y-1)^2&=25 \\
%x^2+8x+16+y^2-2y+1&=25 \\
%x^2+y^2+8x-2y-8&=0.
%\end{align*}

\begin{esimerkki}
Piirrä kuva ympyrästä, jonka yhtälö on
\[
(x-1)^2+(x+1)^2=4.
\]
\begin{esimratk}
Muokataan ympyrän yhtälöä niin, että keskipiste ja säde näkyvät suoraan:
\[
(x-1)^2+(x-(-1))^2=2^2.
\]
Tästä nähdään, että keskipiste on $(1, -1)$ ja säde 2. Nyt kuva on helppo piirtää.

\begin{kuva}
    kuvaaja.pohja(-2, 4, -4, 2, korkeus = 4, nimiX = "$x$", nimiY = "$y$", ruudukko = True)
    kuvaaja.piirraParametri("2*cos(t)+1","2*sin(t)-1", a = 0, b = 2*pi)
\end{kuva}

\end{esimratk}
\end{esimerkki}

%\begin{esimerkki}
%Ympyrän $\Gamma_{1}$\footnote{sdf} keskipiste on $(3, -4)$ ja säde $\sqrt{2}$, ja ympyrän $\Gamma_{2}$ keskipiste origo ja säde 1. Määritä ympyröiden yhtälöt. Hahmottele ympyrät koordinaatistoon.

%\begin{esimratk}
%Edellisen mukaan ympyrän $\Gamma_{1}$ yhtälö on
%\[
%(x-3)^{2}+(y-(-4))^{2} = (\sqrt{2})^{2}
%\]
%eli sievennettynä
%\[
%(x-3)^{2}+(y+4)^{2} = 2.
%\]
%$\Gamma_{2}$:n yhtälö saadaan vastaavasti:
%\[
%x^{2}+y^{2} = 1.
%\]
%Edelliselle yksikkösäteiselle origokeskiselle ympyrälle on vakiintunut nimitys \emph{yksikköympyrä}.

%\end{esimratk}
%\end{esimerkki}

%$(x_0, y_0)$-keskisen, $r$-säteisen ympyrän yhtälön muuntaminen normaalimuotoon:
%\begin{align*}
%(x - x_0)^2 + (y - y_0)^2 = r^2 && \ppalkki \textnormal{Käytetään 		%binomimuistikaavoja.}\\
%x^2 - 2x_0 x + x_0^2 + y^2 - 2y_0 y + y_0^2 = r^2 && \ppalkki -r^2\\
%x^2 + y^2 - 2x_0 x - 2y_0 y - r^2 + x_0^2 + y_0^2 = 0 && \\
%\end{align*}

Aina keskipiste ja säde eivät näy ympyrän yhtälöstä suoraan.
Esimerkiksi yhtälö
\[
x^2+6x+y^2-4y=3
\]
on erään ympyrän yhtälö.
Tämä nähdään täydentämällä summattavat $x^2+6x$ ja $y^2-4y$ neliöiksi.

Neliöksi täydentäminen opittiin kurssissa MAA2, mutta kerrataan se tässä vielä.
Aloitetaan lausekkeesta $x^2+6x$. Se on lähes sama kuin binomin $x+3$ neliö, sillä $(x+3)^2=x^2+6x+9$. Ainoastaan vakiotermit poikkeavat toisistaan ja tämän voi korvata lisäämällä ympyrän yhtälön molemmille puolille luvun $9$: 
\begin{align*}
x^2+6x+y^2-4y &= 3 && \ppalkki +9 \\
(x^2+6x+9)+(y^2-4y) &= 12 && \\
(x+3)^2+(y^2-4y) &= 12.&& 
\end{align*}

Siirrytään sitten tarkastelemaan summattavaa $y^2-4y$. Se on puolestaan melkein binomin $y-2$ neliö, sillä $(y-2)^2=y^2-4y+4$. Binomin neliö saadaan näkyviin lisäämällä yhtälön molemmille puolille luku $4$:
\begin{align*}
(x+3)^2+(y^2-4y) &= 12 && \ppalkki +4\\
(x+3)^2+(y^2-4y+4) &= 16 && \\
(x+3)^2+(y-2)^2 &= 16 && \\
(x+3)^2+(y-2)^2 &= 4^2.&& 
\end{align*}

Nyt huomataan, että kyseessä on $(-3, 2)$-keskisen $4$-säteisen ympyrän yhtälö.

\begin{esimerkki}
Ympyrän yhtälö on $x^2-8x+y^2+5y+3=0$. Määritä ympyrän keskipiste ja säde.
\begin{esimratk}
Yhtälö muuttuu muotoon $x^2-8x+y^2+5y=-3$. Suoritetaan sitten neliöksi täydentäminen:
\begin{align*}
x^2-8x+y^2+5y&=-3 && \ppalkki +16\\
x^2-8x+16+y^2+5y&=-3+16 && \\
(x^2-4)^2+y^2+5y&=13 && \ppalkki +\frac{25}{4}\\
(x^2-4)^2+y^2+5y+\frac{25}{4}&=\frac{77}{4} && \ppalkki +\frac{25}{4}\\
(x^2-4)^2+\left(y+\frac{5}{4}\right)^2&=\frac{102}{4} && \\
(x^2-4)^2+\left(y+\frac{5}{4}\right)^2&=\frac{51}{2} && 
\end{align*}
Nähdään, että keskipiste on $(4, -5/4)$ ja säde $\sqrt{51/2}$.
\end{esimratk}
\begin{esimvast}
Keskipiste on $(4, -5/4)$ ja säde $\sqrt{51/2}$.
\end{esimvast}
\end{esimerkki}

\begin{esimerkki}
Onko yhtälö $x^2-4x+y^2+2y+6=0$ ympyrän yhtälö?
\begin{esimratk}
Suoritetaan neliöksitäydennys:
\begin{align*}
x^2-4x+y^2+2y+6&=0 && \ppalkki -6\\
x^2-4x+y^2+2y&=-6 && \ppalkki +4\\
x^2-4x+4+y^2+2y&=-2 && \\
(x^2-2)^2+y^2+2y&=-2 && \ppalkki +1\\
(x^2-2)^2+y^2+2y+1&=-1 && \\
(x^2-2)^2+(y+1)^2&=-1. &&
\end{align*}
Nyt nähdään, että kyseessä ei voi olla ympyrän yhtälö, sillä säteen neliö ei voi olla negatiivinen.
\end{esimratk}
\begin{esimvast}
Kyseessä ei ole ympyrän yhtälö.
\end{esimvast}
\end{esimerkki}

Edellä tehty neliöön korotus voidaan voidaan suorittaa yleisesti muotoa
\[
x^2+ax+y^2+by+c = 0
\]
oleville yhtälöille. Täydentämällä $x^2+ax$ ja $y^2+by$ neliöiksi saadaan
\begin{align*}
x^2+ax+y^2+by+c &= 0 && \ppalkki +\frac{a^2}{4}+\frac{b^2}{4}-c \\
\Big(x^2+ax+\frac{a^2}{4}\Big)+\Big(y^2+by+\frac{b^2}{4}\Big) &= \frac{a^2}{4}+\frac{b^2}{4}-c  \\
\Big(x+\frac{a}{2}\Big)^2+\Big(y+\frac{b}{2}\Big)^2 &= \frac{a^2}{4}+\frac{b^2}{4}-c
\end{align*}
Jos yhtälön oikea puoli eli $\frac{a^2}{4}+\frac{b^2}{4}-c \geq 0$ yhtälö kuvaa $\sqrt{\frac{a^2}{4}+\frac{b^2}{4}-c}$-säteistä $(-\frac{a}{2}, -\frac{b}{2})$-keskistä ympyrää. Jos oikea puoli on nolla, yhtälö kuvaa vastaavaa pistettä. Jos se on negatiivinen, yhtälö ei kuvaa mitään käyrää: yhtälön vasen puoli on (kahden neliön summana) aina suurempi tai yhtä suuri kuin nolla, joten oikean puolen ollessa negatiivinen mikään reaalilukupari ei toteuta yhtälöä.

%\laatikko{Yhtälöt muotoa
%\[
%x^2+ax+y^2+by+c = 0
%\]
%kuvaavat ympyrää, pistettä tai tyhjää joukkoa.}

\begin{tehtavasivu}

\paragraph*{Opi perusteet}

\begin{tehtava}
	Ympyrän keskipiste on $(0, 0)$ ja säde $5$. Muodosta ympyrän yhtälö.
	\begin{vastaus}
		$x^2+y^2=25$
	\end{vastaus}
\end{tehtava}

\begin{tehtava}
	Ympyrän keskipiste on $(1, -4)$ ja säde $10$. Muodosta ympyrän yhtälö.
	\begin{vastaus}
		$(x-1)^2+(y+4)^2=100$ tai auki kirjoitettuna $x^2+y^2-2x+8y-83=0$ % auki kirjoitettu vai aukikirjoitettu
	\end{vastaus}
\end{tehtava}

\begin{tehtava}
	Määritä keskipiste ja säde.
	\alakohdat{
		§ $(x-3)^2+(y+7)^2=12$
		§ $x^2+y^2=49$
	}
	\begin{vastaus}
		\alakohdat{
			§ keskipiste $(3, -7)$, säde $2\sqrt{3}$
			§ keskipiste $(0, 0)$, säde $7$
		}
	\end{vastaus}
\end{tehtava}

\begin{tehtava}
	Määritä keskipiste ja säde.
	\alakohdat{
		§ $x^2+y^2-10x+16y+72=0$
		§ $x^2+y^2+8x-22y+129=0$
	}
	\begin{vastaus}
		\alakohdat{
			§ keskipiste $(5, -8)$, säde $\sqrt{17}$
			§ keskipiste $(-4, 11)$, säde $2\sqrt{2}$
		}
	\end{vastaus}
\end{tehtava}

\begin{tehtava}
	Määritä ympyrän $(x+10)^2+y^2=2$ keskipiste ja säde ja ratkaise ympyrän yhtälöstä $y$. 
	\begin{vastaus}
		keskipiste $(-10, 0)$, säde $\sqrt{2}$, $y=\pm\sqrt{2-(x+10)^2}$ 
	\end{vastaus}
\end{tehtava}

\begin{tehtava}
	Ympyrän keskipiste on origo ja säde $3$. Mitkä seuraavista pisteistä ovat ympyrän kehällä?
	\alakohdat{
		§ $(10, -2)$
		§ $(-3, 0)$
		§ $(2, \sqrt{5})$
		§ $(1, 3)$
	}
	\begin{vastaus}
		b) ja c)
	\end{vastaus}
\end{tehtava}

\begin{tehtava}
	Määritä $k$ niin, että lauseke $(x-3)^2+(y+3)^2=k$ on
	\alakohdat{
			§ ympyrä
			§ $\sqrt{7}$-säteinen ympyrä
			§ origon kautta kulkeva ympyrä?
	}
	\begin{vastaus}
		\alakohdat{
			§ $k>0$
			§ $k=7$
			§ $k=18$
		}
	\end{vastaus}
\end{tehtava}

\begin{tehtava}
	Ympyrän keskipiste on $(3,5)$ ja piste $(4,8)$ on ympyrällä. Määritä ympyrän yhtälö.
	\begin{vastaus}
		$(x-3)^2+(y-5)^2 = 10$
	\end{vastaus}
\end{tehtava}

\begin{tehtava}
	Tutki, mitä yhtälöiden kuvaajat esittävät.
	\alakohdat{
		§ $x^2+y^2-6x+4y+4=0$
		§ $x^2+y^2+14x-6y+10=0$
	}
	\begin{vastaus}
		\alakohdat{
			§ ympyrä
			§ piste
		}
	\end{vastaus}
\end{tehtava}

\paragraph*{Hallitse kokonaisuus}

\begin{tehtava}
	Määritä ympyrän keskipiste ja säde.
	\alakohdat{
		§ $(x+t)^2+(y+u)^2=k, k>0$
		§ $(x+2)^2+(y-7)^2=-8$
	}
	\begin{vastaus}
		\alakohdat{
			§ keskipiste $(-t, -u)$, säde  $\sqrt{k}$
			§ ei ole ympyrä
		}
	\end{vastaus}
\end{tehtava}

\begin{tehtava}
	Ympyrä sivuaa $y$-akselia pisteessä $(0, -1)$ ja kulkee pisteen $(3, 2)$ kautta. Mikä on ympyrän yhtälö?
	\begin{vastaus}
		$(x-3)^2+(y+1)^2=9$
	\end{vastaus}
\end{tehtava}

\begin{tehtava}
	Kolmen tunnetun pisteen kautta kulkevan ympyrän yhtälö voidaan määrittää monella eri tavalla. Ympyrä kulkee pisteiden $(1, 6), (-2, 5)$ ja $(5, 4)$ kautta. Määritä ympyrän yhtälö seuraavilla tavoilla:
	\alakohdat{
		§ Tutki, millä vakioiden $x_0$, $y_0$ ja $r$ arvoilla pisteet ovat ympyrällä.
		§ Ympyrän keskipiste on minkä tahansa kahden pisteen keskinormaalilla, joten sen voi määrittää kahden eri keskinormaalin leikkauspisteenä.
	}
	\begin{vastaus}
		$(x-1)^2+(y-1)^2=16$
	\end{vastaus}
\end{tehtava}

\begin{tehtava}
	Jana, jonka pituus on $t$ liikkuu koordinaatistossa siten, että sen toinen pää on $x$-akselilla ja toinen $y$-akselilla. Mitä käyrää pitkin liikkuu janan keskipiste?
	\begin{vastaus}
		$x^2+y^2=\frac{1}{4}t^2$
	\end{vastaus}
\end{tehtava}

\begin{tehtava}
	Millä $c$:n reaaliarvoilla yhtälö $x^2-2xc+y^2-2c-2 = 0$ esittää ympyrää? Mikä on tällöin ympyrän keskipiste ja säde? Todista, että ympyrä kulkee tällöin kahden $c$:stä riippumattoman pisteen kautta.
	\begin{vastaus}
		Kaikilla, keskipiste $(c,0)$, säde $\sqrt{(c-1)^2+1}$. Ympyrät kulkevat aina pisteiden $(1,\pm 1)$ kautta.
	\end{vastaus}
\end{tehtava}

\begin{tehtava}
	Ympyrä voidaan määritellä myös monella muulla yhtäpitävällä tavalla
	\alakohdat{
		§ Jos $A = (1,0)$ ja $B = (-1,0)$, määritä kaikki ne pisteet $P$, joilla $AP$ ja $BP$ ovat kohtisuorassa.
		§ Jos $A = (2,0)$ ja $B = (-1,0)$, määritä kaikki ne pisteet $P$, joille $\frac{AP}{BP} = 2$.
		§ Jos $A = (3,0)$ ja $B = (-1,0)$, määritä kaikki ne pisteet $P$, joille $AP^2+BP^2 = 10$.
	}
	\begin{vastaus}
		\alakohdat{
			§ Ympyrän $x^2+y^2 = 1$ pisteet (lukuunottamatta pisteitä $A$ ja $B$)
			§ Ympyrän $(x+2)^2+y^2 = 4$ pisteet
			§ Ympyrän $(x-1)^2+y^2 = 1$ pisteet
		}
	\end{vastaus}
\end{tehtava}

\paragraph*{Sekalaisia tehtäviä}

TÄHÄN TEHTÄVIÄ SIJOITTAMISTA ODOTTAMAAN







\end{tehtavasivu}

	% ympyrän yhtälö määritelmästä
	% ensin origokeskeinen
	% keskipiste ja säde muissa tapauksissa neliöksi täydentämällä
	\newpage \section{Ympyrä ja suora}

\laatikko{
KIRJOITA TÄHÄN LUKUUN

\luettelo{
§ TEHTY - ympyrän ja suoran leikkauspisteiden ratkaiseminen 
§ ympyrän tangetin määrittäminen sekä kehällä olevan (kohtisuorassa sädettä vastaan TEHTY) että
sen ulkopuolisen pisteen kautta (kaksi tapaa: pisteen etäisyys suorasta -kaava (TEHTÄVÄKSI?) tai diskriminantti = 0 TEHTY)
§ TEHTY kahden ympyrän leikkauspisteiden ratkaiseminen yhtälöparilla
}

KIITOS!}

\begin{esimerkki}
Määritä suorien $x+2y-3=0$ ja ympyrän $(x-1)^2+(y+1)^2=4 $ leikkauspisteet.

\begin{esimratk}

Ympyrän ja suoran leikkauspisteet ovat ne pisteet $(x, y)$, jotka ovat sekä suoralla että ympyrällä, eli toteuttavat molempien yhtälöt, eli yhtälöryhmän
$$\left\{    
    \begin{array}{rcl}
        x+2y-3 &=&0 \\
        (x-1)^2+(y+1)^2 &=&4 \\
    \end{array}
    \right.$$
Vaikka yhtälö ei olekaan tutun lineaarinen, myös sitä voi lähestyä sijoitusmenetelmällä. Ratkaistaan ensimmäisestä yhtälöstä $x$ ja saadaan
\[
x = -2y+3
\]
Kun tämä sijoitetaan toiseen yhtälöön ja kerrotaan auki syntyy toisen asteen yhtälö $y$:n suhteen:
\begin{align*}
(-2y+3-1)^2+(y+1)^2=4 \\
(-2y+2)^2+(y+1)^2=4 \\
(-2y)^2-2\cdot 2y\cdot 2 +2^2+y^2+2\cdot y+1^2=4 \\
4y^2-8y+4+y^2+2y+1-4 = 0 \\
5y^2-6y+1 = 0
\end{align*}
Ratkaisukaavalla
\[
y = \frac{-(-6)\pm\sqrt{(-6)^2-4\cdot 5\cdot 1}}{2\cdot5}
\]
eli
\begin{align*}
y = \frac{6\pm\sqrt{16}}{10} \\
y = 1 \vee y = \frac{1}{5}
\end{align*}
Kun nämä $y$:n arvot sijoitetaan suoran yhtälöön saadaan vastaavast $x$:n arvot:
\begin{align*}
x = -2\cdot 1+3 \vee x = -2\cdot\frac{1}{5}+3 \\
x = 1 \vee x = \frac{13}{5}
\end{align*}
eli saatiin 2 ratkaisua: $(x, y) = (1, 1)$ ja $(x, y) = (\frac{13}{5}, \frac{1}{5})$. Tarkistamalla on hyvä vielä todeta, että pisteet todella ovat leikkauspisteitä.

\begin{kuva}
    kuvaaja.pohja(-3, 5, -4, 4, korkeus = 4, nimiX = "$x$", nimiY = "$y$", ruudukko = True)
    kuvaaja.piirraParametri("2*cos(t)+1","2*sin(t)-1", a = 0, b = 2*pi)
    kuvaaja.piirra("(-x+3)/2")
	
\end{kuva}
\begin{esimvast}
Leikkauspisteet ovat $(1, 1)$ ja $(\frac{13}{5}, \frac{1}{5})$.
\end{esimvast}

\end{esimratk}
\end{esimerkki}

Esimerkin avulla huomattiin, että suoralla ja ympyrällä voi olla kaksi leikkauspistettä. Suorasta ja ympyrästä riippuen päädytään toisen asteen yhtälöön, jolla on joko 0, 1 tai 2 ratkaisua. Jos leikkauspisteitä on tasan yksi, sanotaan, että suora sivuaa ympyrää tai suora on ympyrän \termi{tangentti}{tangentti}.

Kuva ympyrästä ja kolmesta suorasta; yksi tangentti, yksi leikkaa kahdessa pisteessä ja yksi ei leikkaa ympyrää.

\laatikko{Suoralla ja ympyrällä on nolla, yksi tai kaksi leikkauspisteitä.

Suora on ympyrän \emph{tangentti}, jos suoralla ja ympyrällä on tasan yksi yhteinen piste.
}

\begin{esimerkki}

Määritä $(2, 2)$-keskisen 5-säteisen ympyrän pisteen $(-5, 3)$ kautta kulkevat tangentit.

\begin{esimratk}
Suora on ympyrän tangentti, jos sillä ja ympyrällä on tasan yksi yhteinen piste. Jos pisteen $(-5, 3)$ suora ei ole $y$-akselin suuntainen, se voidaan esittää muodossa
\[
y-3 = k(x-(-5))
\]
eli
\[    
y = kx+5k+3.
\]
(Oletus voidaan tehdä, sillä selvästi nähdään ettei pisteen $(-5, 3)$ kautta kulkeva pystysuora suora ole kyseisen ympyrän tangentti.)

Kun lisäksi muistamme ympyrän yhtälön $(x-x_0)^2+(y-y_0)^2 = r^2$, suoran ja ympyrän leikkauspisteille saadaan yhtälöpari

\[
\left\{    
    \begin{array}{rcl}
        y &=& kx+5k+3 \\
        (x-2)^2+(y-2)^2 &=& 25 
\end{array}
    \right.
\]
    
Sijoitetaan ensimmäisen yhtälön lauseke $y$:lle toiseen yhtälöön ja saadaan toisen asteen yhtälö $x$:n suhteen
\begin{align*}
(x-2)^2+(kx+5k+3-2)^2&=25 \\
(x-2)^2+(kx+5k+1)^2&=25 \\
x^2-4x + 4+(kx)^2+kx\cdot 5k+kx \quad &\\
+5k\cdot kx+(5k)^2+5k+kx+5k+1& = 25\\
(k^2 +1)x^2+(10k^2+2k-4)x+25k^2+10k-20& = 0. \\
\end{align*}
Tällä yhtälöllä on $x$:n suhteen tasan yksi ratkaisu kun polynomin diskriminantti on nolla, eli
\begin{align*}
D &= b^2 -4ab \\
&= (10k^2+2k-4)^2-4\cdot(k^2 +1)\cdot(25k^2+10k-20) \\ 
&= 100k^4+20k^3-40k^2 + 20k^3 + 4k^2 -8k -40k^2 -8k +16 - 4(k^2 +1)(25k^2+10k-20)\\
&= 100k^4 + 40k^3 - 76k^2 - 16k + 16 - 4(25k^4+10k^3-20k^2 + 25k^2+10k-20) \\
&= 100k^4 + 40k^3 - 76k^2 - 16k + 16  - 100k^4 - 40k^3 - 20k^2 - 40k + 80 \\
& = -96k^2-56k+96 \\
&=-8(12k^2+7k-12) = 0.
\end{align*}
Toisen asteen yhtälön ratkaisukaavalla saadaan
\begin{align*}
k &= \frac{-7\pm \sqrt{7^2-4\cdot 12\cdot (-12)}}{2\cdot 12} \\
k &= \frac{-7\pm \sqrt{625}}{24} \\
k &= \frac{-7\pm 25 }{24}, \\
\end{align*}
joten
\begin{align*}
k = \frac{18}{24} =  \frac{3}{4} \; &\textrm{tai} \; k = -\frac{32}{24} = -\frac{4}{3}
\intertext{Kulmakertoimia vastaavat siis tangenttisuorat}
y = \frac{3}{4}x+5\cdot \frac{3}{4}+3 \; &\textrm{ja} \; y = -\frac{4}{3}x+5\cdot \Big(-\frac{4}{3}\Big)+3 \\
y = \frac{3}{4}x+\frac{27}{4} \; &\textrm{ja} \; y = -\frac{4}{3}x-\frac{11}{3}
\end{align*}
\end{esimratk}
\begin{esimvast}
Suorat $y = \frac{3}{4}x+\frac{27}{4}$ ja $y = -\frac{4}{3}x-\frac{11}{3}$
\end{esimvast}
\end{esimerkki}

Vastaavalla tavalla voidaan määrittää myös kahden ympyrän leikkauspisteet

\begin{esimerkki}
Määritä ympyröiden $(x-1)^2+(y+5)^2 = 13$ ja $(x+2)^2+(y+4)^2 = 17$ leikkauspisteet.

\begin{esimratk}
Ympyröiden leikkauspisteet toteuttavat yhtälöparin
\[
\left\{    
    \begin{array}{rcl}
        (x-1)^2+(y+5)^2 = 13 \\
        (x+2)^2+(y+4)^2 = 17 \\
    \end{array}
    \right.
\]
Kun binomien neliöt kerrotaan auki, sievennetään ja yhtälöt vähennetään toisistaan saadaan $x$:n ja $y$:n välille ensimmäisen asteen yhtälö
\[
\left\{    
    \begin{array}{rcl}
        x^2-2x+1+y^2+10y+25 = 13 \\
        x^2+4x+4+y^2+8y+16 = 17 \\
    \end{array}
    \right.
\]
\[
\left\{    
    \begin{array}{rcl}
        x^2-2x+y^2+10y= -13 \\
        x^2+4x+y^2+8y= -3 \\
    \end{array}
    \right.
\]
joten
\[
-6x+2y=-10
\]
Tämä suoran yhtälö vastaa oikeastaan ympyröiden leikkauspisteiden kautta kulkevaa suoraa. Tästä voidaan ratkaista $y$ $x$:n suhteen ja sijoittaa jompaan kumpaan alkuperäisistä yhtälöistä.
\begin{align*}
y = 3x-5 \\
(x-1)^2+((3x-5)+5)^2 = 13 \\
x^2-2x+1+(3x)^2 = 13 \\
10x^2-2x-12 = 0 \\
5x^2-x-6 = 0 \\
\end{align*}
Nyt leikkauspisteiden $x$-koordinaatit voidaan ratkaista toisen asteen yhtälön ratkaisukaavalla
\begin{align*}
x &= \frac{-(-1)\pm\sqrt{(-1)^2-4\cdot 5\cdot (-6)}}{2\cdot 5} \\
&= \frac{1\pm\sqrt{121}}{10} \\
&= \frac{1\pm 11}{10} \\
\end{align*}
eli
\[
x =  \frac{6}{5} \textrm{  tai  } x = -1
\]
Nyt suoran yhtälöstä voidaan ratkaista vastaavat $y$:n arvot.
\begin{align*}
y = 3\cdot \frac{6}{5}-5 &\textrm{  tai  } y = 3\cdot (-1)-5 \\
y = -\frac{7}{5}x &\textrm{  tai  } y = -8
\end{align*}
\end{esimratk}
\begin{esimvast}
Ympyröiden leikkauspisteet ovat $(\frac{6}{5}, -\frac{7}{5})$ ja $(-1,-8)$.
\end{esimvast}
\end{esimerkki}

Jälleen esimerkistä nähdään, että riippuen syntyvän toisen asteen yhtälön diskriminantista kahdella ympyrällä voi olla $0$, $1$ tai $2$ leikkauspistettä.


%%FIXME: kuvan voisi päivittää liittymään esimerkkiin?
\begin{kuva}
    kuvaaja.pohja(-7, 5, -11, 3, korkeus = 5, nimiX = "$x$", nimiY = "$y$", ruudukko = True)
    kuvaaja.piirraParametri("3.606*cos(t)+1","3.606*sin(t)-5", a = 0, b = 2*pi)
    kuvaaja.piirraParametri("4.123*cos(t)-2","4.123*sin(t)-4", a = 0, b = 2*pi)
	
\end{kuva}

Ympyrän tangentti saadaan yksikäsitteisesti myös kun tiedetään ympyrän kehän piste jossa tangentti sivuaa ympyrää:

\begin{esimerkki}
Mikä on ympyrää $(x-3)^2 + (y -1)^2 = 18$ pisteessä $(6,4)$ sivuavan tangentin yhtälö?
\begin{esimratk}
Koska tangentti kulkee pisteen $(6,4)$ kautta, jolloin sen yhtälö on 
\begin{align*}
y - 4 &= k_1(x-6) \\
y &= k_1x -6k_1 +4.
\end{align*}
Voisimme sijoittaa tämän edellisten esimerkkien tapaan ympyrän yhtälöön, mutta toinen (tässä tapauksessa helpompi) tapa on huomata tangentin olevan kohtisuorassa ympyrän sädettä eli keskipisteen $(3, 1)$ (jonka voimme lukea suoraan ympyrän yhtälöstä) ja pisteen $(6,4)$ välistä janaa vasten. Nämä kaksi pistettä määrittävät suoran, jonka kulmakerroin $k_2$ on
\begin{align*}
k_2 = \frac{y_2 - y_1}{x_2 - x_1} = \frac{4-1}{6-3} = \frac{3}{3} = 1.
\end{align*}
Koska suoran ja sen normaalin kulmakertoimien tulo on $k_1 k_2 = -1$, saadaan tangentin kulmakertoimeksi $k_1 = k_1 \cdot 1 = -1$, jonka voimme sijoittaa tangentin yhtälöön.
\end{esimratk}
\begin{esimvast}
Tangentin yhtälö on $y = -x +10$.
\end{esimvast}
\end{esimerkki}

%%%KUVA? %%%https://www.wolframalpha.com/input/?i=y%20%3D%20%2Dx%20%2B10%2C%20(x%2D3)%5E2%20%2B%20(y%20%2D1)%5E2%20%3D%2018

\begin{tehtavasivu}

\subsubsection*{Opi perusteet}

\begin{tehtava}
Määritä suoran ja ympyrän leikkauspisteet, jos suoran yhtälö, ja ympyrän keskipiste ja säde ovat
\alakohdat{
§ $3x-2y = 1$, $(-1,2)$, $4$
§ $x+1 = 0$, $(-6,-6)$, $10$
§ $x+7y-23 = 0$, $(7,-3)$, $5$
§ $5x+12y-13 = 0$, $(0,0)$, $5$
}
vastaavasti.
\begin{vastaus}
\alakohdat{
§ $(-1,-2)$ ja $(\frac{35}{13},\frac{46}{13})$
§ $(-1,-6+5\sqrt{3})$, $(-1,-6-5\sqrt{3})$
§ Käyrät eivät leikkaa
§ $(\frac{5}{13},\frac{12}{13})$
}
\end{vastaus}
\end{tehtava}

\begin{tehtava}
Määritä kaikki ympyröiden $ (x-3)^2+y^2= 10$, $(x+1)^2+(y-3)^2 = 18$ sekä $(x-6)^2+(y-2)^2 = 8$ leikkauspisteet (pareittain).
\begin{vastaus}
$(\frac{82-9\sqrt{79}}{50},-3\frac{4\sqrt{79}-17}{50})$, $(\frac{82+9\sqrt{79}}{50},3\frac{4\sqrt{79}+17}{50})$,
$(\frac{123-2\sqrt{295}}{26}, 3\frac{10+\sqrt{295}}{26})$
$(\frac{123+2\sqrt{295}}{26}, 3\frac{10-\sqrt{295}}{26})$
$(\frac{16}{5}, \frac{12}{5})$
\end{vastaus}
\end{tehtava}

\begin{tehtava}
Määritä parametrin $k$ arvot siten, että suora $y=x+k$ on ympyrän  $ x^2+y^2= 2$ tangentti.
\begin{vastaus}
$k = \pm sqrt{2} $,
\end{vastaus}
\end{tehtava}


\subsubsection*{Hallitse kokonaisuus}
\begin{tehtava}
	Määritä annetun ympyrän tangentit, jotka kulkevat annetun pisteen kautta.
	\alakohdat{
		§ Piste $(1, -1)$, ympyrä $(x-3)^2 + (y+1)^2 = 2$
		§ Piste $(-4, -2)$, ympyrä $(x+5)^2 + (y+6)^2 = 17$
		§ Piste $(3, 1)$, ympyrä $(x-4)^2 + y^2 = 1$
		§ Piste $(2, 4)$, ympyrä $x^2 + (y-3)^2 = 8$
	}
	\begin{vastaus}
		\alakohdat{
			§ $y=x-2$ ja $y=-x$
			§ $y= -1/4 x -3$
			§ $y=1$ ja $x=3$
			§ Ei ole.
		}
	\end{vastaus}
\end{tehtava}

\begin{tehtava}
Määritä yksikköympyrän $x^2+y^2= 1$ pisteeseen $(x_{0}, y_{0} )$ piirretyn tangentin normaalimuotoinen yhtälö. Entä jos ympyrä on $r$-säteinen?
\begin{vastaus}
$x_0x+y_0y=1 $. $r$-säteisellä ympyrällä $x_0x+y_0y=r^2$
\end{vastaus}
\end{tehtava}

\begin{tehtava}
Yksikköympyrälle $x^2+y^2=1$ piirretään tangentit pisteestä $(0, a)$. Millä $a$:n arvoilla tangentteja on 
\alakohdat{
§ kaksi
§ yksi
§ nolla?
}
Määritä myös tangenttien yhtälöt.
\begin{vastaus}
\alakohdat{
§ $a > 1$ tai $a < -1$
§ $a = \pm1$
§ $ -1 < a < 1 $ 
}
Tangenttien yhtälöt ovat $ y = \pm \sqrt{a^2-1}x+a$
\end{vastaus}
\end{tehtava}

\begin{tehtava}
\alakohdat{
§ Pisteen $P$ etäisyys $O$-keskisestä $r$-säteisestä ympyrästä on $d$ $(d > r) $. Kuinka pitkiä ovat ympyrälle pisteestä $P$ piirretyt tangentit?
§ Yksikköympyrälle ($x^2+y^2 = 1$) ja ympyrälle $(x+3)^2+(y-2)^2 = 2$ piirretään tangentit. Määritä kaikki pisteet $P = (x,y)$, joista piirretyt tangentit ovat yhtä pitkät. (Tätä pistejoukkoa kutsutaan yleensä ympyröiden \emph{radikaaliakseliksi}.)
}

\begin{vastaus}
\alakohdat{
§ $\sqrt{d^2-r^2}$
§ $3x-2y+6$
}
\end{vastaus}
\end{tehtava}

\begin{tehtava}
Määritä kaikkien pisteen $(1,0)$ kautta kulkevien ympyrän $x^2+y^2 = 4$ jänteiden keskipisteiden joukko.
	\begin{vastaus}
		Ympyrä $(x-\frac{1}{2})^2+y^2 = \frac{1}{4}$. Vinkki: Parametrisoi kaikki 			pisteen $(1,0)$ kautta kulkevat suorat, jatutki miten keskipisteiden $x$ ja 		$y$ -koordinaatit riippuvat parametrista.
	\end{vastaus}
\end{tehtava}

\subsubsection*{Sekalaisia tehtäviä}


TÄHÄN TEHTÄVIÄ SIJOITTAMISTA ODOTTAMAAN

\begin{tehtava}
Suoran ulkopuolisesta pisteestä $P$ piirrettyjen tangenttien ja ympyrän sivuamispisteet voidaan määrittää myös monella muulla tavalla:

Pisteen etäisyys suorasta -kaavalla: Jos suora on ympyrän tangentti, sen etäisyyden suorasta on oltava yhtä suuri kuin ympyrän säde.

Määrittämällä tangenttien pituudet: Koska tangentit ja ympyrän säde ovat kohtisuorassa, tangentien pituus voidaan määrittää ympyrän keskipisteen, pisteen $P$ ja sivuamispisteen muodostamasta suorakulmaisesta kolmiosta. Nyt sivuamispisteet voidaan määrittää kahden ympyrän leikkauspisteinä.

Klassiseen tyyliin Thaleen lausetta hyödyntäen: Voidaan osoittaa, että jos $M$ on pisteiden $P$ ja ympyrän keskipisteen $O$ keskipiste, $M$-keskinen ympyrä, joka kulkee pisteiden $P$ ja $O$ kautta leikkaa alkuperäistä ympyrää halutuissa sivuamispisteissä.

Parametrisoimalla kaikki tangentit: Kaikki ympyrän tangentit voidaan myös parametrisoida valitsemalla ympyrältä mielivaltainen piste ja määrittämällä sen kautta kulkeva tangentti (ks. teht ??). Sitten riittää tarkistaa, mitkä tangenteista kulkevat pisteen $P$ kautta.

Malliratkaisun variaatiolla: Parametrisoidaan kaikki $P$:n kautta kulkevat. Nyt saadaan malliratkaisun tavoin yhtälö leikkauspisteille. Sen sijaan, että nyt määritettäisiin, milloin syntyvän toisen asteen yhtälön diskriminantti on nolla, oletetaan, että se on nolla, jolloin leikkauspisteet saadaan määritettyä parametrin funktiona. Nyt riittää enää tarkistaa, milloin nämä pisteet todellakin ovat ympyrällä.

\alakohdat{
§ Jos $P = (0,0)$, $O = (6,4)$ ympyrän säde on $5$, määritä sivuamispisteet näillä viidellä tavalla.
§ Määritä vastaavien tangenttien yhtälöt.
}
\begin{vastaus}
\alakohdat{
§ $(\frac{81-30\sqrt{3}}{26},\frac{54+45\sqrt{3}}{26})$ ja $(\frac{81+30\sqrt{3}}{26},\frac{54-45\sqrt{3}}{26})$
§ $(27-10\sqrt{3})y-(18+15\sqrt{3})x = 0$ ja $(27+10\sqrt{3})y-(18-15\sqrt{3})x = 0$
}
\end{vastaus}
\end{tehtava}

\begin{tehtava}
Kaksi ympyrää leikkaa kohtisuorasti, jos ne leikkaavat, ja niiden leikkauspisteisiin piirretyt tangentit ovat kohtisuorassa. Leikkaavatko ympyrät kohtisuorasti
\alakohdat{
§ $(x+3)^2+(y+3)^2= 13$ ja $(x+2)^2+(y-1)^2 = 5 $
§ $(x-2)^2+y^2 = 16$ ja $(x+3)^2+y^2 = 9$
§ $(x+7)^2+(y-2)^2 = 23$ ja $(x+4)^2+(y+3)^2 = 1$?
}  
\begin{vastaus}
\alakohdat{
§ Eivät
§ Kyllä
§ Eivät
}
\end{vastaus}
\end{tehtava}

\begin{tehtava}
Matti ajaa autolla $(1,0)$-keskistä ympyrärataa, mutta kuin taikaiskusta renkaista lähtee pito ja kauhukseen Matti alkaa luisua irtoamispisteestä ympyräradan tangentin suuntaan. Matti saa pysäytettyä autonsa pisteeseen $(5,5)$. Kuinka pitkä ympyrärata on, jos se on yhtä pitkä kuin Matin luisuma matka?
\begin{vastaus}
$\frac{2\sqrt{41}\pi}{\sqrt{4\pi^2+1}}$
\end{vastaus}
\end{tehtava}

\begin{tehtava}
Todista analyyttisen geometrian keinoin tangenttien tärkeät perusominaisuudet
\alakohdat{
§ Ympyrälle piirretty tangentti ja tangentin sivuamispisteeseen piirretty säde ovat kohtisuorassa.
§ Tangentit ovat säteen etäisyydellä ympyrän keskipisteestä.
§ Suoran ulkopuolisesta pisteestä piirretyt kaksi tangenttia ovat yhtä pitkiä.
}
Vinkki: Yleisyyttä menettämättä voidaan ympyrä asettaa origokeskiseksi ja yksisäteiseksi, ja tangentti valita kulkemaan sopivan pisteen kautta.
\begin{vastaus}
Vinkki on hyvä
\end{vastaus}
\end{tehtava}

\begin{tehtava}
\alakohdat{
§ Määritä ympyrän $(y-r)^2+x^2 = r^2$ pisteen $(1,0)$ kautta kulkevat tangentit.
§ Minkä käyrän muodostavat vastaavat sivuamispisteet, kun $r$ saa kaikki positiiviset kokonaislukuarvot.
}
	\begin{vastaus}
		\alakohdat{
			§ Suorat $y = 0$ ja $2rx-(r^2-1)y-2r = 0$.
			§
		}
	\end{vastaus}
\end{tehtava}


%%%Sarjassamme liian helppoja todistuksia:
\begin{tehtava}
Edellä esimerkissä todettiin, että suoralla ja ympyrällä voi olla nolla, yksi tai kaksi leikkauspistettä riippuen siitä, montako ratkaisua on toisen asteen yhtälöllä joka syntyy kun kirjoitetaan ympyrän ja suoran yhtälöt yhtälöpariksi. Totea, että \emph{minkä tahansa} ympyrän ja suoran yhtälöistä todellakin aina saadaan toisen asteen yhtälö.
\begin{vastaus}
Ympyrän ja suoran yleiset yhtälöt.
\end{vastaus}
\end{tehtava}

\begin{tehtava}
Määritä ympyröiden $x^2+y^2 = 4$ ja $x^2+(y-4)^2 = 1$ yhteiset tangentit.

\begin{vastaus}
$\pm\sqrt{15}x+y-8= 0$ ja $\pm\sqrt{7}x+3y-8$. Vinkki: Hyödynnä pisteen etäisyys suorasta -kaavaa.
\end{vastaus}
\end{tehtava}

\begin{tehtava}
Määritä kaikkien niiden janojen keskipisteiden joukko, joiden toinen päätepiste on suoralla $x = 2$, toinen ympyrällä $x^2+y^2 = 2$, ja jotka kulkevat (tai niiden jatke kulkee) origon kautta. (Käyrä tunnetaan nimellä \textit{Conchoid of Nicomedes})
	\begin{vastaus}
		$4(x-1)^2(x^2+y^2) = x^2$
	\end{vastaus}
	
\end{tehtava}


\end{tehtavasivu}
	% suoran ja ympyrän leikkauspisteet
	% tangentit
	\newpage \section{Paraabeli}

\laatikko{
KIRJOITA TÄHÄN LUKUUN

\luettelo
§ käyrän $y = ax^2+by+c$ kuvaaja on paraabeli
%%%terminologia? kuvaaja - funktion kuvaaja - käyrä ovatko samoja eivät?
§ mainitaan geometrinen määritelmä
§ paraabelin yhtälön huippumuoto $y-y_0=a(x-x_0)^2$
\loppu

KIITOS!}

\laatikko{
\termi{paraabeli}{Paraabeli} on tason niiden pisteiden joukko, joiden etäisyys kiinteästä pisteestä, \termi{polttopiste}{polttopisteestä} on sama kuin etäisyys kiinteästä suorasta, \termi{johtosuora}{johtosuorasta}.
}

%%%%%%MAA2, luku 3.1 Toisen asteen polynomifunktio
Kurssilla 2 mainittiin, että toisen asteen polynomifunktion kuvaaja on paraabeli. Nämä kuvaajat olivat muotoa $y=ax^2+bx+c$ (\termi{paraabelin normaalimuoto}{normaalimuoto}) olevia käyriä, joissa $a$ määräsi paraabelin aukeamissuunnan. Paraabeli aukeaa alaspäin, jos $a<0$, ja ylöspäin, jos $a>0$. Jos $a=0$, kyseessä ei ole paraabeli.

\begin{kuva}
    kuvaaja.pohja(-1.5, 3.5, -0.5, 2.5, korkeus = 4, nimiX = "$x$", nimiY = "$y$", ruudukko = True)
    kuvaaja.piirra("0.5*x**2-x+0.25", a = -1.5, b = 3.5, nimi = "$y= 0,5x^2-x+0,25$", kohta = (3.2,2.1), suunta = 135)
\end{kuva}

\begin{kuva}
    kuvaaja.pohja(-1.5, 3.5, -0.5, 2.5, korkeus = 4, nimiX = "$x$", nimiY = "$y$", ruudukko = True)
    kuvaaja.piirra("-0.5*x**2+x+1.75", a = -1.5, b = 3.5, nimi = "$y= -0,5x^2+x+1,75$", kohta = (3.2,-0.5), suunta = 135)
\end{kuva}

Yksityiskohtaisempi kuvaus toisen asteen polynomifunktion kuvaajan käyttäytymisestä vakioiden eri arvoilla on liitteenä kirjassa Vapaa matikka MAA2: Polynomifunktiot.

\begin{esimerkki}
Määritä pistejoukon yhtälö, jolla on seuraava ominaisuus: Jokainen pistejoukon piste on yhtä etäällä pisteestä $(0, 3)$ ja suorasta $y=-3$
\begin{esimratk}
Pisteen $P=(x, y)$ etäisyys annetusta pisteestä on
\[
\sqrt{(x-0)^2+(y-3)^2}=\sqrt{x^2+(y-3)^2}
\]
Pisteen $P$ etäisyys annetusta suorasta on pisteen ja suoran $y$-koordinaattien erotuksen itseisarvo
\[
|y-(-3)| = |y+3| 
\]
Merkitään nämä etäisyydet yhtäsuuriksi ja ratkaistaan saatu yhtälö $y$:n suhteen.
\begin{align*}
|y+3| & = \sqrt{x^2+(y-3)^2} &&\ppalkki \text{neliöönkorotus, kumpikin puoli $>0$}\\
(y+3)^2  &= x^2+(y-3)^2 \\
y^2+6y+9 &=  x^2+y^2-6y+9\\
12y &= x^2 &&\ppalkki : 12\\
y &= \frac{1}{12}x^2
\end{align*}

Pistejoukon yhtälö on $y=\frac{1}{12}x^2$.

\end{esimratk}
\end{esimerkki}


%%% FIX ME ONKO HUIPPUMUOTOINEN HYVÄ TERMI? %%%%%%%%%%%
\subsection{Paraabelin huippumuotoinen yhtälö}

Kaikki muotoa $y=ax^2+bx+c$ olevat paraabelit voidaan ilmoittaa paraabelin huipun $(x_0, y_0)$ avulla seuraavasti.

\[
y-y_0 = a(x-x_0)^2
\]

Näissä kummassakin muodossa vakio $a$ on sama.

Yksinkertaisin tapa todeta huippumuotoisen yhtälön todellakin määrittelevän normaalimuotoisen paraabelin on kirjoittaa huippumuotoinen yhtälö auki:

\begin{align*}
y-y_0 &= a(x-x_0)^2 \\
y-y_0 &= a(x^2 - 2x_0x + x_0^2) \\
y &= ax^2 - 2ax_0 x + ax_0^2 +y_0
\intertext{Merkitään $b = -2ax_0$ ja $c = ax_0^2 +y_0$:}
y &= ax^2 +bx +c
\end{align*}

%%%%% FIX ME Mikä on tämän alaluvun suhde MAA2-kirjan liitteenä olevaan alalukuun "Toisen asteen polynomin kuvaaja"

\begin{esimerkki}
	Mitkä ovat paraabelin $y=3x^2-12x+13$ huipun koordinaatit?
	\begin{esimratk} \textbf{Tapa 1}
		Paraabelin huipun koordinaatit näkisi suoraan huippumuotoiseksi muutetusta yhtälöstä. Helpompaa on kuitenkin muuttaa huippumuotoinen yhtälö $y$:n suhteen ratkaistuksi ja merkitä yhtälöiden kertoimet samoiksi, jolloin saadaan selville $x_0$ ja $y_0$.

		\begin{align*}
			y-y_0	&= a(x-x_0)^2 \\
			y       &= a(x^2-2x_0x+x_0^2)+y_0\\
			y       &= ax^2-2ax_0x+(ax_0^2+y_0) &&\ppalkki a=3\\
			y       &= 3x^2-6x_0x+(3x_0^2+y_0)
		\end{align*}

		Merkitään $x$:n kertoimet ja vakiotermit samoiksi.

		\begin{align*}
			&\begin{cases}
				-6x_0=-12 \\
				3x_0^2+y_0 =13
			\end{cases}\\
			&\begin{cases}
				x_0=2 \\
				y_0 =1
			\end{cases}
		\end{align*}
		\begin{esimvast}
			Paraabelin huippu on pisteessä $(2, 1)$.
		\end{esimvast}

		\begin{esimratk} \textbf{Tapa 2}
			Täydennetään $3x^2-12x+13$ neliöksi.
			\begin{align*}
				3x^2-12x+13 = 3(x^2-4x)+13 = 3(x^2-4x+4)+1 = 3(x-2)^2+1
			\end{align*}
			eli
			\[ y-1 = 3(x-2)^2 \]
			\begin{esimvast}
			Paraabelin huippu on pisteessä $(2, 1)$.
\end{esimvast}
\end{esimratk}

\end{esimratk}
\end{esimerkki}

%%%FIX ME: Tulisiko tämän olla tässä vai liitteissä 'todistuksia'-osiossa?
\subsection{Geometrisen määritelmän yhteys normaali- ja huippumuotoiseen yhtälöön}

Edellä paraabeli määriteltiin käyräksi, joka koostuu niistä pisteistä jotka ovat yhtä kaukana sekä polttopisteestä että johtosuorasta. Aikaisemmassa esimerkissä saatiin erään tällaisen käyrän yhtälöksi toisen asteen polynomi. Voidaan osoittaa tämän pätevän yleisesti:

Olkoon $(p,q)$ polttopiste, $(x_0, y_0)$ huippupiste, suora $y = r$ johtosuora ja $P=(x, y)$ paraabelin yleinen piste. Koska $P$ on yhtä kaukana sekä polttopisteestä että johtosuorasta, voidaan kirjoittaa kuten aikaisemmassa esimerkissä
\begin{align*}
|y-r| & = \sqrt{(x-p)^2 + (y-q)^2} &&\ppalkki \text{kumpikin puoli jälleen $>0$}\\
(y-r)^2 &= (x-p)^2 + (y-q)^2 \\
y^2 -2yr + r^2 &= x^2 -2xp + p^2 + y^2 -2yq + q^2 &&\ppalkki \text{ryhmitellään tekijät}\\
2(q-r)y &=  x^2 - 2px + p^2 +q^2 -r^2  &&\ppalkki q \not = r\\ 
y &= \frac{x^2 - 2px + p^2 +q^2 -r^2}{2(q-r)}  &&\ppalkki q^2 -r^2 = (q-r)(q+r) \\
y &= \frac{1}{2(q-r)}(x^2 - 2p x +p^2) + \frac{q+r}{2} &&\ppalkki \text{täydennetään neliöksi}\\
y &= \frac{1}{2(q-r)}(x - p)^2 + \frac{q+r}{2} \\
y - \frac{q+r}{2} &= \frac{1}{2(q-r)}(x - p)^2.
\end{align*}
Saatu muoto saattaa näyttää jo tutulta. 

Tarkastellaan paraabelin huipun $(x_0, y_0)$ koordinaatteja: lyhin mahdollinen jana polttopisteestä $(p,q)$ johtosuoralle $y =r$ kulkee huipun $(x_0, y_0)$ kautta ja on kohtisuorassa johtosuoraa vastaan (piirrä kuva). Näin ollen huipun ja polttopisteen $x$-koordinaatit ovat samat, $x_0 = p$. 

Geometrisen määritelmän mukaan paraabelin piste on yhtä kaukana polttopisteestä ja johtosuorasta: toisin sanoen huipun $y$-koordinaatti on pisteiden $(x_0, r)$ ja $(x_0, q)$ välisen janan puolivälissä, jolloin $y_0 = \frac{1}{2}(r+q)$. Sijoitetaan tämä ja $p = x_0$ aikaisempaan yhtälöön:

\begin{align*}
y - \frac{q+r}{2} &= \frac{1}{2(q-r)}(x - p)^2 &&\ppalkki p = x_0, \, y_0 = \frac{r+q}{2} \\
y -y_0 &= \frac{1}{2(q-r)}(x - x_0)^2 &&\ppalkki \text{Valitaan } a = \frac{1}{2(q-r)} \\
y - y_0 &= a(x - x_0)^2
\end{align*}

Tämä on paraabelin huippumuotoinen yhtälö.

Huippumuotoisen yhtälön kohdalla totesimme, että $b = -2ax_0$ ja $c = ax_0^2 +y_0$. Tämän voi nähdä myös aikaisemmasta välivaiheesta:
\begin{align*}
y &= \frac{1}{2(q-r)}x^2 - \frac{2p}{2(q-r)}x +\frac{p^2 +q^2 -r^2}{2(q-r)} \\
y &= \frac{1}{2(q-r)}x^2 - \frac{2p}{2(q-r)}x +\frac{p^2}{2(q-r)} + \frac{q+r}{2}
\intertext{Sijoitetaan $a = \dfrac{1}{2(q-r)}, p = x_0 $ ja $y_0 = (q+r)/2$ :}
y &= ax^2 - 2ax_0x + ax_0^2 + y_0 \\
y &= ax^2 + bx +c
\end{align*}

Entuudestaan tiedämme, että paraabeli $y = ax^2 + bx +c$ aukeaa ylöspäin kun $a >0$ ja alaspäin kun $a < 0$. Yhtälö \[a = \frac{1}{2(q-r)}\] antaa tälle geometrisen yhteyden polttopisteen ja johtosuoran keskinäiseen sijaintiin: 

$a > 0$ kun $q > r$, eli polttopisteen $y$-koordinaatti on suurempi kuin johtosuoran (eli polttopiste on johtosuoran yläpuolella). Vastaavasti $a < 0$ kun $q < r$ ja polttopiste sijaitsee johtosuoran alapuolella.

%%%Harjoitustehtäväksi?
Yllä teimme oletuksen, että $q \not = r$. Millainen käyrä syntyisi, jos pätisi $q = r$?

\begin{tehtavasivu}

\subsubsection*{Opi perusteet}

\begin{tehtava}
    Kuinka monta leikkauspistettä voi olla paraabelilla ja
    \alakohdat
        § suoralla,
        § ympyrällä,
        § toisella paraabelilla?
    \loppu
    \begin{vastaus}
        \alakohdat
            § 0--2
            § 0--4
            § 0--4
        \loppu
    \end{vastaus}
\end{tehtava}

\begin{tehtava}
Ratkaise paraabelien $y=5(x-2)^2$  ja $y=-x^2+2x+4$ leikkauspisteet?
\begin{vastaus}
%tulee toisen asteen yhtälö ratkaistavaksi
% http://www.wolframalpha.com/input/?i=y%3D5%28x-2%29%5E2%2C+y%3D-x%5E2%2B2x%2B4
$(1, 5)$ ja $(\frac{8}{3}, \frac{20}{9})$
\end{vastaus}
\end{tehtava}

\begin{tehtava}
%tehtävä helpottuu, koska vakiotermin saa suoraan
Määritä sen ylöspäin aukeavan paraabelin yhtälö, joka kulkee pisteiden $(-5, 6)$, $(0, -4)$ ja $(1, 0)$ kautta.
\begin{vastaus}
% http://www.wolframalpha.com/input/?i=y%3D+x%5E2%2B3x-4+at+x%3D%7B-5%2C+0%2C+1%7D
$y= x^2+3x-4$
\end{vastaus}
\end{tehtava}

\begin{tehtava}
Määritä sen alaspäin aukeavan paraabelin yhtälö, joka kulkee pisteiden $(-5, 6)$, $(0, -4)$ ja $(1, 0)$ kautta.
\begin{vastaus}
% http://www.wolframalpha.com/input/?i=y%3D+x%5E2%2B3x-4+at+x%3D%7B-5%2C+0%2C+1%7D
$y= x^2+3x-4$
\end{vastaus}
\end{tehtava}

\begin{tehtava}
Millaisella käyrällä ovat ympyröiden keskipisteet, kun ympyrät kulkevat pisteen $(0, 0)$ kautta ja sivuavat suoraa $x=4$?
\begin{vastaus}
%keskipiste (x, y)
% x< 4 
% kulkee origon kautta, joten (0-x)^2+(0-y)^2=r^2
% etäisyys suorasta |x-4| = r
$x=-\frac{1}{8}y^2+2$
\end{vastaus}
\end{tehtava}

\begin{tehtava}
Millä vakion $a$ arvolla suora $y=x$ on paraabelin $y=x^2-3x+a$ tangentti?
\begin{vastaus}
% yhtälöllä x = x^2-3x+a pitäisi olla tasan yksi ratkaisu 0 = (x-2)^2-4+a
% http://www.wolframalpha.com/input/?i=y%3D+x%5E2-3x%2B4%2C+y%3Dx
$a=4$
\end{vastaus}
\end{tehtava}

\subsubsection*{Hallitse kokonaisuus}

\subsubsection*{Sekalaisia tehtäviä}

TÄHÄN TEHTÄVIÄ SIJOITTAMISTA ODOTTAMAAN

\begin{tehtava}
Mikä on sen käyrän yhtälö, jonka kukin piste on yhtä etäällä suorasta $y=0$ ja pisteestä $(-3, 1)$.
\begin{vastaus}
% http://www.wolframalpha.com/input/?i=+sqrt%28%28-3-x%29%5E2+%2B+%281-y%29%5E2%29%3D+abs%280-y%29
$y = \frac{1}{2}x^2+3 x+5$
\end{vastaus}
\end{tehtava}

\begin{tehtava}
%%% ONKO JOHTOSUORA NIIN OLEELLINEN KÄSITE, ETTÄ KÄYTETÄÄN TEHTÄVISSÄ?
%% tehtävänhän voi kirjoittaa ilman tuota termiä
Mikä on sen paraabelin yhtälö, jonka polttopiste on $(2, 3)$ ja johtosuora $y=1$?
\begin{vastaus}
% http://www.wolframalpha.com/input/?i=+sqrt%28%282-x%29%5E2+%2B+%283-y%29%5E2%29%3D+abs%281-y%29
$y = \frac{1}{4}x^2-x+3$
\end{vastaus}
\end{tehtava}

\begin{tehtava}
Mitkä ovat suoran $x+y=6$ ja paraabelin $y=4x^2-3x$ leikkauspisteet?
\begin{vastaus}
% http://www.wolframalpha.com/input/?i=y%3D4x%5E2-3x%2C+x%2By%3D6
$x = -1$, $ y = 7$ tai $x = \frac{3}{2}$, $y = \frac{9}{2}$
\end{vastaus}
\end{tehtava}

%%%%%TÄMÄ JA SEURAAVA PITÄISI SIIRTÄÄ VASEMMALLE/OIKEALLE AUKEAVIEN PARAABELIEN JÄLKEEN?
\begin{tehtava}
Millaisella käyrällä ovat ympyröiden keskipisteet, kun ympyrät kulkevat pisteen $(0, 0)$ kautta ja sivuavat suoraa $x=4$?
\begin{vastaus}
%keskipiste (x, y)
% x< 4 
% kulkee origon kautta, joten (0-x)^2+(0-y)^2=r^2
% etäisyys suorasta |x-4| = r
$x=-\frac{1}{8}y^2+2$
\end{vastaus}
\end{tehtava}



\begin{tehtava}
Määritä ne paraabelin $y=x^2-1$ pisteet, jotka ovat yhtä kaukana pisteistä $(4, 4)$ ja $(4, 2)$?
\begin{vastaus}
%suoran y=3 ja paraabelin leikkauspisteet
$x=-2$, $y=3$ ja $x=2$, $y=3$
\end{vastaus}
\end{tehtava}

\begin{tehtava}
Määritä ne paraabelin $y=x^2-1$ pisteet, jotka ovat yhtä kaukana pisteistä $(4, 4)$ ja $(3, 3)$?
\begin{vastaus}
%pisteiden keskipisteen (3,5; 3,5) kautta kulkeva suora y=-x+7
%suoran  ja paraabelin leikkauspistee
% http://www.wolframalpha.com/input/?i=y%3Dx%5E2-1%2C+y%3D-x%2B7
$x = -\frac{1+\sqrt{33}}{2}$,   $y = \frac{15+\sqrt{33}}{2}$ tai $x = -\frac{\sqrt{33}-1}{2}$,   $y = \frac{15-\sqrt{33}}{2}$
\end{vastaus}
\end{tehtava}



\begin{tehtava}
% VAIKEA
Määritä kaksi sellaista paraabelia, että niillä on täsmälleen kolme yhteistä pistettä
\begin{vastaus}
%idea valitaan suora, joka on tangetti kummallekin paraabelille esim. y=x ja kumpikin paraabeli sivuaa suoraa samassa kohtaa
% http://www.wolframalpha.com/input/?i=y%3Dx%5E2-x%2C+x%3D2y%5E2-y
Esimerkiksi $y=x^2-x$ ja  $x=2y^2-y$
\end{vastaus}
\end{tehtava}

\begin{tehtava}
Millä parametrin $a$ arvoilla paraabelin $y=x^2-ax+a$ huippu on $x$-akselilla?
\begin{vastaus}
% esim. neliöksi täydentäminen y=(x-a/2)^2 -a^2/4 +a, josta -a^2/4 +a =0
$a=0$ ja $a=4$
\end{vastaus}
\end{tehtava}

\begin{tehtava}
Määritä vakio $a$ siten, että lausekkeen $2x^2+12x+a$ pienin arvo on 10.
\begin{vastaus}
% esim. neliöksi täydentäminen 2x^2+12x+a= 2((x+3)^2-9+a/2), josta 2(-9+a/2)=10
$a=28$
\end{vastaus}
\end{tehtava}

\begin{tehtava}
Paraabelin $y = x^2$ kaarevuutta origossa voidaan tutkia ympyröiden avulla. Ympyrän $x^2+(y-r)^2=r^2$ keskipiste on $y$-akselilla ja se kulkee origon kautta. Millä $r$:n arvoilla ympyrällä ja paraabelilla on
\alakohdat
§ yksi
§ kolme leikkauspistettä?
§ Mikä on suurin $r$, jolla leikkauspisteitä on tasan yksi? Tämän voidaan ajatella olevan paraabelin kaarevuussäde origossa.
\loppu
\begin{vastaus}
\alakohdat
§ $0 < r \geq \frac{1}{2}$
§ $\frac{1}{2} < r $
§ $r = \frac{1}{2}$
\loppu
\end{vastaus}
\end{tehtava}

\end{tehtavasivu}


	% merkitys toisen asteen polynomin kuvaajana
	% geometrisen määritelmän maininta
	\newpage \section{Paraabelin sovelluksia}

\laatikko{
KIRJOITA TÄHÄN LUKUUN

\begin{itemize}
\item TEHTY paraabelin huippu on kohdassa $x=-b/2a$, todistus
\item (Jo edellisessä kappaleessa? Vai pitäisikö siirtää tähän?) paraabelin yhtälön huippumuoto $y-y_0=a(x-x_0)^2$
\item TEHTY paraabelin yhtälön ratkaiseminen kolmen pisteen avulla
\item soveltavia tehtäviä, ne iänikuiset holvikaaret jne.
\end{itemize}
KIITOS!}

\subsection{Paraabelin huipun sijainti}
Joissain tehtävissä hyödylliseksi osoittautuu tulos, että paraabelin $y = ax^2 +b +c$ huipun $(x_0, y_0)$ $x$-koordinaatti on aina
\[x_0 = - \frac{b}{2a}\]
ja $y$-koordinaatti
\[y_0 = c - \frac{b^2}{4a}.\]

%%% FIXME MAA2:n liitteistä toteamus että x= \frac{-b}{2a} on neliön minimi tjsp tähän?

Huipun sijainnin kohdassa $x = - \frac{b}{2a}$ voi osoittaa tutkimalla paraabelin määräävän polynomifunktion ääriarvoja funktion derivaatan avulla; derivaattaan tutustutaan kurssilla MAA7. Saman tuloksen näkee myös paraabelin yhtälön huippumuodon $y - y_0 = a(x- x_0)^2$ ja normaalimuodon $y = ax^2 +bx +c$ yhtäpitävyydestä. Palautetaan edellisestä kappaleesta mieleen, että huippumuodon saattaa kirjoittaa auki

\begin{align*}
y-y_0 &= a(x-x_0)^2 \\
y-y_0 &= a(x^2 - 2x_0x + x_0^2) \\
y &= ax^2 - 2ax_0 x + ax_0^2 +y_0.
\end{align*}
Nyt
\[ax^2 - 2ax_0 x + ax_0^2 +y_0 = ax^2 +bx +c\]
kun $b = -2ax_0$ ja $c = ax_0^2 +y_0$, ja siis
\[x_0 = -\frac{b}{2a}\]
ja sijoittamalla tämä saadaan myös $y_0$
\begin{align*}
y_0 &= c - ax_0^2 = c -a\left(-\frac{b}{2a}\right)^2 \\
&= c - \frac{ab^2}{4a^2} \\
&= c - \frac{b^2}{4a}.
\end{align*}

\subsection{Paraabelin yhtälön ratkaiseminen kolmen pisteen avulla}

\begin{esimerkki}
	Paraabeli kulkee pisteiden $(0,1)$, $(2,0)$ ja $(4,1)$ kautta. Muodosta paraabelin yhtälö.
	
	\begin{esimratk} 
		Jokainen annettu piste toteuttaa saman paraabelin yhtälön $y - y_0 = k(x-x_0)^2$. Näin ollen kannattaa sijoittaa kunkin pisteen koordinaatit paraabelin yhtälöön ja muodostaa kolmen pisteen muodostamista yhtälöistä yhtälöryhmä:
		\[
		\left\{
		\begin{aligned}
		1 - y_0 = k(0 - x_0)^2 \\
		0 - y_0 = k(2 - x_0)^2 \\
		1 - y_0 = k(4 - x_0)^2
		\end{aligned}
		\right.
		\]

		Yhtälöryhmä ratkaistaan samalla lailla kuten aiemminkin:
		
		Esimerkiksi voidaan ratkaista $y_0$ ensimmäisestä yhtälöstä ja sijoittaa se kolmanteen.

\begin{align*}
	1 - y_0 &= k(0 - x_0)^2 &&\ppalkki \text{1. yhtälö}\\
	y_0 &= 1- kx^2_0	 \\
    \intertext{Sijoitetaan tämä kolmanteen:}
	1-y_0 &= k(4-x_0)^2 &&\ppalkki \text{Avataan binomin neliö} \\
	1-y_0 &= k(16 - 8x_0 + x_0^2) &&\ppalkki \text{ja sijoitetaan }y_0 \\
	1-1+kx^2_0 &= 16k - 8kx_0 + x^2_0 &&\ppalkki -kx^2_0 \\
	-8kx_0 + 16k &= 0 &&\ppalkki \cdot \, \frac{1}{8k}, \, k \neq 0 \\
	x_0 &= 2 \\
	y_0 &= 1- kx^2_0 = 1 - 4k	
\end{align*}	

		 Sijoitetaan saadut $x_0$:n ja $y_0$:n arvot keskimmäiseen yhtälöön.

\begin{align*}	
	0 - y_0 &= k(2 - x_0)^2 \\
	0 - (1 - 4k) &= k(2-2)^2 \\
	4k -1 &= 0 \\
	k &= \frac{1}{4} \\
	y_0 &= -k(2 - x_0)^2 = - \frac{1}{4} \cdot 0 = 0
\end{align*}

		Sijoitetaan lopulliset arvot paraabelin yhtälöön:

\begin{align*}	
    y - y_0 &= k(x-x_0)^2 \\
	y - 0 &= \frac{1}{4}(x-2)^2 \\
	y &= \frac{1}{4}(x^2-4x+4) \\
	y &= \frac{1}{4}x^2 - x + 1
\end{align*}


		\begin{esimvast}
			$y=\frac{1}{4}x^2 - x + 1$
			
			Aivan yhtä hyvin voitaisiin käyttää myös yhtälöä $ax^2 + bx +c$.
		\end{esimvast}


	\end{esimratk}
\end{esimerkki}




\begin{tehtavasivu}

\subsubsection*{Opi perusteet}

\begin{tehtava}
Arkkitehti Guggenheim suunnittelee uuteen taidemuseoon kahta paraabelin muotoista holvikaarta. Holvikaarten leveys on 5 metriä ja korkeus 7 metriä, ne ovat puolen metrin päässä toisistaan ja ne ovat sijoitettu symmetrisesti (y-akseliin nähden) julkisivulle. Määritä holvikaarten yhtälöt.
\begin{vastaus}
%holvikaaret puoli metriä toisistaan, siis etäisyys y-akselista 0,25m.
%Tässä ensimmäinen piste, toinen piste on 5,25m päässä, ja kolmas on (2,75; 7). %Riittää määrittää vain yksi paraabeli, ja toisen saa x_0:n vastaluvusta.
$y-7 = -\frac{6,25}{7}(x - 2,75)^2$ ja $y-7 = -\frac{6,25}{7}(x + 2,75)^2$
\end{vastaus}
\end{tehtava}



\subsubsection*{Hallitse kokonaisuus}

\subsubsection*{Sekalaisia tehtäviä}

TÄHÄN TEHTÄVIÄ SIJOITTAMISTA ODOTTAMAAN

%%% Helppoa intuitiota funktion ääriarvopisteiden etsimiseen
\begin{tehtava}
Usein huomataan, että halutaan tietää missä pisteessä jokin funktio saa pienimmän tai suurimman arvonsa.

Onko seuraavilla funktioilla suurinta tai pienintä arvoa? Jos on, mikä se on, ja millä $x$:n arvolla se saavutetaan?
\begin{alakohdat}
    \alakohta{$f(x) = 4x^2 - 8x + 8$}
    \alakohta{$f(x) = -x^2 - 3x - 1$}
    \alakohta{$f(x) = (x-2)(x-3)$}
\end{alakohdat}

Kurssilla MAA??? (derivaattakurssi?) tutustutaan tarkemmin funktion ääriarvojen (pieninten ja suurinten arvojen) etsimiseen, erityisesti myös silloin kun funktio ei ole toisen asteen polynomi.
	\begin{vastaus}
	\begin{alakohdat}
	    \alakohta{Pienin arvo $4$ pisteessä $x = 1$ (funktion kuvaaja on ylöspäin aukeava paraabeli).}
	    \alakohta{Suurin arvo $\frac{5}{4}$ pisteessä $\frac{-3}{2}$.}
	    \alakohta{Pienin arvo $\frac{-1}{4}$ pisteessä $\frac{5}{2}$.}
	\end{alakohdat}
	\end{vastaus}
\end{tehtava}

\end{tehtavasivu}
	% huipun x-koordinaatti on -b/2a
		% todistus liitteeksi -Ville
		% todistus tähän -Niko
	% paraabelin tangentit
	\newpage \section{Vasemmalle ja oikealle aukeavat paraabelit}

\laatikko{
KIRJOITA TÄHÄN LUKUUN

\luettelo{
§ muotoa $x=ay^2+by+c$ olevat paraabelit aukeavat oikealle tai vasemmalle
}

KIITOS!}

%Määritelmä

%FIXME Sopisiko tähän kuva jossa taval. ja vaakasuunt. paraabeleiden symmetria-akselit?

Edellisissä kappaleissa käsiteltiin muotoa $y = ax^2 + b + c$ olevia paraabeleita, jotka ovat symmetrisiä pystyakselin suhteen (eli akselin oikea puoli on vasemman puolen peilikuva). Vaihtamalla paraabelin yhtälössä $x$- ja $y$-muuttujat keskenään saadaan yhtälö
\[x=ay^2+by+c.\]
Sen kuvaaja on joko oikealle tai vasemmalle aukeava paraabeli, jonka symmetria-akseli on vaakasuora. Kun $a>0$ paraabeli aukeaa oikealle, ja kun $a < 0$ paraabeli aukeaa vasemmalle.

%FIXME Mikä on vakiintunut suomenkielinen terminologia? 
%FIXME Tämän laatikko asettelee itsensä(?) rumasti seuraavalle sivulle, jättäen edel. sivulle ison tyhjän tilan
%Tässä ja alla käytetty samaa "oikealle tai vasemmalla aukeava (...)" -rakennetta kuin otsikossa
\laatikko{Yhtälö \[x=ay^2+by+c\] määrittelee paraabelin, joka aukeaa oikealle tai vasemmalle.}

%FIXME mainitse kuvien paraabelit tekstissä?
%Kuva: oikealle aukeava paraabeli
\begin{kuva}
    kuvaaja.pohja(-3, 4, -2, 4)
    kuvaaja.piirraParametri("0.5*t**2 -t - 1", "t", a=-2, b=4, nimi = r"$x = \frac{1}{2}y^2  - y -1$", kohta = (0.5, 1.5))
\end{kuva}

%Kuva: vasemmalle aukeava paraabeli

\begin{kuva}
    kuvaaja.pohja(-4, 3, -4, 2)
    kuvaaja.piirraParametri("-t**2 -2*t + 1", "t", a=-4, b=2, nimi = r"$x = -y^2 - 2y +1$", kohta = (0.3, 1.2))
\end{kuva}

%FIXME Perusesimerkki tähän

%Käännettyjen(???) paraabeleiden ominaisuuksia

Vasemmalle ja oikealle aukeaville paraabeleille pätevät vastaavat tulokset kuin aikaisemmin käsitellyille pystysuorille paraabeleille. Esimerkiksi vaakasuuntaisen paraabelin huippu sijaitsee aina pisteessä

\[y = \frac{-b}{2a}.\]

%Huippumuoto käännetylle paraabelille

Myös paraabelille, jonka yhtälö on $x=ay^2+by+c$ voidaan kirjoittaa huippumuotoinen yhtälö:

\laatikko{Oikealle tai vasemmalle aukeavan paraabelin huippumuotoinen yhtälö on
\[
x-x_0 = a(y-y_0)^2.
\]}

%FIXME Tähän tehtäväesimerkki ylläolevasta
\begin{esimerkki}
    Määritä sellaisen vasemmalle tai oikealle aukeavan paraabelin yhtälö, jonka huippu on samassa pisteessä kuin paraabelin $y = x^2 + 4x +1$ huippu ja myös leikkaa $y$-akselin samassa pisteessä kuin kyseinen paraabeli.
    \begin{esimratk} % ratkaisu
        Etsitään ensiksi paraabelin $y = x^2 + 4x +1$ huippu. Voisimme täydentää paraabelin neliöksi, mutta helpompaa on käyttää aikaisemmin johtamaamme tulosta (pystysuuntaisen) paraabelin huipun $x$-koordinaatille
        \begin{align*}
        x &= \frac{-b}{2a} \\
          &= \frac{-4}{2\,\cdot\,1} = -2, \\
        \intertext{minkä jälkeen saamme huipun $y$-koordinaatin sijoittamalla}
        y &= x^2 + 4x +1 = (-2)^2 - 8 +1 = -3.
        \end{align*}
        Huipun koordinaatit ovat siis $(-2, -3)$. Tämä on myös kysytyn paraabelin huippu, joten sijoitetaan se vasemmalle tai oikealle aukeavan paraabelin huippumuotoiseen yhtälöön.
        \begin{align*}
        x-x_0 &= a(y-y_0)^2\\
        x-(-2) &= a(y-(-3))^2 \\
        x +2 &= a(y^2 +6y +9)\\
        x &= ay^2 +6ay +9a -2
        \end{align*}
        Vielä on selvitettävä kerroin $a$. Tehtävänannon mukaan kysytty paraabeli leikkaa $y$-akselin samassa pisteessä kuin $y = x^2 + 4x +1$, selvitetään siis tuo piste. Tutkimalla koordinaatistoa huomataan, että $y$-akseli on itse asiassa suora $x = 0$, mikä on siten leikkauspisteen $x$-koordinaatti. Sijoittamalla tämän paraabelin yhtälöön saamme $y$-koordinaatin:
        \begin{align*}
        y &= x^2 + 4x +1\\
        y &= 0 + 0 + 1 = 1
        \end{align*}
        Näin ollen molemmat paraabelit kulkevat pisteen $(0,1)$ kautta. Sijoittamalla tämä kysytyn paraabelin yhtälöön voimme ratkaista tuntemattoman vakion $a$ arvon.
        \begin{align*}
        x &= ay^2 +6ay +9a -2 && \ppalkki (x,y) = (0,1) \\
        0 &= a +6a +9a -2 \\
        2 = 16a \\
        a = \frac{1}{8}\\
        \intertext{Tästä saamme vaakasuuntaan (oikealle) aukeavan paraabelin yhtälöksi:}
        x &= ay^2 +6ay +9a -2 \\
        x &= \frac{1}{8}y^2 + \frac{6}{8}y + \frac{9}{8} -2  && \ppalkki \frac{9}{8} -2 = \frac{9 - 16}{8}\\
        x &= \frac{1}{8}y^2 + \frac{3}{4}y - \frac{7}{8}
        \end{align*}
    \end{esimratk}
    \begin{esimvast} % vastaus
        Paraabelin yhtälö on $x = \frac{1}{8}y^2 + \frac{3}{4}y - \frac{7}{8}$.
        %%%FIXME: KUVA!
        %%https://www.wolframalpha.com/input/?i=y+%3D+x%5E2++%2B+4x++%2B1%2C+x+%3D+1%2F8%28y%5E2+%2B+6y+-7%29
    \end{esimvast}
\end{esimerkki}

%Muita esimerkkejä?

%FIXME Paraabelin yleinen muoto (johtosuora ax +by +c = 0) (kartioleikkaukset??) liitteisiin?

%Tehtävät:

\begin{tehtavasivu}

\subsubsection*{Opi perusteet}

\begin{tehtava}
    Mihin suuntaan aukeaa paraabeli, jonka yhtälö on
    \alakohdat{
        § $x = -2y^2 + y + 1$
        § $x = y^2 - 3y -2$
        § $x = 4y^2 - 1y +3$
    }
    \begin{vastaus}
          \alakohdat{
          § Vasemmalle.
          § Oikealle.
          § Oikealle.
          }
    \end{vastaus}
\end{tehtava}

\begin{tehtava}
    Hahmottele edellisen tehtävän paraabelit ja laske niiden huippujen koordinaatit.
    \begin{vastaus}
          Huiput sijaitsevat pisteissä $\left(\frac{9}{8}, \frac{1}{4}\right)$, $\left(\frac{17}{4},\frac{3}{2}\right)$ ja $\left(\frac{47}{16},\frac{1}{8}\right).$
    \end{vastaus}
\end{tehtava}

\subsubsection*{Hallitse kokonaisuus}

\subsubsection*{Sekalaisia tehtäviä}

TÄHÄN TEHTÄVIÄ SIJOITTAMISTA ODOTTAMAAN


\end{tehtavasivu}
	% paraabeli x = ay^2  +by + c
	\newpage \section{Sekalaista}

\laatikko{
KIRJOITA TÄHÄN LUKUUN

\luettelo
§ kaikkea jännää kurssiin liittyen !
\loppu

KIITOS!}

\begin{tehtavasivu}

\subsubsection*{Opi perusteet}

\subsubsection*{Hallitse kokonaisuus}

\subsubsection*{Sekalaisia tehtäviä}

TÄHÄN TEHTÄVIÄ SIJOITTAMISTA ODOTTAMAAN

\end{tehtavasivu}
	% esimerkkejä ja tehtäviä (erikoisia, vaikeita, yms.)

\chapter{Ulkoasukokeiluja}
	\newpage \input{content/ulkoasukokeiluja}

\Closesolutionfile{ans}

\liitetyyli

\section{Vastaukset} \input{content/LIITE_vastaukset}

\newpage \input{content/LIITE_harjoituskokeita}
\newpage \section{Tehtäviä ylioppilaskokeista}

%%Pitäisikö tämän olla TEHT_ylioppilaskokeet kuten MAA1 ja MAA2? Nyt esim vastaukset ei toimi.
%%Vastaukset ei mene kohilleen ja nämä on vastausten jälkeen
%%Eikä vastauksia ole

\subsubsection*{Lyhyen oppimäärän tehtäviä}


\begin{tehtava} (K2014/2 a ja b)
\alakohdat{
		§ Missä pisteessä suora $x-5y=4$ leikkaa $y$-akselin?
		§ Ratkaise yhtälö $4x^3=48$. Anna tarkka arvo ja kolmidesimaalinen likiarvo.
	}
\end{tehtava}
%Lisännyt Aleksi Sipola 17.5.2014

\begin{tehtava}  (S2013/2 a ja b)
\alakohdat{
		§ Missä pisteissä suora $y=-3x+12$ leikkaa koordinaattiakselit?
		§ Ratkaise yhtälöpari 
		\[
\left\{
\begin{aligned}
 2x+y=4  \\
 -x+2y=1  
\end{aligned}
\right. 
\]
	}
\end{tehtava}
%Lisännyt Aleksi Sipola 17.5.2014

\begin{tehtava} (S2013/5) TARVITSEE KUVAN
Oheinen kuvaaja esittää paraabelia $y=ax^2+bx+c$. Määritä vakiot $a$, $b$ ja $c$ käyttämällä kuvioon ympyröillä merkittyjä pisteitä.
\end{tehtava}
%Lisännyt Aleksi Sipola 17.5.2014

\begin{tehtava} (K2013/2)
\alakohdat{
		§ Millä muuttujan $x$ arvoilla $4x+17$ on suurempi kuin $2-x$?
		§ Ratkaise yhtälö $x^2+14=-49$.
		§ Suora kulkee origon ja pisteen $(2,3)$ kautta. Kulkeeko se myös pisteen $(48,75)$ kautta?
	}
\end{tehtava}
%Lisännyt Aleksi Sipola 17.5.2014

\begin{tehtava} (K2012/4)
\alakohdat{
		§ Funktion $f(x)=\frac{3}{2}x+b$ nollakohta $2$. Määritä vakion $b$ arvo
		§ Ratkaise yhtälö $x^2+14=-49$.
		§ Suora kulkee origon ja pisteen $(2,3)$ kautta. Kulkeeko se myös pisteen $(48,75)$ kautta?
	}
\end{tehtava}
%Lisännyt Aleksi Sipola 17.5.2014


\begin{tehtava} (S2012/1)
\alakohdat{
		§ Ratkaise yhtälö $x^2-2x=0$.
		§ Ratkaise yhtälö $\frac{2}{3}x-1=\frac{2}{3}$.
		§ Ratkaise yhtälöpari
		\[
\left\{
\begin{aligned}
 x+2y=-4  \\
 2x-y=-3  
\end{aligned}
\right. 
\]
	}
\end{tehtava}
%Lisännyt Aleksi Sipola 17.5.2014

\begin{tehtava} (S2012/11) 
Aikuisen ihmisen sääriluun pituus $y$ riippuu henkilön pituudesta $x$ kaavojen 
\[
\begin{aligned}
 y=0,43-27 \text{(nainen)} \\
 y=0,45x-31  \text{(mies)}
\end{aligned}
\]
mukaisesti,kun yksikkönä on senttimetri.

  \alakohdat{
		§ Arkeologi löytää naisen sääriluun, joka on 41 cm pitkä. Kuinka pitkä nainen oli?
		§ Kaivauksissa löytyneen miehen pituudeksi arvioidaan 175 cm. Miehen läheltä löytyy sääriluu, jonka pituus on 42 cm. Onko  kyseessä saman henkilön sääriluu?		
  }
 

\end{tehtava}
%Lisännyt Aleksi Sipola 17.5.2014



\begin{tehtava} (K2011/1)
\alakohdat{
		§ Ratkaise yhtälö $4x+(5x-4)=12+3x$.
		§ Sievennä lauseke $x^2+x-(x^2-x)$ ja laske sen arvo, kun $x=\frac{1}{2}$
		§ Ratkaise yhtälöpari
		\[
\left\{
\begin{aligned}
 x-2y=0  \\
 x-3y=1  
\end{aligned}
\right. 
\]
	}
\end{tehtava}
%Lisännyt Aleksi Sipola 17.5.2014


\begin{tehtava}  (S2011/4)
Ludwig van Beethoven, Wolfgang Amadeus Mozart ja Johann Sebastian Bach elivät yhteensä 156 vuotta. Bach eli yhdeksän vuotta vanhemmaksi kuin Beethoven, Mozart kuoli 21 vuotta nuorempana kuin Beethoven. Kuinka vanhoiksi säveltäjät elivät?
\end{tehtava}
%Lisännyt Aleksi Sipola 17.5.2014

\begin{tehtava}  (S2010/1b)
Ratkaise yhtälö $(x-2)^2-4(2-x)=0$
\end{tehtava}
%Lisännyt Aleksi Sipola 17.5.2014

\begin{tehtava}  (S2010/3)
Oheisessa kuviossa on kaksi suoraa. Määritä näiden yhtälöt, ja laske niiden leikkauspisteen koordinaatit. Mikä on suorien ja $y$-akselin raajaman kolmion pinta-ala?
\end{tehtava}
% \begin{kuva}
%     kuvaaja.pohja(-2, 4, -2, 4, korkeus = 6, nimiX = "$x$", nimiY = "$y$", ruudukko = True)
%     kuvaaja.piirra("-1.5*x+3")
%     kuvaaja.piirra("1", kohta = 1, suunta = 0)
% \end{kuva}
%Lisännyt Aleksi Sipola 17.5.2014


\begin{tehtava}  (S2010/13)
Millä vakion $a$ arvolla yhtälöparilla 
\[
\left\{
\begin{aligned}
 2x+(a+1)y=5  \\
 3x+(a-2)y=a  
\end{aligned}
\right. 
\]
ei ole ratkaisua?
\end{tehtava}
%Lisännyt Aleksi Sipola 17.5.2014

\begin{tehtava}  (K2010/7)
Suorakulmaisen kolmion kateettien pituudet ovat $3,2 cm$ ja $5,7 cm$. Laske hypotenuusan pituus ja suoran kulman kärjen etäisyys hypotenuusasta.
\end{tehtava}
%Lisännyt Aleksi Sipola 17.5.2014

\begin{tehtava}  (K2010/10)
Määritä sen suoran yhtälö, joka kulkee pisteiden $A=(-1,1)$ ja $B=(8,4)$ yhdysjanan keskipisteen kautta ja on kohtisuorassa tätä janaa vastaan. Missä pisteissä suora leikkaa koordinaattiakselit? Piirrä kuvio.
\end{tehtava}
%Lisännyt Aleksi Sipola 17.5.2014

\begin{tehtava}  (K2010/13)
Määritä sen suoran yhtälö, joka kulkee pisteiden $A=(-1,1)$ ja $B=(8,4)$ yhdysjanan keskipisteen kautta ja on kohtisuorassa tätä janaa vastaan. Missä pisteissä suora leikkaa koordinaattiakselit? Piirrä kuvio.
\end{tehtava}
%Lisännyt Aleksi Sipola 17.5.2014


\begin{tehtava}  (S2009/12)
Suorat $x+y=8, x+3y=18$ ja $y-3=0$ rajoittavat kolmion. Piirrä kuvio ja laske kolmion pinta-ala. Muodosta epäyhtälöryhmä, jonka ratkaisuna on piirtämäsi kolmio sivut mukaan lukien.
\end{tehtava}
%Lisännyt Aleksi Sipola 17.5.2014

\begin{tehtava}  (S2009/3)
\alakohdat{
		§ Suoran kulmakerroin on $-\frac{1}{3}$, ja suora kulkee pisteen $(-1,2)$ kautta. Esitä suoran yhtälö muodossa $y=kx+b$
		§ Tutki, millä muuttujan $x$ arvoilla polynomi $2x^2+5x-3$ saa negatiivisia arvoja.
	}
\end{tehtava}
%Lisännyt Aleksi Sipola 17.5.2014


\begin{tehtava}  (S2008/8)
Millä vakion $a$ arvoilla suorat $y=-3x+2$ ja $y=ax+6$ erottavat $x$-akselista janan, jonka pituus on 3?
\end{tehtava}
%Lisännyt Aleksi Sipola 17.5.2014


\begin{tehtava}  (S1959/4)
Määrää $a$ siten, että paraabelin $y=ax^2$ suorasta $y=x+1$ erottama jänne on 8 pituudenyksikköä. Piirrä kuvio.
\end{tehtava}
%Lisännyt Aleksi Sipola 17.5.2014

\begin{tehtava}(S1959/9)
Kolme ympyrää, joiden säteet ovat $1cm$, $2cm$ ja $3cm$, sivuaa toisiaan ulkopuolisesti. Laske niiden kaarien rajoittaman (pienimmän) ``kaarikolmion'' piiri. 
\end{tehtava}
%Lisännyt Aleksi Sipola 17.5.2014

\begin{tehtava}(S1959/10)
Millä $k$:n reaaliarvoilla yhtälön $kx^2+2kx+3x+2k+1=0$ juuret ovat samanmerkkisiä reaalilukuja? 
\end{tehtava}
%Lisännyt Aleksi Sipola 17.5.2014

\subsubsection*{Pitkän oppimäärän tehtäviä}

\begin{tehtava}(S07/1b)
	Muodosta sen suoran yhtälö, joka kulkee pisteiden $(4, -3)$ ja $(-2,6)$ kautta. 
\end{tehtava}

\begin{tehtava}(K06/1b)
	Missä pisteessä suora $y=3x-4$ leikkaa x-akselia? 
\end{tehtava}

\begin{tehtava} (S07/5)
	Määritä ympyrän $x^2+y^2+4x-2y+1=0$ niiden tangettien yhtälöt, jotka kulkevat pisteen $(1,3)$ kautta.
\end{tehtava}



\begin{tehtava}(K2014/5)
Ympyrä sivuaa suoraa $3x-4y=0$ pisteessä $(8,6)$. Lisäksi se sivuaa positiivista $x$-akselia.
Määritä ympyrän keskipiste ja säde. 
\end{tehtava}
%Lisännyt Aleksi Sipola 17.5.2014
% VASTAUKSET EIVÄT MENE VASTAUS OSIOON VAAN JÄÄVÄT TEHTÄVÄNANNON PERÄÄN 
 % \begin{vastaus} VASTAUKSET EIVÄT MENE VASTAUS OSIOON VAAN JÄÄVÄT TEHTÄVÄNANNON PERÄÄN 
%	Ympyrän keskipiste on $(10,frac[10][3])$ ja $r=30$
 %   \end{vastaus}


\begin{tehtava}(K2014/9)
Taso $x+2y+3z=6$ leikkaa positiiviset koordinaattiakselit pisteissä $A$, $B$ ja $C$.
\alakohdat{
		§ Määritä sen tetraedrin tilavuus, jonka kärjet ovat origossa $O$ sekä pisteissä $A$,$B$ ja $C$. 
		§ Määritä kolmion $ABC$ pinta-ala
	}
% 	VASTAUKSET EIVÄT MENE VASTAUS OSIOON VAAN JÄÄVÄT TEHTÄVÄNANNON PERÄÄN 
%  \begin{vastaus} 
%	 Tilavuus on $6$
%  \end{vastaus}
\end{tehtava}
%Lisännyt Aleksi Sipola 17.5.2014

\begin{tehtava} (S2013/1)

\alakohdat{
		§ Ratkaise yhtälöä $x^2+6x=2x^2+9$.
		§ Ratkaise yhtälö $\frac{1+x}{1-x}=\frac{1-x^2}{1+x^2}$
		§ Esitä polynomi $x^2-9x+14$
	}
\end{tehtava}
%Lisännyt Aleksi Sipola 17.5.2014

\begin{tehtava}(S2013/10)
Pöydällä on kolme samankokoista palloa, joista kukin koskettaa kahta muuta. Niiden päälle asetetaan neljäs samanlainen  pallo, joka koskettaa kaikkia kolmea alkuperäistä palloa.
Mikä on rakennelman korkeus? Anna vastauksena tarkka arvo pallojen säteen avulla lausuttuna. 
\end{tehtava}
%Lisännyt Aleksi Sipola 17.5.2014

\begin{tehtava} (S2013/*14)
Tarkastellaan tasokäyrää, jonka yhtälö on $2x^2+2y^2-3xy-2x+2y-4=0$.
\alakohdat{
		§ Määritä käyrän ja koordinaattiakselien leikkauspisteet. (2p.)
		§ Osoita, että kaikki leikkauspisteet ovat saman ympyrän kehällä, ja määritä tämän ympyrän yhtälö. (3p.)
		§ Suora kulkee origon ja b-kohdan ympyrän keskipisteen kautta. Missä pisteissä tämä suora leikkaa alkuperäisen käyrän? (2p.)
		§ Onko alkuperäinen käyrä ympyrä? (2p.)
	}
\end{tehtava}
%Lisännyt Aleksi Sipola 17.5.2014

\begin{tehtava} (K2013/1)
\alakohdat{
		§ Ratkaise yhtälöä $(x-4)^2=(x-4)(x+4)$.
		§ Ratkaise epäyhtälöä $\frac{3}{5}x-\frac{7}{10} < -\frac{2}{15}x$.
		§ Suora kulkee pisteiden $(1,7)$ ja $2,4$ kautta. Missä pisteessä se leikkaa $x$-akselin
	}
\end{tehtava}
%Lisännyt Aleksi Sipola 17.5.2014


\begin{tehtava}(K2013/10) KUVA TARVITAAN
Oheisen kuution särmän pituus on 2. Sen sisällä on vaaleanpunainen pallo,joka sivuaa jokaista kuution tahkoa. Kuution yhdessä kulmassa on pienempi sininen pallo, joka sivuaa suurta palloaja kolmea kuution tahkoa kuvion mukaisesti. Laske sinisen pallon säteen tarkka arvo. 
\end{tehtava}
%Lisännyt Aleksi Sipola 17.5.2014


\begin{tehtava}(K2013/*15a) KUVA TARVITAAN
Ympyrä, jonka säde on $r>\frac{1}{2}$, asetetaan paraabelin $y=x^2$ sisäpuolelle alla olevan kuvan mukaisesti. Näytä, että ympyrän keskipisteen $y$‐koordinaatti on $r^2+\frac{1}{4}. $ (3
p.)
\end{tehtava}
%Lisännyt Aleksi Sipola 17.5.2014

\begin{tehtava} (S2012/1) Ratkaise yhtälöt
  \alakohdat{
		§ $2(1-3x+3x^2)=3(1+2x+2x^2)$
		§ $|x|=1+x$
		§ $1-x=\frac{1}{1-x}$	
  }
\end{tehtava}
%Lisännyt Aleksi Sipola 17.5.2014

\begin{tehtava}(S2012/*15)
Suora ympyrälieriö on pallon sisällä niin, että sen molempien pohjien reunat sivuavat pallon pintaa. Pallon pinta‐alan suhdetta lieriön koko pinta‐alaan merkitäänsymbolilla. $t$ Lieriön kokopinta‐alalla tarkoitetaan sen vaipan ja pohjien yhteenlaskettuja pinta‐aloja. 
\alakohdat{
		§ Määritä lieriön korkeuden suhde lieriön pohjan säteeseen parametrin $t$ avulla lausuttuna. (2p.) \\
		\\
		Millä parametrin $t$ arvoilla
		§ tällaista lieriötä ei ole olemassa (2p.)
		§ on täsmälleen yksi tällainen lieriö (3p.)	
		§ on kaksi tällaista lieriötä? (2p.)	
  }
\end{tehtava}

\begin{tehtava} (K2012/*15) Ratkaise   KUVA TARVITAAN
\alakohdat{
		§ Kaksi ympyrää sivuaa toisiaan ja $x$-akselia kuvan 1 mukaisesti. Määritä ympyröiden keskipisteiden vaakasuora etäisyys $d$ niiden säteiden avulla lausuttuna. (3 p.)
		§ Kolme ympyrää sivuaa toisiaan ja $x$-akselia kuvan 2 mukaisesti. Määritä keskimmäisen ympyrän säde $r_3$ kahden reunimmaisen ympyrän säteiden avulla lausuttuna. (3 p.)
		§ Todista René Descartesin (1596-1650) keksimä b-kohdan ympyröihin liittyvä kaava
		
		$(k_1+k_2+k_3)^2=2(k_1^2+k_2^2+k_3^2)$, \\
		jossa $k_i=\frac{1}{r_i}$,  $i=1,2,3.$ (3 p.)
	}
\end{tehtava}
%Lisännyt Aleksi Sipola 17.5.2014


\begin{tehtava} (S2011/1b) Ratkaise
Laske suoran $y=2x$ ja ympyrän $x^2+y^2=1$ leikkauspisteet.
\end{tehtava}
%Lisännyt Aleksi Sipola 17.5.2014

\begin{tehtava} (K2011/1) Ratkaise
  \alakohdat{
		§ $\frac{2}{x}=\frac{3}{x-2}$
		§ $x^2-2\leq x$
		§ $\left|\frac{3}{2}x-6\right|=6$	
  }
\end{tehtava}
%Lisännyt Aleksi Sipola 17.5.2014

\begin{tehtava}(K2010/4)
Puolipallon sisällä on kuutio siten, että sen yksi sivutahko on puolipallon pohjatasolla ja vastakkaisen sivutahkon kärkipisteet ovat pallopinnalla. Kuinka monta prosenttia kuution tilavuus on puolipallon tilavuudesta?
\end{tehtava}

\begin{tehtava}(K2010/8)
Tietunnelin poikkileikkaus on osa alaspäin aukeavaa paraabelia. Tien leveys on $10 m$, ja tunnelin poikkileikkauksen pinta-ala on $25 m^2$. Määritä tunnelin korkeus senttimetrin tarkkuudella.
\end{tehtava}


\begin{tehtava}(S2009/9)
Mikä paraabelin $y=5-x^2$ piste on lähinnä origoa? Piirrä kuvio.
\end{tehtava}

\begin{tehtava}(K2009/1 a ja b)
  \alakohdat{
		§ Ratkaise epäyhtälö $(x-3)^2=(x-1)(x+1)$
		§ Määritä suorien $\frac{x}{3}+\frac{y}{2}=1$ ja $3x-2y+3=0$ leikkauspiste	
  }
\end{tehtava}

\begin{tehtava} (K2008/2 a ja c)
  \alakohdat{
		§ Määritä suorien $2x+y=8$ ja $3x+2y=5$ leikkauspiste
		§ Ratkaise yhtälö $|3x-2|=5$	
  }
\end{tehtava}
%Lisännyt Aleksi Sipola 17.5.2014


\begin{tehtava}(K2008/7a)
Laske paraabelien $y=x^2-3$ ja $y=-x^2+2x+1$ leikkauspisteiden koordinaatit.
\end{tehtava}
%Lisännyt Aleksi Sipola 17.5.2014

\begin{tehtava}(K1983/7)
Ympyrä ja piste sen ulkopuolella ovat tunnetut. Hae pistettä ja ympyrän keskipistettä yhdistävällä suoralla viivalla semmoinen piste jonka etäisyys tunnetusta pisteestä on yhtäsuuri kuin ne tangentit, jotka haettavasta pisteestä saatetaan piirtää ympyrälle. 
\end{tehtava}
%Lisännyt Aleksi Sipola 17.5.2014


\begin{tehtava}(K1941/3)
Määrää $a$ siten, että yhtälön $x^2+(a+2)x-a^2=0$ suuremman ja pienemmän juuren eroitus saa mahdollisimman pienen arvon. Mitkä ovat juuret? 
\end{tehtava}
%Lisännyt Aleksi Sipola 17.5.2014


\begin{tehtava}(S1893/3)
Ratkaise ekvationit:
\[
\left\{
\begin{aligned}
 x+\frac{1}{2}(y+z)=102    \\
 y+\frac{1}{2}(x+z)=78  \\
 z+\frac{1}{2}(x+y)=61
\end{aligned}
\right. 
\]
\end{tehtava}
%Lisännyt Aleksi Sipola 17.5.2014



\newpage \section{Yleinen toisen asteen tasokäyrä}

\laatikko{
KIRJOITA TÄHÄN LUKUUN

\luettelo{
§ Tässä luvussa tarkastellaan lyhyesti yleisesti toisen asteen tasokäyriä, joista ympyrä ja paraabeli on jo käsitelty edellä
§ (kahden muuttujan) toisen asteen yhtälön määritelmä
§ joku tuttu esimerkki, vaikka paraabeli
§ esimerkkeinä yksi piste, kaksi suoraa, tyhjä
§ maininta siitä, että voi tulla myös ellipsi tai hyperbeli,
joista sitten joskus kirjoitetaan liite
}

KIITOS!}

\begin{tehtavasivu}

\subsubsection*{Opi perusteet}

\subsubsection*{Hallitse kokonaisuus}

\subsubsection*{Sekalaisia tehtäviä}

TÄHÄN TEHTÄVIÄ SIJOITTAMISTA ODOTTAMAAN

\end{tehtavasivu}
	% esim. piste, kaksi suoraa, tyhjä
\newpage \input{content/LIITE_ellipsi}
\newpage \input{content/LIITE_hyperbeli}
\newpage \input{content/LIITE_kolmioepayhtalo}
\newpage \section{Todistuksia}

%%% FIXME onko tämä oikea paikka tälle?
\subsection{Pisteen etäisyys suorasta yhdenmuotoisilla kolmioilla}
	% kaava pisteen etäisyydelle suorasta
	% kaikkien paraabelien yhdenmuotoisuus
