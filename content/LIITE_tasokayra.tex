\section{Yleinen toisen asteen tasokäyrä}

\laatikko{
KIRJOITA TÄHÄN LUKUUN

\luettelo{
§ Tässä luvussa tarkastellaan lyhyesti yleisesti toisen asteen tasokäyriä, joista ympyrä ja paraabeli on jo käsitelty edellä
§ (kahden muuttujan) toisen asteen yhtälön määritelmä
§ joku tuttu esimerkki, vaikka paraabeli
§ esimerkkeinä yksi piste, kaksi suoraa, tyhjä
§ maininta siitä, että voi tulla myös ellipsi tai hyperbeli,
joista sitten joskus kirjoitetaan liite
}

KIITOS!}

\begin{tehtavasivu}

\subsubsection*{Opi perusteet}

\subsubsection*{Hallitse kokonaisuus}

\subsubsection*{Sekalaisia tehtäviä}

TÄHÄN TEHTÄVIÄ SIJOITTAMISTA ODOTTAMAAN

\end{tehtavasivu}