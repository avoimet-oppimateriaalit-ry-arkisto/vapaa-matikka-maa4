\section{Taustaa}

% ajatuksia analyyttisen geometrian taustoista

Analyyttinen geometria on silta geometrian ja algebran välillä. Klassisessa geometriassa kuvioita käsitellään kokonaisuuksina, tasossa oleskelevina olioina, mutta miltä maailma näyttää, jos jokainen kuvio jaetaan pisteisiin, ja jokaista pistettä ajatellaan yksilönä? Analyyttinen geometria lähtee ajatuksesta antaa jokaiselle tason pisteelle nimi, koordinaatti.

Miten nimeäminen pitäisi hoitaa? Yhtä oikeaa tapaa ei ole, mutta yksi luonnollisimmista on varmaankin käyttää tuttuja reaalilukuja. Osoittautuu, että jos tasoa alkaa nimeämään suoraan reaaliluvuilla, lopputulos ei ole kovin mielekäs. Reaaliluvuilla on suuruusjärjestys, mutta tason pisteille sellaista on suoraan vaikea mieltää. Tasossa voi kuitenkin ajatella suuruusjärjestyksen pysty- ja vaakasuunnassa, mistä saadaan idea nimetä pisteitä kahdella luvulla, reaalilukuparilla. Reaalilukuparia, pisteen koordinaatteja, merkitään yleensä muodossa $(x,y)$, missä $x$ ja $y$ ovat siis reaalilukuja. Ensimmäistä lukua/koordinaattia kutsutaan yleensä $x$-koordinaatiksi ja toista lukua/koordinaattia $y$-koordinaatiksi, mutta nimet vaihtelevat sen mukaan, miten nimet on sijoitettu tasoon.

Miten nimeäminen pitäisi hoitaa? Edelleen, tyylejä on monia, mutta seuraava vaatimus tuottaa melko mukavia tuloksia: pisteillä toinen koordinaateista on sama, jos ja vain jos pisteet ovat samalla suoralla.